%%%%%%%%%%%%%%%%%%%%%%%%%%%%%%%%%%%%%%%%%%%%
%%%%%%%%%%%%%%%%%%%%%%%%%%%%%%%%%%%%%%%%%%%%
% Import preamble
\input{preamble_slides.tex}
\input{abbreviations.tex}
%%%%%%%%%%%%%%%%%%%%%%%%%%%%%%%%%%%%%%%%%%%%
%%%%%%%%%%%%%%%%%%%%%%%%%%%%%%%%%%%%%%%%%%%%
% Define title


%\title{Climate survey - France}
%\author[OECD]{ \\  Antoine Dechezleprêtre, Adrien Fabre, Bluebery Planterose, Stefanie Stantcheva \\ (OECD) }
%
%              \date{}
              
\begin{document}



%%%%%%%%%%%%%%%%%%%%%%%%%%%%%%%%%%%%%%%%%%%%
%%%%%%%%%%%%%%%%%%%%%%%%%%%%%%%%%%%%%%%%%%%%
% Begin of the document
% TITLE PAGE

\begin{frame}
\thispagestyle{empty}
\begin{center}
\begin{LARGE}
\textcolor{blue}{Climate survey - France}
\end{LARGE}

\vspace{1cm}


Antoine Dechezleprêtre, Adrien Fabre, Bluebery Planterose, Stefanie Stantcheva 

\vspace{-0.3cm}
(OECD)

\DTMlangsetup{showdayofmonth=false}
\textit{\today} 

\end{center}

\bigskip


\end{frame}

\begin{frame}{What do people think about climate?}
\bbs
\ip Climate policies have been difficult to pass.
\ip The design of climate policies needs to account for the political economy and public acceptability.
\ip \textcolor{magenta}{Resistance to climate policies} arises largely from:
\bbs
\ip Legitimate concerns about \textcolor{magenta}{distributional and lifestyle impacts}
\ip \textcolor{magenta}{Misconceptions} about the impacts of climate change and the effects of climate policies on the economy and the environment
\ee 
\ip Addressing concerns and misconceptions may be difficult, as they are influenced by personal attributes, country specificities and political views.
\ee
\end{frame}
%\begin{frame}{What do people think about climate?}
%\bbs
%\item What are the \textcolor{magenta}{mental models} people use to think about tax policy?
%\bbs 
%\item What do they \textcolor{magenta}{know}? How do they \textcolor{magenta}{reason}? % We are interested in what people know, but not only. 
%\ee
%\item {\small $\text{Desired tax policy} = f(\text{perceived efficiency effects}, \text{perceived distributional impacts}, \text{fairness considerations}, X_1, X_2, ...)$} 
%\item Why is understanding reasoning important? 
%\bbs
%\item Advantages of a more structural approach to policy views, over reduced-form approach. 
%\item Heterogeneity (even if same overall policy view). Where does disagreement lie?
%\item Identifying (correctable) gaps in knowledge or inconsistent reasoning. 
%\item Where is intervention needed versus not (e.g.: misperception of distributional impacts vs. fairness concerns)?
%\item Can we improve the policy debate with better understanding of economic policies?
%\ee
%\ee
%\end{frame}

\section{Questionnaire}

\begin{frame}{Questionnaire}
\vspace{.2cm}
%\makebox[\textwidth][c]{
%\begin{itemize}[<+>]
\only<+>{\makebox[\textwidth][c]{ \includegraphics[width=.9\textwidth]{../figures/survey_flow-1.pdf}}}
\only<+>{\makebox[\textwidth][c]{ \includegraphics[width=.9\textwidth]{../figures/survey_flow-2.pdf}}}
\only<+>{\makebox[\textwidth][c]{ \includegraphics[width=.9\textwidth]{../figures/survey_flow-3.pdf}}}
\only<+>{\makebox[\textwidth][c]{ \includegraphics[width=.9\textwidth]{../figures/survey_flow-4.pdf}}}
\only<+>{\makebox[\textwidth][c]{ \includegraphics[width=.9\textwidth]{../figures/survey_flow-5.pdf}}}
\only<+>{\makebox[\textwidth][c]{ \includegraphics[width=.9\textwidth]{../figures/survey_flow-6.pdf}}}
\only<+>{\makebox[\textwidth][c]{ \includegraphics[width=.9\textwidth]{../figures/survey_flow-7.pdf}}}
\only<+>{\makebox[\textwidth][c]{ \includegraphics[width=.9\textwidth]{../figures/survey_flow-8.pdf}}}
\only<+>{\makebox[\textwidth][c]{ \includegraphics[width=.9\textwidth]{../figures/survey_flow-9.pdf}}}
\only<+>{\makebox[\textwidth][c]{ \includegraphics[width=.9\textwidth]{../figures/survey_flow-10.pdf}}}
%\end{itemize}
%}
\end{frame}

\section{Sample quality}

\begin{table}%[h!]
	%\caption{Sample Characteristics -- France}
	\begin{center}
		\scalebox{0.7}{\input{"../tables/FR/summary_stat_fr"}}
	\end{center}
\end{table}	

\section{Descriptive statistics}



%\begin{frame}[label=fig_misp_top1_professions]{Perceived Composition of the Top 1\%: \\ so many entrepreneurs, scientists, government, teachers, arts, media \& sports!}
%\vspace{.5cm}
%\begin{centering}
%	\includegraphics[width=\textwidth]{\FigurePath/Income/misp_top1} 
%\end{centering}
%\end{frame}
%
%
%\begin{frame}[label = tab_knowledge]
%\smallskip
%\begin{table}
%\caption*{\textbf{Misperceptions about the Income Tax}}
%\vspace{-.5cm}
%\begin{center}
%\scalebox{0.42}{\input{\TablePath/Income/knowledge_income_taxslides.tex}}
%\end{center}
%\end{table}
%
%\vspace{-.2cm}
%\begin{table}
%\caption*{\textbf{Misperceptions about the Estate Tax}}
%\vspace{-.5cm}
%\begin{center}
%\scalebox{0.45}{\input{\TablePath/Estate/knowledge_estate_taxslides.tex}}
%\end{center}
%\end{table}
%\vspace{-0.2cm}
%\begin{footnotesize}
%Conditional on income, Republicans' self-perceived social class is higher. \hyperlink{piechart_socialclass}{\beamergotobutton{Social-class}}
%\end{footnotesize}
%\end{frame}

%%%%%%%%%%%%%%%%%%%%%%%%%%%%%%%%%%%%%%%%%%%%%%%%%%%%%%%%%%
%%%%%%%%%%%%%%%%%%%%%%%%%%%%%%%%%%%%%%%%%%%%%%%%%%%%%%%%%%
%%%%%%%%%%%%%%%%%%%%%%%%%%%%%%%%%%%%%%%%%%%%%%%%%%%%%%%%%%
%\section{Appendix}
%%%%%%%%%%%%%%%%%%%%%%%%%%%%%%%%%%%%%%%%%%%%%%%%%%%%%%%%%%
%%%%%%%%%%%%%%%%%%%%%%%%%%%%%%%%%%%%%%%%%%%%%%%%%%%%%%%%%%
%%%%%%%%%%%%%%%%%%%%%%%%%%%%%%%%%%%%%%%%%%%%%%%%%%%%%%%%%%







\end{document}













