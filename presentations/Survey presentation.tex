%%%%%%%%%%%%%%%%%%%%%%%%%%%%%%%%%%%%%%%%%%%%
%%%%%%%%%%%%%%%%%%%%%%%%%%%%%%%%%%%%%%%%%%%%
% Import preamble
\input{preamble_slides.tex}
\input{abbreviations.tex}
%%%%%%%%%%%%%%%%%%%%%%%%%%%%%%%%%%%%%%%%%%%%
%%%%%%%%%%%%%%%%%%%%%%%%%%%%%%%%%%%%%%%%%%%%
% Define title


%\title{Climate survey - France}
%\author[OECD]{ \\  Antoine Dechezleprêtre, Adrien Fabre, Bluebery Planterose, Stefanie Stantcheva \\ (OECD) }
%
%              \date{}
              
\begin{document}



%%%%%%%%%%%%%%%%%%%%%%%%%%%%%%%%%%%%%%%%%%%%
%%%%%%%%%%%%%%%%%%%%%%%%%%%%%%%%%%%%%%%%%%%%
% Begin of the document
% TITLE PAGE

\begin{frame}
\thispagestyle{empty}
\begin{center}
\begin{LARGE}
\textcolor{blue}{Climate survey - France}
\end{LARGE}

\vspace{1cm}


Laurence Boone, Antoine Dechezleprêtre, Adrien Fabre, Tobias Kruse, Bluebery Planterose, Ana Sanchez-Chico, Stefanie Stantcheva \\

%\vspace{-0.3cm}
OECD/CAE \\

\DTMlangsetup{showdayofmonth=false}
\textit{\today} 

\end{center}

\bigskip

\end{frame}

\begin{frame}{Motivation}
\bbs
\ip Climate policies have been difficult to pass.
\ip The design of climate policies needs to account for the political economy and public acceptability.
\ip \textcolor{blue}{Resistance to climate policies} arises largely from:
\bbs
\ip Legitimate concerns about \textcolor{blue}{distributional and lifestyle impacts}
\ip \textcolor{blue}{Misconceptions} about the impacts of climate change and the effects of climate policies on the economy and the environment
\ee 
\ip Addressing concerns and misconceptions may be difficult, as they are influenced by personal attributes, country specificities and political views.
\ee
\end{frame}

\begin{frame}{Project objectives}
\bbs
\ip \textcolor{blue}{Overall goal:} contribute to construct country-specific advice on policies to deal with the transition to a low-carbon economy by understanding people’s perceptions about climate change and preferences over available climate policies.
\ip \textcolor{blue}{Research questions:}
\bbs
\ip How social attitudes, values, and perceptions drive support or opposition for climate policies across socio-economic groups?
\ip How social preferences on climate change mitigation policies differ between countries?
\ip How perceptions may change after receiving new information on the effects of policies/climate change (in a video format) and how it translates into beliefs and support?
\ee
\ee
\end{frame}

\begin{frame}{An international survey}
\bbvs
\ip \textcolor{blue}{Large-scale cross-country survey} to analyse attitudes on climate change and climate policies.
\ip Wide country coverage:
\bbvs
\ip \textcolor{blue}{20 countries} in all world regions, low-income as well as high-income, 
\ip \textcolor{blue}{covering three-quarters of global CO$_\text{2}$ emissions}, including 18 out of the 21 largest polluters.% 72% of fossil CO2
\ee
\ee
\makebox[\textwidth][c]{ \includegraphics[width=.75\textwidth]{"../figures/Country_coverage"}}
\end{frame}

\begin{frame}{Method: Large-scale Social Economics Surveys and Experiments}
\bb
\ip \textcolor{magenta}{Surveys are a key tool:}
\bbs 
\ip Some aspects cannot be perceived in other data, not matter how good they are: Perceptions, attitudes, knowledge, views.
\ip Revealed preference with observational data has limits (data and assumptions required).

\ip Unlike old-style surveys (that measure variables now better captured in admin data) or opinion polls (less rigorous, often partial, without randomized treatments). 
\ip New generation surveys: Customizable, controllable, interactive.
\ee
\ip This study: $\approx$ 2,000 respondents per survey, broadly representative of the country, done through commercial survey companies. 
\ee
\end{frame}

\begin{frame}{Improvements upon existing research}
\bbs
\ip \textcolor{blue}{Wide scope}: past surveys are typically limited to a single (developed) country, focus on carbon pricing,
and existing international surveys include only very general questions.
\bbs
\ip For reviews of existing studies: Carattini et al. 2018, Drews and van den Bergh, 2016
\ee 
\ip \textcolor{blue}{Cross-country comparability}: having an international questionnaire informs whether differences in people’s attitudes are driven by survey characteristics (e.g. format and phrasing) or by true cross-country differences.
\ip \textcolor{blue}{Incentive compatibility}: we offer an incentive relying on an actual payment and propose tangible actions.
\ip \textcolor{blue}{Causal evidence}: we document effects of informational treatments in a video format, whereas cross-country evidence remains largely descriptive and national surveys use less effective treatments.
\ee
\end{frame}


%\begin{frame}{Main research questions}
%\bbs
%\ip \textcolor{blue}Drivers of policy support}: reveal how socio-demographics, values, and perceptions drive
%support or opposition for climate policies.
%\ip \textcolor{blue}{Cross-country comparisons}: analyze how social preferences on climate change
%mitigation policies differ between countries.
%\ip \textcolor{blue}{Effects of informational treatments}: understand how perceptions may change after
%receiving new information on the effects of policies/climate change (in a video
%format) and how it translates into beliefs and support.
%\ee
%\end{frame}

\begin{frame}{Informational treatments}
\bbs
\ip Treatments consist in one or two 2--5 min video either informing the respondent about
\textcolor{blue}{three main climate policies}, or detailing the \textcolor{blue}{impacts of climate change in their country}.
\ip \textcolor{magenta}{Goal}: understand \textcolor{blue}{how perceptions may change after
receiving new information} on the effects of policies/climate change (in a video
format) and how it translates into policy support.
\ip Will contribute to ongoing academic debate on importance of information for the acceptance of climate policies (e.g. Kahan, 2015; Sunstein et al., 2017):
\bbvs
\ip What do people really learn when information is provided?
\ip Do they accept the info or can it backfire?
\ip How does the effect interact with political values, previous knowledge, socio-demographics, country?
\ee
\ee
\end{frame}

\section{Questionnaire (for France)}

\begin{frame}{Questionnaire}
\vspace{.2cm}
%\makebox[\textwidth][c]{
%\begin{itemize}[<+>]
\only<+>{\makebox[\textwidth][c]{ \includegraphics[width=.9\textwidth]{../figures/survey_flow/survey_flow_new-1.pdf}}}
\only<+>{\makebox[\textwidth][c]{ \includegraphics[width=.9\textwidth]{../figures/survey_flow/survey_flow_new-2.pdf}}}
\only<+>{\makebox[\textwidth][c]{ \includegraphics[width=.9\textwidth]{../figures/survey_flow/survey_flow_new-3.pdf}}}
\only<+>{\makebox[\textwidth][c]{ \includegraphics[width=.9\textwidth]{../figures/survey_flow/survey_flow_new-4.pdf}}}
\only<+>{\makebox[\textwidth][c]{ \includegraphics[width=.9\textwidth]{../figures/survey_flow/survey_flow_new-5.pdf}}}
\only<+>{\makebox[\textwidth][c]{ \includegraphics[width=.9\textwidth]{../figures/survey_flow/survey_flow_new-6.pdf}}}
\only<+>{\makebox[\textwidth][c]{ \includegraphics[width=.9\textwidth]{../figures/survey_flow/survey_flow_new-7.pdf}}}
\only<+>{\makebox[\textwidth][c]{ \includegraphics[width=.9\textwidth]{../figures/survey_flow/survey_flow_new-8.pdf}}}
\only<+>{\makebox[\textwidth][c]{ \includegraphics[width=.9\textwidth]{../figures/survey_flow/survey_flow_new-9.pdf}}}
\only<+>{\makebox[\textwidth][c]{ \includegraphics[width=.9\textwidth]{../figures/survey_flow/survey_flow_new-10.pdf}}}
\only<+>{\makebox[\textwidth][c]{ \includegraphics[width=.9\textwidth]{../figures/survey_flow/survey_flow_new-11.pdf}}}
%\end{itemize}
%}
\end{frame}

\begin{frame}{Video screenshots}
\vspace{.2cm}
%\makebox[\textwidth][c]{
%\begin{itemize}[<+>]
\only<+>{\makebox[\textwidth][c]{ \includegraphics[width=.9\textwidth]{../figures/survey_flow/survey_flow_new-4.pdf}}}
%\end{itemize}
%}
\end{frame}
\begin{frame}{Questionnaire}
\vspace{.2cm}
%\makebox[\textwidth][c]{
%\begin{itemize}[<+>]
\only<+>{\makebox[\textwidth][c]{ \includegraphics[width=.9\textwidth]{../figures/survey_flow/survey_flow_new-5.pdf}}}
\only<+>{\makebox[\textwidth][c]{ \includegraphics[width=.9\textwidth]{../figures/survey_flow/survey_flow_new-6.pdf}}}
\only<+>{\makebox[\textwidth][c]{ \includegraphics[width=.9\textwidth]{../figures/survey_flow/survey_flow_new-7.pdf}}}
\only<+>{\makebox[\textwidth][c]{ \includegraphics[width=.9\textwidth]{../figures/survey_flow/survey_flow_new-8.pdf}}}
\only<+>{\makebox[\textwidth][c]{ \includegraphics[width=.9\textwidth]{../figures/survey_flow/survey_flow_new-9.pdf}}}
\only<+>{\makebox[\textwidth][c]{ \includegraphics[width=.9\textwidth]{../figures/survey_flow/survey_flow_new-10.pdf}}}
\only<+>{\makebox[\textwidth][c]{ \includegraphics[width=.9\textwidth]{../figures/survey_flow/survey_flow_new-11.pdf}}}
%\end{itemize}
%}
\end{frame}

\section{Sample quality -- France}

%- Add more tables? Effect of misperception + heterogeneity (indices) for instance
%- Do we add something on the LDA ?%

%- In table underline some key lines ? Like in slide 36 here: https://scholar.harvard.edu/files/stantcheva/files/understanding_taxes_slides.pdf
%- Pour le moment la présentation est vraiment descriptive (liste les réponses à chaque question), mais est-ce que l'on veut faire ça ou avoir un narrative (à travers une research question à laquelle on veut réponse et en faisant des sections sur les résultats préliminaires que l'on a, plutôt que de grouper block par block - même si ça va sûrement être très proche)

\begin{frame}{Ensuring data quality}
\bbs
\ip 2,006 respondents selected through quotas that  \textcolor{blue}{ensure representativeness} along: \\ \textcolor{magenta}{gender, age, income, region, diploma, urban/rural}.
\ip \textcolor{blue}{All results are re-weighted} along quota variables (except rural/urban) to increase representativeness even further.
\ip \textcolor{blue}{Screening question} in the middle of the survey. 
\ip Appeal to people's social responsibility. 
\ip Warn that ``incoherent and \textcolor{blue}{rushed responses'' (< 11 min) are dismissed} and disqualified for monetary compensation.
\ip Record time spent on separate questions \& overall survey (median: 27 min).
\ip Ask for feedback post survey, whether felt survey was biased (\textcolor{magenta}{78\% find it unbiased}).
%\ip Check careless response patterns (clicking same ``middle'' answer).
\ee
\end{frame}

\begin{frame}{Representativeness of the Survey Sample}
\begin{table}[h!]
	%\caption{Sample Characteristics -- France}
	\begin{center}
		\scalebox{0.55}{\input{"../tables/FR/summary_stat_fr_prez"}}
	\end{center}
\end{table}	
\end{frame}

\end{document}