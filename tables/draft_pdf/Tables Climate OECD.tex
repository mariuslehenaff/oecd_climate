\documentclass{article}

%%% Packages: 
\usepackage{eurosym} % for euros symbol
\usepackage{amsfonts}
\usepackage{fancyhdr}
\usepackage[usenames,dvipsnames,svgnames,table]{xcolor}
\usepackage[hypertexnames=false]{hyperref} %This makes hyperref ``dumber'', and, hence, more robust! (otherwise sometimes the appendix links don't work).
\usepackage[pdftex]{graphicx}
\usepackage{amsmath, amsthm, amssymb, dsfont, amsfonts}
\usepackage[american]{babel}
\usepackage{color}
%\usepackage{subfig}
\usepackage{morefloats}
\usepackage{tabulary}
\usepackage{tabularx}
\usepackage{booktabs}
\usepackage{fullpage}
%\usepackage{bbm}
\usepackage{setspace}
\usepackage{float}
\usepackage{pdfpages}
\usepackage{lscape}
\usepackage{multirow}
\usepackage{array}
\usepackage{sectsty}
\usepackage{pdflscape}
\usepackage{placeins}
\usepackage[font={large,sc}]{caption}
\usepackage{comment}
\usepackage[margin=1in,headsep=.4in]{geometry}
\usepackage[normalem]{ulem}
\usepackage{natbib}
\usepackage{tikz}
\usepackage{tikzscale}
\usepackage{bibunits}
\usepackage{xr}
\usepackage[figuresright]{rotating}
\usepackage{subcaption}
\usepackage{caption}
\usepackage{makecell}
\usepackage{graphicx}
\usepackage{hyperref}
\usepackage{pdfpages}
\usepackage{afterpage}
\usepackage{eurosym}
\setcounter{MaxMatrixCols}{10}
\usepackage{ulem}
\renewcommand{\ULdepth}{1.8pt}

%%%%%%%%%%%%%%%%%%%%%%%%%%%%%%%%%%%%%%%%%%%%%%%%%%%%%%%%%
%% COLORS AND LINKS
\definecolor{dark-red}{rgb}{0.4,0.15,0.15}
\definecolor{dark-blue}{rgb}{0.15,0.15,0.4}
\definecolor{medium-blue}{rgb}{0,0,0.5}
\hypersetup{
 colorlinks, linkcolor={dark-red},
citecolor={dark-red}, urlcolor={dark-red}
}



%%%%%%%%%%%%%%%%%%%%%%%%%%%%%%%%%%%%%%%%%%%%%%%%%%%%%%%%%
%%% THEOREMS and PROPOSITIONS
\newtheorem{definit}{Definition}
\newtheorem{prop}{Proposition}
\newtheorem{cor}{Corollary}

\renewcommand{\topfraction}{0.9}
    \renewcommand{\bottomfraction}{0.8}
\setcounter{topnumber}{2}
\setcounter{bottomnumber}{2}
\setcounter{totalnumber}{4}
\setcounter{dbltopnumber}{2}
    \renewcommand{\dbltopfraction}{0.9}
    \renewcommand{\textfraction}{0.07}
    \renewcommand{\floatpagefraction}{0.7}
    \renewcommand{\dblfloatpagefraction}{0.7}

\newcommand{\sym}[1]{{#1}}

%%%%% PAGE LAYOUT 
\textwidth 6.5in
\textheight 8.84in
\setlength{\topmargin}{-0.3in}
\setlength{\oddsidemargin}{0.0in}
\setlength{\evensidemargin}{0.0in}
\setlength{\abovecaptionskip}{0pt}
\setlength{\belowcaptionskip}{5pt}
\setlength{\textfloatsep}{25pt}
\setlength{\intextsep}{5pt}

\captionsetup[table]{skip=-10pt}

%%%%%%%%%%%%%%%%%%%%%%%%%%%%%%%%%%%%%%%%%%%%%%%%%%%%%%%%%%%%%%%%%%
%%%%% TIKZ
\usetikzlibrary{er, positioning,decorations.pathmorphing,calc}
\tikzset{every entity/.style={draw=black, fill=white}}
\tikzset{comment/.style={draw=white, fill=white}}
%%%%%%%%%%%%%%%%%%%%%%%%%%%%%%%%%%%%%%%%%%%%%%%%%%%%%%%%%%%%%%%%%%


%%%%%%%%%%%%%%%%%%%%%%%%%%%%%%%%%%%%%%%%%%%%%%%%%%%%%%%%%%%%%%%%%% BIBLIOGRAPHY

%\bibliographystyle{chicago}




\usepackage{amsmath}



\begin{document}


\begin{LARGE}
	\begin{center}
		Preliminary Results – OECD Climate surveys	
	\end{center}
	
\end{LARGE}
	\tableofcontents
	\listoftables


\clearpage
\section{Pre-treatment}
\subsection{Political views and media consumption}


\begin{table}[h!]
	\caption{Political views}  \label{table polviews}
	\begin{center}
		\scalebox{0.7}{
\begin{tabular}{@{\extracolsep{5pt}}lccc} 
\\[-1.8ex]\hline 
\hline \\[-1.8ex] 
 & \multicolumn{3}{c}{Political views} \\ 
\cline{2-4} 
\\[-1.8ex] & Interest in politics & Environmental org. member & Relative is environmentalist \\ 
\hline \\[-1.8ex] 
 Control group mean & 0.839 & 0.153 & 0.169  \\ \hline \\[-1.8ex] race: White only & 0.017 & 0.006 & 0.017 \\ 
  & (0.042) & (0.036) & (0.038) \\ 
  & & & \\ 
 Male & 0.067$^{*}$ & 0.116$^{***}$ & 0.102$^{***}$ \\ 
  & (0.037) & (0.031) & (0.033) \\ 
  & & & \\ 
 Children & 0.044 & 0.102$^{***}$ & 0.156$^{***}$ \\ 
  & (0.039) & (0.033) & (0.035) \\ 
  & & & \\ 
 No college & $-$0.047 & 0.005 & $-$0.014 \\ 
  & (0.043) & (0.036) & (0.039) \\ 
  & & & \\ 
 status: Retired & 0.029 & 0.078 & 0.066 \\ 
  & (0.068) & (0.058) & (0.062) \\ 
  & & & \\ 
 status: Student & 0.154 & 0.048 & $-$0.067 \\ 
  & (0.122) & (0.103) & (0.110) \\ 
  & & & \\ 
 staths: Working & 0.030 & 0.065 & 0.082 \\ 
  & (0.060) & (0.050) & (0.054) \\ 
  & & & \\ 
 Income Q2 & 0.112$^{**}$ & $-$0.018 & $-$0.036 \\ 
  & (0.054) & (0.045) & (0.048) \\ 
  & & & \\ 
 Income Q3 & 0.027 & $-$0.003 & 0.004 \\ 
  & (0.055) & (0.046) & (0.050) \\ 
  & & & \\ 
 Income Q4 & 0.059 & 0.073 & 0.089$^{*}$ \\ 
  & (0.060) & (0.050) & (0.054) \\ 
  & & & \\ 
 age: 30-49 & $-$0.009 & $-$0.035 & $-$0.065 \\ 
  & (0.060) & (0.051) & (0.054) \\ 
  & & & \\ 
 age: 50-87 & $-$0.002 & $-$0.203$^{***}$ & $-$0.267$^{***}$ \\ 
  & (0.065) & (0.055) & (0.059) \\ 
  & & & \\ 
 vote: Biden & 0.291$^{***}$ & 0.078$^{*}$ & 0.073 \\ 
  & (0.053) & (0.045) & (0.048) \\ 
  & & & \\ 
 vote: Trump & 0.232$^{***}$ & 0.018 & 0.016 \\ 
  & (0.056) & (0.048) & (0.051) \\ 
  & & & \\ 
 Both treatments & $-$0.027 & 0.019 & 0.075 \\ 
  & (0.051) & (0.043) & (0.046) \\ 
  & & & \\ 
 Climate treatment only & $-$0.073 & $-$0.022 & 0.001 \\ 
  & (0.052) & (0.044) & (0.047) \\ 
  & & & \\ 
 Policy treatment only & $-$0.008 & 0.064 & 0.080$^{*}$ \\ 
  & (0.050) & (0.042) & (0.045) \\ 
  & & & \\ 
 Constant & 0.457$^{***}$ & $-$0.006 & 0.021 \\ 
  & (0.092) & (0.078) & (0.083) \\ 
  & & & \\ 
\hline \\[-1.8ex] 

Observations & 499 & 499 & 499 \\ 
\hline 
\hline \\[-1.8ex] 
\end{tabular} }
	\end{center}
	{\footnotesize Note: The dependent variables are indicator variables. The \textit{Interest in politics} variable equals one if the respondent is interested in politics ``A lot" or ``A great deal." The \textit{Environmental org. member} variable equals one if the respondent is a member of an environmental organization, the \textit{Relative is environmentalist} variable equals one if the respondent has any relatives who are environmentalists, and the \textit{Econ. Conservative} variable equals one if the respondent is``Conservative'' or ``Very conservative'' on economic policy matters.
	The \textit{race: White only} indicator variable equals one if the respondent's self reported race is only ``White." The regression includes controls for gender, having children and having completed a college degree. The three \textit{status} indicator variables indicate the difference in mean compared to a reference group of people not working (either unemployed or inactive). The \textit{status: Working} indicator variable includes respondents who self-reported being either ``Full-time employed", ``Part-time employed", or ``Self-employed". The three \textit{Income} indicator variables indicate difference in mean compared to a reference group of people in the first quartile of household's annual income in 2019 (i.e. income $<$ \textdollar 35,000). The four \textit{age} indicator variables indicate difference in mean compared to a reference group of people aged between 18 and 24. The two \textit{vote} indicator variables include either people who actually voted for the candidate in the 2020 Presidential election or who did not vote but indicate they would have voted for this candidate. They indicate difference in mean compared to a reference group of people who voted for -- or indicate they would have voted for -- another candidate than Biden or Trump.
	\newline  *p$<$0.1; **p$<$0.05; ***p$<$0.01}
\end{table}	

\begin{landscape}
	\begin{table}[h!]
	\caption{Position on political spectrum}
	\begin{center}
		\scalebox{0.6}{
\begin{tabular}{@{\extracolsep{5pt}}lcccccccccccc} 
\\[-1.8ex]\hline 
\hline \\[-1.8ex] 
 & \multicolumn{12}{c}{Political positions} \\ 
\cline{2-13} 
\\[-1.8ex] & Far Left & Left & Center & Right & Far Right & Liberal & Conservative & Humanist & Patriot & Apolitical & Environmentalist & Feminist \\ 
\hline \\[-1.8ex] 
 Control group mean & NaN & NaN & NaN & NaN & NaN & NaN & NaN & NaN & NaN & NaN & NaN & NaN  \\ \hline \\[-1.8ex] race: White only & 0.063$^{**}$ & 0.050 & 0.001 & $-$0.006 & 0.073$^{***}$ & $-$0.020 & $-$0.026 & 0.017 & 0.004 & 0.003 & $-$0.023 & 0.017 \\ 
  & (0.031) & (0.037) & (0.049) & (0.033) & (0.026) & (0.037) & (0.044) & (0.029) & (0.034) & (0.021) & (0.025) & (0.023) \\ 
  & & & & & & & & & & & & \\ 
 Male & 0.043 & $-$0.028 & 0.011 & 0.033 & 0.031 & $-$0.020 & 0.026 & $-$0.012 & 0.056$^{*}$ & 0.002 & 0.019 & $-$0.061$^{***}$ \\ 
  & (0.027) & (0.033) & (0.043) & (0.029) & (0.023) & (0.033) & (0.039) & (0.025) & (0.030) & (0.019) & (0.022) & (0.020) \\ 
  & & & & & & & & & & & & \\ 
 Children & 0.068$^{**}$ & 0.060$^{*}$ & 0.060 & 0.053$^{*}$ & 0.041$^{*}$ & 0.052 & $-$0.012 & 0.052$^{*}$ & $-$0.007 & 0.014 & 0.016 & $-$0.019 \\ 
  & (0.029) & (0.035) & (0.046) & (0.031) & (0.024) & (0.035) & (0.041) & (0.027) & (0.032) & (0.020) & (0.023) & (0.022) \\ 
  & & & & & & & & & & & & \\ 
 No college & $-$0.038 & 0.016 & $-$0.103$^{**}$ & 0.003 & $-$0.052$^{*}$ & $-$0.011 & 0.048 & 0.053$^{*}$ & 0.015 & $-$0.028 & $-$0.001 & 0.020 \\ 
  & (0.032) & (0.039) & (0.051) & (0.034) & (0.027) & (0.039) & (0.046) & (0.030) & (0.036) & (0.022) & (0.026) & (0.024) \\ 
  & & & & & & & & & & & & \\ 
 status: Retired & 0.036 & 0.088 & 0.031 & $-$0.030 & $-$0.030 & $-$0.029 & 0.062 & $-$0.053 & $-$0.072 & $-$0.072$^{**}$ & 0.005 & $-$0.003 \\ 
  & (0.051) & (0.063) & (0.083) & (0.055) & (0.044) & (0.063) & (0.074) & (0.048) & (0.058) & (0.036) & (0.042) & (0.039) \\ 
  & & & & & & & & & & & & \\ 
 status: Student & $-$0.047 & $-$0.057 & 0.171 & $-$0.039 & 0.00001 & $-$0.015 & $-$0.159 & 0.154$^{**}$ & $-$0.081 & $-$0.043 & 0.031 & 0.009 \\ 
  & (0.082) & (0.100) & (0.133) & (0.088) & (0.070) & (0.100) & (0.118) & (0.077) & (0.092) & (0.057) & (0.067) & (0.062) \\ 
  & & & & & & & & & & & & \\ 
 staths: Working & 0.025 & 0.095$^{*}$ & $-$0.036 & 0.004 & 0.068$^{*}$ & 0.024 & 0.033 & $-$0.005 & $-$0.048 & $-$0.043 & $-$0.013 & $-$0.006 \\ 
  & (0.044) & (0.053) & (0.071) & (0.047) & (0.037) & (0.053) & (0.063) & (0.041) & (0.049) & (0.031) & (0.036) & (0.033) \\ 
  & & & & & & & & & & & & \\ 
 Income Q2 & $-$0.041 & 0.002 & $-$0.017 & 0.056 & $-$0.062$^{*}$ & 0.001 & $-$0.006 & $-$0.031 & 0.032 & $-$0.040 & $-$0.030 & $-$0.022 \\ 
  & (0.041) & (0.050) & (0.066) & (0.044) & (0.035) & (0.050) & (0.059) & (0.038) & (0.046) & (0.028) & (0.033) & (0.031) \\ 
  & & & & & & & & & & & & \\ 
 Income Q3 & $-$0.058 & 0.048 & 0.001 & 0.062 & $-$0.083$^{**}$ & $-$0.023 & 0.035 & $-$0.037 & 0.024 & $-$0.023 & $-$0.011 & $-$0.050 \\ 
  & (0.043) & (0.052) & (0.069) & (0.046) & (0.036) & (0.052) & (0.062) & (0.040) & (0.048) & (0.030) & (0.035) & (0.032) \\ 
  & & & & & & & & & & & & \\ 
 Income Q4 & 0.005 & 0.088$^{*}$ & 0.007 & 0.153$^{***}$ & $-$0.103$^{***}$ & $-$0.027 & 0.041 & $-$0.038 & 0.074 & $-$0.046 & $-$0.050 & $-$0.039 \\ 
  & (0.043) & (0.052) & (0.069) & (0.046) & (0.036) & (0.052) & (0.061) & (0.040) & (0.048) & (0.030) & (0.035) & (0.032) \\ 
  & & & & & & & & & & & & \\ 
 age: 30-49 & $-$0.075$^{*}$ & $-$0.116$^{**}$ & $-$0.134$^{**}$ & $-$0.046 & $-$0.025 & $-$0.013 & $-$0.001 & 0.042 & $-$0.012 & 0.059$^{**}$ & 0.029 & 0.002 \\ 
  & (0.042) & (0.051) & (0.067) & (0.045) & (0.036) & (0.051) & (0.060) & (0.039) & (0.047) & (0.029) & (0.034) & (0.032) \\ 
  & & & & & & & & & & & & \\ 
 age: 50-87 & $-$0.233$^{***}$ & $-$0.132$^{**}$ & $-$0.085 & $-$0.062 & $-$0.035 & $-$0.030 & 0.050 & $-$0.018 & 0.034 & 0.053 & 0.008 & $-$0.014 \\ 
  & (0.046) & (0.056) & (0.074) & (0.050) & (0.039) & (0.056) & (0.066) & (0.043) & (0.052) & (0.032) & (0.038) & (0.035) \\ 
  & & & & & & & & & & & & \\ 
 vote: Biden & 0.134$^{***}$ & 0.150$^{***}$ & $-$0.149$^{**}$ & $-$0.102$^{**}$ & 0.047 & 0.216$^{***}$ & $-$0.055 & 0.057 & $-$0.013 & $-$0.053$^{*}$ & 0.013 & 0.055$^{*}$ \\ 
  & (0.039) & (0.047) & (0.063) & (0.042) & (0.033) & (0.047) & (0.056) & (0.036) & (0.044) & (0.027) & (0.032) & (0.029) \\ 
  & & & & & & & & & & & & \\ 
 vote: Trump & 0.043 & $-$0.064 & $-$0.421$^{***}$ & 0.013 & 0.108$^{***}$ & 0.002 & 0.410$^{***}$ & 0.055 & 0.071 & $-$0.077$^{***}$ & 0.002 & $-$0.001 \\ 
  & (0.041) & (0.050) & (0.067) & (0.044) & (0.035) & (0.050) & (0.059) & (0.039) & (0.046) & (0.029) & (0.034) & (0.031) \\ 
  & & & & & & & & & & & & \\ 
 Both treatments & 0.023 & 0.018 & $-$0.055 & $-$0.077$^{*}$ & 0.017 & $-$0.084$^{*}$ & $-$0.013 & 0.061$^{*}$ & $-$0.006 & $-$0.038 & 0.010 & 0.023 \\ 
  & (0.038) & (0.046) & (0.061) & (0.041) & (0.032) & (0.046) & (0.055) & (0.036) & (0.043) & (0.026) & (0.031) & (0.029) \\ 
  & & & & & & & & & & & & \\ 
 Climate treatment only & 0.068$^{*}$ & 0.043 & $-$0.021 & $-$0.053 & 0.034 & $-$0.116$^{***}$ & $-$0.007 & 0.080$^{**}$ & 0.024 & 0.020 & 0.013 & $-$0.002 \\ 
  & (0.036) & (0.044) & (0.059) & (0.039) & (0.031) & (0.044) & (0.052) & (0.034) & (0.041) & (0.025) & (0.030) & (0.028) \\ 
  & & & & & & & & & & & & \\ 
 Policy treatment only & 0.002 & $-$0.059 & $-$0.020 & $-$0.048 & 0.061$^{*}$ & $-$0.108$^{**}$ & 0.003 & 0.012 & 0.061 & $-$0.012 & $-$0.015 & 0.007 \\ 
  & (0.038) & (0.046) & (0.061) & (0.041) & (0.032) & (0.046) & (0.054) & (0.035) & (0.042) & (0.026) & (0.031) & (0.029) \\ 
  & & & & & & & & & & & & \\ 
 wave: Pilote 2 & $-$0.092$^{***}$ & $-$0.008 & $-$0.029 & $-$0.008 & $-$0.016 & $-$0.004 & $-$0.089$^{**}$ & 0.009 & $-$0.023 & $-$0.019 & 0.009 & 0.034$^{*}$ \\ 
  & (0.026) & (0.032) & (0.043) & (0.029) & (0.023) & (0.032) & (0.038) & (0.025) & (0.030) & (0.018) & (0.022) & (0.020) \\ 
  & & & & & & & & & & & & \\ 
\hline \\[-1.8ex] 

Observations & 499 & 499 & 499 & 499 & 499 & 499 & 499 & 499 & 499 & 499 & 499 & 499 \\ 
\hline 
\hline \\[-1.8ex] 
\end{tabular} }
	\end{center}
	{\footnotesize Note: The dependent variables are indicator variables equal to one if the respondent defines herself as being part of the category. See notes under Table \ref{table polviews} for a description of the covariates.
	\newline *p$<$0.1; **p$<$0.05; ***p$<$0.01}
\end{table}	
\end{landscape}



\clearpage
\subsection{Energie Characteristics}



\begin{table}[h!]
	\caption{Main way of heating} \label{table heating}
	\begin{center}
		\scalebox{0.7}{
\begin{tabular}{@{\extracolsep{5pt}}lcccc} 
\\[-1.8ex]\hline 
\hline \\[-1.8ex] 
 & \multicolumn{4}{c}{At home} \\ 
\cline{2-5} 
\\[-1.8ex] & Electricity & Gas & Heating oil & Renewable \\ 
\hline \\[-1.8ex] 
 Mean & 0.479 & 0.37 & 0.064 & 0.04  \\ \hline \\[-1.8ex] race: White only & $-$0.023 & $-$0.029 & 0.008 & 0.030 \\ 
  & (0.053) & (0.051) & (0.026) & (0.021) \\ 
  & & & & \\ 
 Male & 0.019 & $-$0.078$^{*}$ & 0.020 & 0.015 \\ 
  & (0.046) & (0.045) & (0.023) & (0.018) \\ 
  & & & & \\ 
 Children & 0.026 & $-$0.007 & 0.050$^{**}$ & 0.011 \\ 
  & (0.049) & (0.048) & (0.025) & (0.020) \\ 
  & & & & \\ 
 No college & 0.095$^{*}$ & $-$0.070 & $-$0.009 & $-$0.008 \\ 
  & (0.055) & (0.053) & (0.027) & (0.022) \\ 
  & & & & \\ 
 status: Retired & 0.214$^{**}$ & $-$0.045 & $-$0.083$^{*}$ & 0.014 \\ 
  & (0.089) & (0.086) & (0.044) & (0.035) \\ 
  & & & & \\ 
 status: Student & $-$0.188 & 0.130 & 0.179$^{**}$ & $-$0.053 \\ 
  & (0.141) & (0.138) & (0.071) & (0.056) \\ 
  & & & & \\ 
 status: Working & 0.216$^{***}$ & $-$0.046 & $-$0.019 & $-$0.055$^{*}$ \\ 
  & (0.076) & (0.074) & (0.038) & (0.030) \\ 
  & & & & \\ 
 Income Q2 & $-$0.110 & 0.103 & 0.058 & $-$0.002 \\ 
  & (0.071) & (0.069) & (0.035) & (0.028) \\ 
  & & & & \\ 
 Income Q3 & $-$0.082 & 0.061 & $-$0.029 & 0.049$^{*}$ \\ 
  & (0.074) & (0.072) & (0.037) & (0.029) \\ 
  & & & & \\ 
 Income Q4 & $-$0.175$^{**}$ & 0.176$^{**}$ & 0.007 & 0.004 \\ 
  & (0.073) & (0.071) & (0.037) & (0.029) \\ 
  & & & & \\ 
 age: 30-49 & $-$0.125$^{*}$ & 0.076 & $-$0.033 & $-$0.014 \\ 
  & (0.072) & (0.070) & (0.036) & (0.028) \\ 
  & & & & \\ 
 age: 50-87 & $-$0.323$^{***}$ & 0.266$^{***}$ & 0.043 & $-$0.029 \\ 
  & (0.079) & (0.077) & (0.040) & (0.031) \\ 
  & & & & \\ 
 vote: Biden & 0.158$^{**}$ & $-$0.111$^{*}$ & 0.041 & $-$0.020 \\ 
  & (0.066) & (0.065) & (0.033) & (0.026) \\ 
  & & & & \\ 
 vote: Trump & 0.061 & $-$0.036 & 0.036 & $-$0.001 \\ 
  & (0.071) & (0.069) & (0.035) & (0.028) \\ 
  & & & & \\ 
 wave: Pilote 2 & $-$0.045 & 0.086$^{*}$ & $-$0.030 & 0.005 \\ 
  & (0.045) & (0.044) & (0.023) & (0.018) \\ 
  & & & & \\ 
\hline \\[-1.8ex] 

Observations & 497 & 497 & 497 & 497 \\ 
\hline 
\hline \\[-1.8ex] 
\end{tabular} }
	\end{center}
	{\footnotesize Note: The dependent variables are indicator variables equal to one if the respondent indicates that the source of energy was her main way of heating at home. The \textit{Renewable} variable corresponds to the answer ``Wood, solar, geothermal, or heat pump.". The \textit{Heating expenses \textdollar 200+} variable is an indicator variable equal to one if the respondent indicates paying more than USD200 per month for heating expenses.
	See notes under Table \ref{table polviews} for a description of the covariates.
	\newline  *p$<$0.1; **p$<$0.05; ***p$<$0.01}
\end{table}	

\begin{table}[h!]
	\caption{Consumption and GHG}
	\begin{center}
		\scalebox{0.7}{
\begin{tabular}{@{\extracolsep{5pt}}lccc} 
\\[-1.8ex]\hline 
\hline \\[-1.8ex] 
 & \multicolumn{3}{c}{Own household} \\ 
\cline{2-4} 
\\[-1.8ex] & Km driven (2019) & Flights (2015-19) & Rarely eat beef \\ 
\hline \\[-1.8ex] 
 Mean & 18387.393 & 10.108 & 0.292  \\
Observations & 190 & 191 & 191 \\ 
\hline 
\hline \\[-1.8ex] 
\end{tabular} }
	\end{center}
	{\footnotesize Note: The \textit{Gas expenses \textdollar 125+} variable is an indicator variable equal to one if the respondent indicates spending more than USD125 per motnh for gas expenses. The \textit{Flights (2015-19) +5} variable equals one if the respondent indicates having taken more than 5 round-trip flights between 2015 and 2019 included. The \textit{Often eat beef} variable is an indicator variable equal to one if the respondent indicates that she eats beef weekly or daily.}
\end{table}	

\begin{landscape}
\begin{table}[h!]
	\caption{Main mode of transports used}
	\begin{center}
		\scalebox{0.6}{
\begin{tabular}{@{\extracolsep{5pt}}lccccccccc} 
\\[-1.8ex]\hline 
\hline \\[-1.8ex] 
 & \multicolumn{9}{c}{Transports} \\ 
\cline{2-10} 
\\[-1.8ex] & Car/Bike (work) & Public (work) & Bicycle/Walk (work) & Car/Bike (shop) & Public (shop) & Bicycle/Walk (shop) & Car/Bike (leisure) & Public (leisure) & Bicycle/Walk (leisure) \\ 
\hline \\[-1.8ex] 
 Mean & 0.779 & 0.139 & 0.066 & 0.819 & 0.08 & 0.09 & 0.773 & 0.074 & 0.102  \\ \hline \\[-1.8ex] race: White only & 0.199$^{**}$ & $-$0.117 & $-$0.045 & 0.068 & $-$0.051 & 0.006 & 0.106 & 0.014 & $-$0.033 \\ 
  & (0.098) & (0.081) & (0.054) & (0.069) & (0.050) & (0.055) & (0.079) & (0.053) & (0.059) \\ 
  & & & & & & & & & \\ 
 Male & $-$0.052 & 0.040 & $-$0.031 & $-$0.076 & 0.041 & 0.011 & $-$0.205$^{***}$ & 0.072 & 0.096$^{*}$ \\ 
  & (0.089) & (0.074) & (0.049) & (0.059) & (0.043) & (0.047) & (0.067) & (0.045) & (0.050) \\ 
  & & & & & & & & & \\ 
 Children & 0.010 & $-$0.023 & 0.017 & $-$0.012 & 0.034 & $-$0.021 & $-$0.007 & 0.007 & 0.034 \\ 
  & (0.092) & (0.077) & (0.051) & (0.061) & (0.044) & (0.048) & (0.070) & (0.047) & (0.052) \\ 
  & & & & & & & & & \\ 
 No college & 0.053 & $-$0.030 & 0.038 & $-$0.031 & 0.052 & 0.018 & 0.027 & $-$0.004 & 0.015 \\ 
  & (0.106) & (0.088) & (0.059) & (0.067) & (0.049) & (0.053) & (0.077) & (0.052) & (0.057) \\ 
  & & & & & & & & & \\ 
 status: Retired & $-$0.038 & $-$0.043 & 0.042 & $-$0.038 & 0.062 & $-$0.032 & 0.098 & 0.013 & $-$0.096 \\ 
  & (0.197) & (0.164) & (0.109) & (0.113) & (0.082) & (0.089) & (0.127) & (0.085) & (0.094) \\ 
  & & & & & & & & & \\ 
 status: Student & $-$0.598$^{*}$ & $-$0.036 & 0.615$^{***}$ & $-$0.685$^{***}$ & 0.389$^{**}$ & 0.275 & $-$0.255 & 0.361$^{*}$ & $-$0.053 \\ 
  & (0.326) & (0.272) & (0.181) & (0.255) & (0.184) & (0.200) & (0.281) & (0.188) & (0.208) \\ 
  & & & & & & & & & \\ 
 status: Working & 0.030 & $-$0.007 & $-$0.034 & $-$0.090 & 0.032 & 0.037 & $-$0.034 & 0.091 & $-$0.048 \\ 
  & (0.176) & (0.147) & (0.098) & (0.109) & (0.079) & (0.086) & (0.122) & (0.082) & (0.090) \\ 
  & & & & & & & & & \\ 
 Income Q2 & 0.097 & 0.146 & $-$0.181$^{**}$ & 0.173$^{*}$ & $-$0.039 & $-$0.105 & 0.259$^{**}$ & $-$0.029 & $-$0.175$^{**}$ \\ 
  & (0.149) & (0.124) & (0.083) & (0.091) & (0.066) & (0.072) & (0.108) & (0.072) & (0.080) \\ 
  & & & & & & & & & \\ 
 Income Q3 & 0.174 & $-$0.061 & $-$0.085 & 0.220$^{**}$ & $-$0.083 & $-$0.120$^{*}$ & 0.258$^{**}$ & $-$0.026 & $-$0.181$^{**}$ \\ 
  & (0.138) & (0.115) & (0.077) & (0.086) & (0.062) & (0.067) & (0.100) & (0.067) & (0.074) \\ 
  & & & & & & & & & \\ 
 Income Q4 & 0.144 & 0.051 & $-$0.141$^{*}$ & 0.175$^{*}$ & $-$0.006 & $-$0.133$^{*}$ & 0.175$^{*}$ & $-$0.007 & $-$0.125 \\ 
  & (0.140) & (0.116) & (0.077) & (0.090) & (0.065) & (0.071) & (0.103) & (0.069) & (0.076) \\ 
  & & & & & & & & & \\ 
 age: 30-49 & 0.100 & $-$0.276$^{*}$ & 0.169 & 0.080 & $-$0.228$^{**}$ & 0.135 & 0.166 & 0.013 & $-$0.262$^{**}$ \\ 
  & (0.185) & (0.154) & (0.103) & (0.146) & (0.106) & (0.115) & (0.161) & (0.108) & (0.119) \\ 
  & & & & & & & & & \\ 
 age: 50-87 & 0.094 & $-$0.186 & 0.061 & 0.179 & $-$0.319$^{***}$ & 0.101 & 0.280$^{*}$ & $-$0.022 & $-$0.360$^{***}$ \\ 
  & (0.196) & (0.164) & (0.109) & (0.150) & (0.109) & (0.118) & (0.164) & (0.110) & (0.121) \\ 
  & & & & & & & & & \\ 
 vote: Biden & $-$0.030 & 0.210$^{**}$ & $-$0.121$^{*}$ & 0.153$^{*}$ & $-$0.017 & $-$0.095 & 0.008 & 0.096 & $-$0.065 \\ 
  & (0.122) & (0.101) & (0.068) & (0.084) & (0.061) & (0.066) & (0.095) & (0.064) & (0.070) \\ 
  & & & & & & & & & \\ 
 vote: Trump & $-$0.067 & 0.174 & $-$0.095 & 0.093 & 0.005 & $-$0.079 & $-$0.002 & 0.042 & $-$0.034 \\ 
  & (0.139) & (0.115) & (0.077) & (0.090) & (0.065) & (0.071) & (0.103) & (0.069) & (0.076) \\ 
  & & & & & & & & & \\ 
 PT not available & 0.047 & $-$0.127 & 0.074 & 0.083 & $-$0.054 & $-$0.030 & $-$0.023 & 0.007 & 0.017 \\ 
  & (0.097) & (0.081) & (0.054) & (0.061) & (0.044) & (0.048) & (0.068) & (0.045) & (0.050) \\ 
  & & & & & & & & & \\ 
 Constant & 0.458$^{**}$ & 0.288 & 0.182 & 0.482$^{***}$ & 0.343$^{***}$ & 0.147 & 0.399$^{**}$ & $-$0.086 & 0.580$^{***}$ \\ 
  & (0.227) & (0.189) & (0.126) & (0.167) & (0.121) & (0.131) & (0.186) & (0.124) & (0.137) \\ 
  & & & & & & & & & \\ 
\hline \\[-1.8ex] 

Observations & 118 & 118 & 118 & 184 & 184 & 184 & 174 & 174 & 174 \\ 
\hline 
\hline \\[-1.8ex] 
\end{tabular} }
	\end{center}
	{\footnotesize Note: The dependent variables are indicator variables equal to one if the respondent indicates she mainly uses the mode of transport for the activity in brackets. For instance, the \textit{Car/Bike (work)} variable equals one if the respondent mainly uses a car or a motorbike to go to work, school of university. \textit{Public} variables stand for ``Public Transports", \textit{Bicycle/Walk} stands for ``Walking or cycling", \textit{shop} for ``Grocery shopping" and \textit{leisure} for ``Leisure (excluding holidays)."
	See note under Table \ref{table polviews} for a description of the covariates. \textit{PT not available} is an indicator variable equal to 1, if the availability of public transports where the respondent lives is ``Very poor'' or ``Poor.''
	\newline *p$<$0.1; **p$<$0.05; ***p$<$0.01}	
\end{table}	
\end{landscape}



\clearpage
\section{Post-treatment}
\subsection{Climate change (knowledge)}



\begin{table}[h!]
	\caption{Talks often about climate change}
	\begin{center}
		\scalebox{0.8}{
\begin{tabular}{@{\extracolsep{5pt}}lccc} 
\\[-1.8ex]\hline 
\hline \\[-1.8ex] 
\\[-1.8ex] & Never & Yearly & Monthly \\ 
\hline \\[-1.8ex] 
 Mean & 0.446 & 0.215 & 0.231  \\
Observations & 191 & 191 & 191 \\ 
\hline 
\hline \\[-1.8ex] 
\end{tabular} }
	\end{center}
	{\footnotesize Note: The variables are indicator variables. For instance, \textit{Never} equals one if the respondent never talks about climate change.}
\end{table}		

\begin{table}[h!]
	\caption{Climate change existence} \label{table exists}
	\begin{center}
		\scalebox{0.7}{
\begin{tabular}{@{\extracolsep{5pt}}lccc} 
\\[-1.8ex]\hline 
\hline \\[-1.8ex] 
\\[-1.8ex] & not a reality & mainly due to natural climate variability & mainly due to human activity \\ 
\hline \\[-1.8ex] 
 Mean & 0.068 & 0.275 & 0.582  \\ \hline \\[-1.8ex] race: White only & $-$0.041 & $-$0.019 & 0.034 \\ 
  & (0.026) & (0.045) & (0.046) \\ 
  & & & \\ 
 Male & $-$0.002 & $-$0.007 & 0.025 \\ 
  & (0.023) & (0.039) & (0.041) \\ 
  & & & \\ 
 Children & $-$0.017 & 0.027 & 0.058 \\ 
  & (0.025) & (0.042) & (0.043) \\ 
  & & & \\ 
 No college & 0.019 & $-$0.003 & $-$0.004 \\ 
  & (0.027) & (0.046) & (0.047) \\ 
  & & & \\ 
 status: Retired & 0.028 & 0.169$^{**}$ & $-$0.172$^{**}$ \\ 
  & (0.043) & (0.073) & (0.075) \\ 
  & & & \\ 
 status: Student & $-$0.002 & 0.123 & $-$0.036 \\ 
  & (0.076) & (0.130) & (0.135) \\ 
  & & & \\ 
 status: Working & 0.043 & 0.111$^{*}$ & $-$0.171$^{***}$ \\ 
  & (0.037) & (0.063) & (0.066) \\ 
  & & & \\ 
 Income Q2 & $-$0.033 & 0.012 & 0.039 \\ 
  & (0.033) & (0.057) & (0.059) \\ 
  & & & \\ 
 Income Q3 & $-$0.025 & 0.050 & 0.027 \\ 
  & (0.034) & (0.059) & (0.061) \\ 
  & & & \\ 
 Income Q4 & 0.048 & 0.038 & $-$0.004 \\ 
  & (0.037) & (0.064) & (0.066) \\ 
  & & & \\ 
 age: 30-49 & 0.056 & 0.013 & $-$0.153$^{**}$ \\ 
  & (0.037) & (0.064) & (0.066) \\ 
  & & & \\ 
 age: 50-87 & 0.046 & 0.011 & $-$0.124$^{*}$ \\ 
  & (0.041) & (0.069) & (0.072) \\ 
  & & & \\ 
 vote: Biden & $-$0.011 & $-$0.169$^{***}$ & 0.327$^{***}$ \\ 
  & (0.033) & (0.057) & (0.058) \\ 
  & & & \\ 
 vote: Trump & 0.097$^{***}$ & 0.188$^{***}$ & $-$0.198$^{***}$ \\ 
  & (0.035) & (0.060) & (0.062) \\ 
  & & & \\ 
 Constant & 0.004 & 0.148$^{*}$ & 0.656$^{***}$ \\ 
  & (0.051) & (0.087) & (0.090) \\ 
  & & & \\ 
\hline \\[-1.8ex] 

Observations & 499 & 499 & 499 \\ 
\hline 
\hline \\[-1.8ex] 
\end{tabular} }
	\end{center}
	{\footnotesize Note: The dependent variables are indicator variables. The \textit{is real} variable equals one if the respondent believes climate change is real. The \textit{mostly due to human activity} variable equals one if the respondent thinks ``A lot'' or ``Most'' of climate change is due to human activity. The \textit{important problem} variable equals one if the respondent ``Agrees'' or ``Strongly agress'' that climate change is an important problem. The \textit{knowledgeable} variable equals one if the respondent consider herself ``A lot'' or ``A great deal'' knowledgeable about climate change. See note under Table \ref{table polviews} for a description of the covariates. The three \textit{treatment} indicator variables indicate difference in mean compared to the control group (people who did not see any video).
	\newline *p$<$0.1; **p$<$0.05; ***p$<$0.01}
\end{table}		

\begin{table}[h!]
	\caption{Climate change knowledge}
	\begin{center}
		\scalebox{0.7}{
\begin{tabular}{@{\extracolsep{5pt}}lccc} 
\\[-1.8ex]\hline 
\hline \\[-1.8ex] 
\\[-1.8ex] & score GHG & not sufficient to halve GHG & score impacts \\ 
\hline \\[-1.8ex] 
 Mean & 0.772 & 0.448 & 0.221  \\ \hline \\[-1.8ex] race: White only & 0.034 & $-$0.010 & 0.031 \\ 
  & (0.022) & (0.025) & (0.021) \\ 
  & & & \\ 
 Male & 0.081$^{***}$ & $-$0.023 & 0.053$^{***}$ \\ 
  & (0.019) & (0.022) & (0.019) \\ 
  & & & \\ 
 Children & $-$0.043$^{**}$ & $-$0.150$^{***}$ & $-$0.024 \\ 
  & (0.021) & (0.024) & (0.020) \\ 
  & & & \\ 
 No college & 0.012 & $-$0.036 & $-$0.016 \\ 
  & (0.022) & (0.025) & (0.022) \\ 
  & & & \\ 
 status: Retired & 0.083$^{**}$ & 0.083$^{*}$ & 0.077$^{**}$ \\ 
  & (0.040) & (0.045) & (0.039) \\ 
  & & & \\ 
 status: Student & 0.009 & $-$0.127$^{**}$ & 0.005 \\ 
  & (0.055) & (0.064) & (0.054) \\ 
  & & & \\ 
 status: Working & $-$0.022 & 0.024 & 0.005 \\ 
  & (0.030) & (0.035) & (0.030) \\ 
  & & & \\ 
 Income Q2 & 0.068$^{**}$ & 0.007 & 0.006 \\ 
  & (0.029) & (0.033) & (0.028) \\ 
  & & & \\ 
 Income Q3 & 0.053$^{*}$ & 0.052 & $-$0.008 \\ 
  & (0.030) & (0.035) & (0.030) \\ 
  & & & \\ 
 Income Q4 & 0.074$^{**}$ & $-$0.008 & 0.016 \\ 
  & (0.031) & (0.036) & (0.030) \\ 
  & & & \\ 
 age: 25-34 & 0.083$^{**}$ & $-$0.062 & $-$0.088$^{**}$ \\ 
  & (0.037) & (0.043) & (0.036) \\ 
  & & & \\ 
 age: 35-49 & 0.007 & $-$0.030 & $-$0.079$^{**}$ \\ 
  & (0.037) & (0.043) & (0.036) \\ 
  & & & \\ 
 age: 50-64 & 0.008 & 0.123$^{***}$ & 0.042 \\ 
  & (0.040) & (0.046) & (0.039) \\ 
  & & & \\ 
 age: 65+ & $-$0.003 & 0.215$^{***}$ & 0.089$^{*}$ \\ 
  & (0.048) & (0.055) & (0.046) \\ 
  & & & \\ 
 vote: Biden & 0.014 & $-$0.005 & $-$0.020 \\ 
  & (0.032) & (0.037) & (0.031) \\ 
  & & & \\ 
 vote: Trump & $-$0.018 & 0.125$^{***}$ & $-$0.138$^{***}$ \\ 
  & (0.034) & (0.039) & (0.033) \\ 
  & & & \\ 
 Climate treatment only & 0.128$^{***}$ & $-$0.039 & 0.012 \\ 
  & (0.026) & (0.029) & (0.025) \\ 
  & & & \\ 
 Policy treatment only & 0.093$^{***}$ & $-$0.044 & 0.001 \\ 
  & (0.025) & (0.029) & (0.024) \\ 
  & & & \\ 
 Both treatments & 0.144$^{***}$ & $-$0.040 & 0.022 \\ 
  & (0.026) & (0.030) & (0.026) \\ 
  & & & \\ 
\hline \\[-1.8ex] 

Observations & 2,010 & 2,010 & 2,010 \\ 
\hline 
\hline \\[-1.8ex] 
\end{tabular} }
	\end{center}
	{\footnotesize Note: The \textit{score GHG} variable is a discrete variable in $[\![ 0;4 ]\!]$ reflecting knowledge about greenhouse gases: the higher the more knowledgeable. The \textit{not sufficient to halve GHG} is an indicator variable equal to one if the respondent thinks that cutting global greenhouse gas emissions by half would not be sufficient to stop temperatures from rising. The \textit{score impact} variable is a discrete variable in $[\![ 0;4 ]\!]$ reflecting knowledge about the impacts of climate change: the higher the more knowledgeable. See notes under Table \ref{table polviews} and Table \ref{table exists} for a description of the covariates.
	\newline *p$<$0.1; **p$<$0.05; ***p$<$0.01}
\end{table}

\begin{table}[h!]
	\caption{Comparisons of GHG emissions}
	\begin{center}
		\scalebox{0.7}{
\begin{tabular}{@{\extracolsep{5pt}}lccccc} 
\\[-1.8ex]\hline 
\hline \\[-1.8ex] 
\\[-1.8ex] & score transport & score food & score electricity & score region emissions & score per capita emissions \\ 
\hline \\[-1.8ex] 
 Mean & 0.723 & 0.827 & 0.715 & 0.406 & 0.388  \\ \hline \\[-1.8ex] race: White only & $-$0.001 & $-$0.010 & 0.087$^{***}$ & 0.010 & $-$0.036 \\ 
  & (0.024) & (0.021) & (0.023) & (0.026) & (0.026) \\ 
  & & & & & \\ 
 Female & 0.033 & 0.021 & $-$0.016 & 0.026 & 0.004 \\ 
  & (0.022) & (0.018) & (0.021) & (0.023) & (0.023) \\ 
  & & & & & \\ 
 Children & $-$0.027 & $-$0.050$^{**}$ & $-$0.076$^{***}$ & $-$0.0001 & 0.005 \\ 
  & (0.023) & (0.019) & (0.022) & (0.024) & (0.024) \\ 
  & & & & & \\ 
 No college & $-$0.021 & $-$0.030 & $-$0.044$^{*}$ & 0.046$^{*}$ & $-$0.034 \\ 
  & (0.025) & (0.021) & (0.023) & (0.026) & (0.026) \\ 
  & & & & & \\ 
 status: Retired & $-$0.012 & $-$0.003 & 0.040 & 0.039 & 0.052 \\ 
  & (0.044) & (0.037) & (0.041) & (0.047) & (0.047) \\ 
  & & & & & \\ 
 status: Student & 0.040 & 0.020 & 0.086 & 0.183$^{***}$ & 0.062 \\ 
  & (0.062) & (0.052) & (0.058) & (0.065) & (0.065) \\ 
  & & & & & \\ 
 status: Working & $-$0.020 & $-$0.017 & $-$0.014 & 0.041 & $-$0.035 \\ 
  & (0.034) & (0.029) & (0.032) & (0.036) & (0.036) \\ 
  & & & & & \\ 
 Income Q2 & 0.002 & 0.033 & 0.059$^{*}$ & $-$0.033 & $-$0.039 \\ 
  & (0.032) & (0.027) & (0.030) & (0.034) & (0.034) \\ 
  & & & & & \\ 
 Income Q3 & $-$0.005 & 0.018 & 0.111$^{***}$ & $-$0.018 & $-$0.015 \\ 
  & (0.034) & (0.028) & (0.032) & (0.036) & (0.036) \\ 
  & & & & & \\ 
 Income Q4 & 0.002 & 0.052$^{*}$ & 0.149$^{***}$ & $-$0.079$^{**}$ & 0.007 \\ 
  & (0.035) & (0.029) & (0.033) & (0.037) & (0.037) \\ 
  & & & & & \\ 
 age: 25-34 & $-$0.086$^{**}$ & 0.027 & 0.068$^{*}$ & 0.005 & $-$0.009 \\ 
  & (0.041) & (0.035) & (0.039) & (0.044) & (0.044) \\ 
  & & & & & \\ 
 age: 35-49 & $-$0.103$^{**}$ & 0.055 & 0.104$^{***}$ & $-$0.027 & $-$0.022 \\ 
  & (0.042) & (0.035) & (0.040) & (0.044) & (0.044) \\ 
  & & & & & \\ 
 age: 50-64 & $-$0.119$^{***}$ & 0.077$^{**}$ & 0.271$^{***}$ & $-$0.007 & $-$0.012 \\ 
  & (0.044) & (0.037) & (0.042) & (0.047) & (0.047) \\ 
  & & & & & \\ 
 age: 65+ & $-$0.181$^{***}$ & 0.152$^{***}$ & 0.323$^{***}$ & $-$0.038 & $-$0.006 \\ 
  & (0.053) & (0.045) & (0.050) & (0.056) & (0.056) \\ 
  & & & & & \\ 
 Left or Very left & $-$0.049$^{*}$ & $-$0.024 & 0.031 & 0.025 & $-$0.044$^{*}$ \\ 
  & (0.025) & (0.021) & (0.024) & (0.027) & (0.027) \\ 
  & & & & & \\ 
 Right or Very right & $-$0.003 & $-$0.026 & $-$0.026 & $-$0.064$^{**}$ & $-$0.060$^{**}$ \\ 
  & (0.026) & (0.022) & (0.025) & (0.028) & (0.028) \\ 
  & & & & & \\ 
 Center &  &  &  &  &  \\ 
  &  &  &  &  &  \\ 
  & & & & & \\ 
 Climate treatment only & 0.047$^{*}$ & $-$0.013 & 0.084$^{***}$ & 0.073$^{**}$ & 0.057$^{*}$ \\ 
  & (0.028) & (0.024) & (0.027) & (0.030) & (0.030) \\ 
  & & & & & \\ 
 Policy treatment only & 0.027 & $-$0.007 & 0.046$^{*}$ & 0.0005 & 0.045 \\ 
  & (0.028) & (0.024) & (0.026) & (0.030) & (0.030) \\ 
  & & & & & \\ 
 Both treatments & $-$0.006 & $-$0.022 & $-$0.001 & 0.015 & 0.087$^{***}$ \\ 
  & (0.029) & (0.025) & (0.028) & (0.031) & (0.031) \\ 
  & & & & & \\ 
\hline \\[-1.8ex] 

Observations & 1,886 & 1,884 & 1,874 & 2,010 & 2,010 \\ 
\hline 
\hline \\[-1.8ex] 
\end{tabular} }
	\end{center}
	{\footnotesize Note: The variables are discrete variables in $[\![ 0;3 ]\!]$ for \textit{score transport}, \textit{score food} and \textit{score electricity}, and in $[\![ 0;6 ]\!]$ for \textit{score region emissions} and \textit{score per capita emissions}. The variables are Kendall tau distances and reflect the number of errors when ranking items in terms of greenhouse gases emissions: the higher, the more wrong answers. For instance, a \textit{score food} of two means that the respondent's ranking of a beef steak, a serving of pasta or chicken wings in terms of greenhouse gas emissions is two swaps away from the actual ranking. See notes under Table \ref{table polviews} and Table \ref{table exists} for a description of the covariates.
	\newline *p$<$0.1; **p$<$0.05; ***p$<$0.01}
\end{table}		



\clearpage
\subsection{Climate change (attitudes and risks)}



\begin{landscape}
	\begin{table}[h!]
		\caption{Responsible party for CC}
		\begin{center}
			\scalebox{0.6}{
\begin{tabular}{@{\extracolsep{5pt}}lcccccccc} 
\\[-1.8ex]\hline 
\hline \\[-1.8ex] 
 & \multicolumn{8}{c}{Predominantly responsible for CC…} \\ 
\cline{2-9} 
\\[-1.8ex] & Each of us & The rich & Governments & Companies & Previous generations & Some foreign countries & Natural causes & Climate change is not a reality \\ 
\hline \\[-1.8ex] 
 Mean & 0.536 & 0.185 & 0.273 & 0.464 & 0.225 & 0.273 & 0.363 & 0.064  \\ \hline \\[-1.8ex] race: White only & 0.050 & $-$0.032 & $-$0.057 & $-$0.005 & 0.003 & 0.091$^{*}$ & $-$0.036 & $-$0.047$^{*}$ \\ 
  & (0.052) & (0.041) & (0.047) & (0.053) & (0.044) & (0.048) & (0.051) & (0.025) \\ 
  & & & & & & & & \\ 
 Male & 0.024 & 0.001 & $-$0.010 & $-$0.063 & $-$0.085$^{**}$ & 0.023 & 0.023 & $-$0.014 \\ 
  & (0.046) & (0.036) & (0.041) & (0.047) & (0.039) & (0.042) & (0.044) & (0.022) \\ 
  & & & & & & & & \\ 
 Children & $-$0.040 & 0.003 & 0.020 & $-$0.013 & 0.004 & $-$0.022 & 0.080$^{*}$ & $-$0.005 \\ 
  & (0.049) & (0.038) & (0.044) & (0.050) & (0.041) & (0.045) & (0.047) & (0.024) \\ 
  & & & & & & & & \\ 
 No college & $-$0.003 & $-$0.021 & 0.061 & 0.035 & $-$0.027 & $-$0.044 & $-$0.060 & 0.037 \\ 
  & (0.053) & (0.042) & (0.048) & (0.054) & (0.045) & (0.049) & (0.052) & (0.026) \\ 
  & & & & & & & & \\ 
 status: Retired & $-$0.064 & 0.054 & 0.117 & 0.041 & 0.011 & 0.002 & $-$0.010 & 0.021 \\ 
  & (0.085) & (0.067) & (0.076) & (0.086) & (0.072) & (0.077) & (0.082) & (0.041) \\ 
  & & & & & & & & \\ 
 status: Student & $-$0.325$^{**}$ & 0.132 & 0.278$^{**}$ & 0.132 & 0.206 & 0.096 & $-$0.002 & $-$0.057 \\ 
  & (0.152) & (0.120) & (0.136) & (0.154) & (0.129) & (0.139) & (0.148) & (0.073) \\ 
  & & & & & & & & \\ 
 status: Working & $-$0.041 & 0.047 & 0.095 & $-$0.009 & $-$0.024 & $-$0.080 & $-$0.101 & 0.018 \\ 
  & (0.074) & (0.058) & (0.066) & (0.075) & (0.063) & (0.067) & (0.072) & (0.036) \\ 
  & & & & & & & & \\ 
 Income Q2 & 0.065 & 0.023 & $-$0.093 & 0.008 & $-$0.080 & 0.049 & $-$0.057 & $-$0.046 \\ 
  & (0.067) & (0.052) & (0.060) & (0.068) & (0.056) & (0.061) & (0.065) & (0.032) \\ 
  & & & & & & & & \\ 
 Income Q3 & $-$0.021 & $-$0.027 & $-$0.068 & 0.049 & $-$0.044 & 0.099 & 0.069 & $-$0.047 \\ 
  & (0.069) & (0.054) & (0.062) & (0.070) & (0.058) & (0.062) & (0.067) & (0.033) \\ 
  & & & & & & & & \\ 
 Income Q4 & $-$0.067 & $-$0.052 & $-$0.051 & 0.022 & $-$0.109$^{*}$ & 0.162$^{**}$ & 0.025 & $-$0.001 \\ 
  & (0.074) & (0.058) & (0.067) & (0.075) & (0.063) & (0.068) & (0.072) & (0.036) \\ 
  & & & & & & & & \\ 
 age: 30-49 & $-$0.043 & $-$0.066 & $-$0.011 & $-$0.020 & 0.071 & 0.049 & $-$0.090 & $-$0.003 \\ 
  & (0.075) & (0.059) & (0.067) & (0.076) & (0.063) & (0.068) & (0.072) & (0.036) \\ 
  & & & & & & & & \\ 
 age: 50-87 & 0.009 & $-$0.157$^{**}$ & $-$0.084 & 0.016 & 0.010 & 0.050 & $-$0.085 & 0.020 \\ 
  & (0.081) & (0.064) & (0.073) & (0.082) & (0.068) & (0.074) & (0.079) & (0.039) \\ 
  & & & & & & & & \\ 
 vote: Biden & 0.131$^{**}$ & 0.105$^{**}$ & 0.175$^{***}$ & 0.169$^{**}$ & 0.155$^{***}$ & $-$0.030 & $-$0.085 & $-$0.090$^{***}$ \\ 
  & (0.066) & (0.052) & (0.059) & (0.067) & (0.056) & (0.060) & (0.064) & (0.032) \\ 
  & & & & & & & & \\ 
 vote: Trump & $-$0.134$^{*}$ & $-$0.013 & $-$0.008 & $-$0.052 & $-$0.001 & $-$0.050 & 0.132$^{*}$ & 0.029 \\ 
  & (0.070) & (0.055) & (0.063) & (0.071) & (0.059) & (0.064) & (0.068) & (0.034) \\ 
  & & & & & & & & \\ 
 Constant & 0.561$^{***}$ & 0.238$^{***}$ & 0.207$^{**}$ & 0.397$^{***}$ & 0.227$^{***}$ & 0.173$^{*}$ & 0.466$^{***}$ & 0.132$^{***}$ \\ 
  & (0.102) & (0.080) & (0.091) & (0.103) & (0.086) & (0.093) & (0.099) & (0.049) \\ 
  & & & & & & & & \\ 
\hline \\[-1.8ex] 

Observations & 499 & 499 & 499 & 499 & 499 & 499 & 499 & 499 \\ 
\hline 
\hline \\[-1.8ex] 
\end{tabular} }
		\end{center}
	{\footnotesize Note: The dependent variables are indicator variables equal to one if the respondent thinks the category is responsible ``A lot'' or ``A great deal'' for climate change. For instance, the variable \textit{Each of us} equals one if the respondent thinks that each of us are responsible ``A lot'' or ``A great deal''for climate change. See notes under Table \ref{table polviews} and Table \ref{table exists} for a description of the covariates.
	\newline *p$<$0.1; **p$<$0.05; ***p$<$0.01}
	\end{table}		
\end{landscape}

\begin{landscape}
	\begin{table}[h!]
		\caption{Possible to halt CC}
		\begin{center}
			\scalebox{0.55}{
\begin{tabular}{@{\extracolsep{5pt}}lccccc} 
\\[-1.8ex]\hline 
\hline \\[-1.8ex] 
\\[-1.8ex] & Human have no noticeable influence & Better live with CC than try to halt it & Should stop emissions, but not going to happen & Ambitious policies and awareness will succeed & Technologies and habits will suffice \\ 
\hline \\[-1.8ex] 
 Mean & 0.129 & 0.137 & 0.26 & 0.287 & 0.187  \\ \hline \\[-1.8ex] race: White only & $-$0.015 & 0.032 & 0.050 & 0.012 & $-$0.080$^{*}$ \\ 
  & (0.038) & (0.040) & (0.051) & (0.053) & (0.047) \\ 
  & & & & & \\ 
 Male & 0.057$^{*}$ & $-$0.040 & $-$0.046 & 0.027 & 0.001 \\ 
  & (0.033) & (0.034) & (0.044) & (0.045) & (0.040) \\ 
  & & & & & \\ 
 Children & 0.023 & 0.044 & $-$0.035 & $-$0.091$^{*}$ & 0.059 \\ 
  & (0.036) & (0.037) & (0.048) & (0.050) & (0.044) \\ 
  & & & & & \\ 
 No college & 0.008 & $-$0.021 & 0.011 & 0.005 & $-$0.003 \\ 
  & (0.040) & (0.042) & (0.053) & (0.055) & (0.049) \\ 
  & & & & & \\ 
 status: Retired & $-$0.080 & $-$0.001 & 0.005 & $-$0.078 & 0.155$^{*}$ \\ 
  & (0.066) & (0.068) & (0.088) & (0.090) & (0.080) \\ 
  & & & & & \\ 
 status: Student & $-$0.170 & 0.073 & 0.307$^{**}$ & $-$0.214 & 0.004 \\ 
  & (0.104) & (0.107) & (0.138) & (0.142) & (0.126) \\ 
  & & & & & \\ 
 status: Working & $-$0.098$^{*}$ & 0.066 & $-$0.033 & $-$0.001 & 0.065 \\ 
  & (0.057) & (0.059) & (0.075) & (0.078) & (0.069) \\ 
  & & & & & \\ 
 Income Q2 & $-$0.025 & $-$0.016 & 0.010 & 0.066 & $-$0.035 \\ 
  & (0.054) & (0.056) & (0.072) & (0.074) & (0.065) \\ 
  & & & & & \\ 
 Income Q3 & 0.052 & $-$0.050 & $-$0.083 & 0.098 & $-$0.017 \\ 
  & (0.057) & (0.059) & (0.075) & (0.077) & (0.068) \\ 
  & & & & & \\ 
 Income Q4 & $-$0.013 & $-$0.060 & $-$0.076 & 0.049 & 0.100 \\ 
  & (0.055) & (0.057) & (0.073) & (0.075) & (0.067) \\ 
  & & & & & \\ 
 age: 30-49 & $-$0.038 & 0.103$^{*}$ & $-$0.111 & 0.068 & $-$0.022 \\ 
  & (0.052) & (0.054) & (0.069) & (0.071) & (0.063) \\ 
  & & & & & \\ 
 age: 50-87 & 0.043 & $-$0.012 & $-$0.075 & 0.090 & $-$0.046 \\ 
  & (0.057) & (0.059) & (0.075) & (0.077) & (0.068) \\ 
  & & & & & \\ 
 vote: Biden & $-$0.150$^{***}$ & 0.065 & $-$0.059 & 0.132$^{*}$ & 0.013 \\ 
  & (0.053) & (0.055) & (0.071) & (0.073) & (0.064) \\ 
  & & & & & \\ 
 vote: Trump & 0.049 & 0.187$^{***}$ & $-$0.175$^{**}$ & $-$0.140$^{*}$ & 0.079 \\ 
  & (0.056) & (0.058) & (0.075) & (0.077) & (0.068) \\ 
  & & & & & \\ 
 wave: Pilote 2 & $-$0.028 & $-$0.100$^{***}$ & 0.001 & 0.061 & 0.066$^{*}$ \\ 
  & (0.033) & (0.034) & (0.043) & (0.045) & (0.040) \\ 
  & & & & & \\ 
\hline \\[-1.8ex] 

Observations & 420 & 420 & 420 & 420 & 420 \\ 
\hline 
\hline \\[-1.8ex] 
\end{tabular} }
		\end{center}
	{\scriptsize Note: The dependent variables are indicator variables. The \textit{Technically feasible} variable equals one if the respondent thinks it is ``A lot'' or ``A great deal'' technically feasible to stop greenhouse gas emissions by the end of the century while maintaining satisfactory standards of living in the U.S.. The \textit{Affected personally} variable equals one if the respondents thinks that climate change already affects or will affect her personal life negatively ``A lot'' or ``A great deal''. The \textit{Halt by end of century} variable equals one if the respondent thinks it is ``Somewhat likely'' or ``Very likely'' that human kind halts climate change by the end of the century. The \textit{Positive effects on the economy} variable equals one if the respondent thinks that if we decide to halt climate change through ambitious policies, there would be ``Positive'' or ``Very positive'' effects on the U.S economy and employment. The \textit{Negative effects personally} variable equals one if the respondent thinks that if we decide to halt climate change through ambitious policies, it would negatively affect ``A lot'' or ``A great deal'' her lifestyle. See notes under Table \ref{table polviews} and Table \ref{table exists} for a description of the covariates.
	\newline *p$<$0.1; **p$<$0.05; ***p$<$0.01}
	\end{table}		
\end{landscape}

\begin{table}[h!]
	\caption{Willing to change behavior}
	\begin{center}
		\scalebox{0.7}{
\begin{tabular}{@{\extracolsep{5pt}}lccccc} 
\\[-1.8ex]\hline 
\hline \\[-1.8ex] 
 & \multicolumn{5}{c}{Willing to change lifestyle?} \\ 
\cline{2-6} 
\\[-1.8ex] & Limit flying & Limit driving & Have an eletric vehicle & Limit beef consumption & Limit heating \\ 
\hline \\[-1.8ex] 
 Mean & 0.404 & 0.312 & 0.491 & 0.373 & 0.283  \\ \hline \\[-1.8ex] race: White only & 0.088$^{***}$ & $-$0.030 & 0.025 & 0.002 & $-$0.016 \\ 
  & (0.026) & (0.024) & (0.025) & (0.024) & (0.023) \\ 
  & & & & & \\ 
 Male & $-$0.005 & 0.018 & 0.020 & $-$0.056$^{**}$ & 0.016 \\ 
  & (0.023) & (0.021) & (0.022) & (0.022) & (0.021) \\ 
  & & & & & \\ 
 Children & 0.047$^{*}$ & 0.050$^{**}$ & 0.009 & $-$0.014 & 0.017 \\ 
  & (0.024) & (0.022) & (0.024) & (0.023) & (0.022) \\ 
  & & & & & \\ 
 No college & 0.007 & $-$0.040$^{*}$ & $-$0.061$^{**}$ & $-$0.074$^{***}$ & $-$0.012 \\ 
  & (0.026) & (0.024) & (0.025) & (0.025) & (0.023) \\ 
  & & & & & \\ 
 status: Retired & 0.014 & $-$0.055$^{*}$ & 0.017 & 0.076$^{**}$ & 0.028 \\ 
  & (0.033) & (0.031) & (0.033) & (0.032) & (0.030) \\ 
  & & & & & \\ 
 status: Student & $-$0.066$^{*}$ & 0.021 & 0.071$^{**}$ & 0.115$^{***}$ & 0.080$^{**}$ \\ 
  & (0.035) & (0.033) & (0.035) & (0.034) & (0.032) \\ 
  & & & & & \\ 
 status: Working & $-$0.066$^{*}$ & $-$0.020 & 0.134$^{***}$ & 0.135$^{***}$ & 0.089$^{***}$ \\ 
  & (0.036) & (0.034) & (0.036) & (0.035) & (0.033) \\ 
  & & & & & \\ 
 Income Q2 & $-$0.020 & 0.096$^{**}$ & 0.022 & 0.062 & 0.048 \\ 
  & (0.043) & (0.040) & (0.043) & (0.041) & (0.039) \\ 
  & & & & & \\ 
 Income Q3 & $-$0.138$^{***}$ & 0.059 & $-$0.064 & $-$0.034 & 0.028 \\ 
  & (0.044) & (0.041) & (0.043) & (0.042) & (0.040) \\ 
  & & & & & \\ 
 Income Q4 & $-$0.226$^{***}$ & $-$0.047 & $-$0.076$^{*}$ & $-$0.101$^{**}$ & $-$0.090$^{**}$ \\ 
  & (0.046) & (0.043) & (0.046) & (0.044) & (0.042) \\ 
  & & & & & \\ 
 age: 25-34 & $-$0.303$^{***}$ & $-$0.126$^{**}$ & $-$0.104$^{*}$ & $-$0.117$^{**}$ & $-$0.134$^{***}$ \\ 
  & (0.055) & (0.052) & (0.055) & (0.053) & (0.051) \\ 
  & & & & & \\ 
 age: 35-49 & 0.133$^{***}$ & 0.190$^{***}$ & 0.257$^{***}$ & 0.122$^{***}$ & 0.155$^{***}$ \\ 
  & (0.037) & (0.035) & (0.037) & (0.036) & (0.034) \\ 
  & & & & & \\ 
 age: 50-64 & 0.035 & 0.003 & $-$0.028 & $-$0.137$^{***}$ & $-$0.009 \\ 
  & (0.040) & (0.037) & (0.039) & (0.038) & (0.036) \\ 
  & & & & & \\ 
 age: 64+ & $-$0.010 & $-$0.043 & $-$0.001 & $-$0.024 & $-$0.024 \\ 
  & (0.030) & (0.028) & (0.029) & (0.028) & (0.027) \\ 
  & & & & & \\ 
 vote: Biden & 0.042 & 0.039 & $-$0.003 & 0.017 & $-$0.012 \\ 
  & (0.029) & (0.027) & (0.029) & (0.028) & (0.027) \\ 
  & & & & & \\ 
 vote: Trump & 0.036 & 0.023 & $-$0.058$^{*}$ & 0.032 & $-$0.002 \\ 
  & (0.031) & (0.028) & (0.030) & (0.029) & (0.028) \\ 
  & & & & & \\ 
\hline \\[-1.8ex] 

Observations & 2,010 & 2,010 & 2,010 & 2,010 & 2,010 \\ 
\hline 
\hline \\[-1.8ex] 
\end{tabular} }
	\end{center}
	{\footnotesize Note: The dependent variable are indicator variable equal to one, if the respondent is willing ``A lot'' or ``A great deal'' to adopt the behavior. For instance, the \textit{Limit flying} variable equals one if the respondent is willing ``A lot'' or a ``A great deal'' to limit flying. See notes under Table \ref{table polviews} and Table \ref{table exists} for a description of the covariates.
	\newline *p$<$0.1; **p$<$0.05; ***p$<$0.01}
\end{table}		

\begin{landscape}
	\begin{table}[h!]
		\caption{Conditions to change lifestyle}
		\begin{center}
			\scalebox{0.5}{
\begin{tabular}{@{\extracolsep{5pt}}lcccc} 
\\[-1.8ex]\hline 
\hline \\[-1.8ex] 
 & \multicolumn{4}{c}{Important factors} \\ 
\cline{2-5} 
\\[-1.8ex] & Ambitious policies & Financial support & People around changing & Rich changing \\ 
\hline \\[-1.8ex] 
 Mean & 0.394 & 0.517 & 0.466 & 0.518  \\ \hline \\[-1.8ex] race: White only & 0.006 & 0.032 & 0.035 & 0.021 \\ 
  & (0.025) & (0.026) & (0.026) & (0.026) \\ 
  & & & & \\ 
 Female & $-$0.014 & 0.011 & 0.028 & 0.056$^{**}$ \\ 
  & (0.022) & (0.023) & (0.023) & (0.023) \\ 
  & & & & \\ 
 Children & 0.017 & 0.0002 & 0.028 & $-$0.040 \\ 
  & (0.024) & (0.025) & (0.025) & (0.025) \\ 
  & & & & \\ 
 No college & $-$0.095$^{***}$ & $-$0.080$^{***}$ & $-$0.051$^{*}$ & $-$0.030 \\ 
  & (0.025) & (0.026) & (0.026) & (0.026) \\ 
  & & & & \\ 
 status: Retired & 0.041 & 0.013 & 0.022 & 0.026 \\ 
  & (0.045) & (0.047) & (0.047) & (0.047) \\ 
  & & & & \\ 
 status: Student & 0.002 & $-$0.066 & 0.026 & 0.151$^{**}$ \\ 
  & (0.063) & (0.066) & (0.066) & (0.066) \\ 
  & & & & \\ 
 status: Working & 0.024 & $-$0.006 & 0.021 & 0.042 \\ 
  & (0.035) & (0.036) & (0.036) & (0.036) \\ 
  & & & & \\ 
 Income Q2 & 0.052 & 0.007 & 0.001 & 0.050 \\ 
  & (0.032) & (0.034) & (0.034) & (0.034) \\ 
  & & & & \\ 
 Income Q3 & 0.095$^{***}$ & 0.021 & 0.073$^{**}$ & 0.068$^{*}$ \\ 
  & (0.034) & (0.036) & (0.036) & (0.036) \\ 
  & & & & \\ 
 Income Q4 & 0.133$^{***}$ & 0.046 & 0.133$^{***}$ & 0.086$^{**}$ \\ 
  & (0.035) & (0.037) & (0.037) & (0.037) \\ 
  & & & & \\ 
 age: 25-34 & 0.014 & $-$0.040 & 0.069 & 0.145$^{***}$ \\ 
  & (0.042) & (0.044) & (0.044) & (0.044) \\ 
  & & & & \\ 
 age: 35-49 & $-$0.044 & $-$0.128$^{***}$ & $-$0.001 & 0.063 \\ 
  & (0.043) & (0.045) & (0.044) & (0.044) \\ 
  & & & & \\ 
 age: 50-64 & $-$0.121$^{***}$ & $-$0.212$^{***}$ & $-$0.063 & 0.032 \\ 
  & (0.045) & (0.047) & (0.047) & (0.047) \\ 
  & & & & \\ 
 age: 65+ & $-$0.098$^{*}$ & $-$0.233$^{***}$ & $-$0.062 & 0.091 \\ 
  & (0.054) & (0.057) & (0.057) & (0.057) \\ 
  & & & & \\ 
 Left or Very left & 0.136$^{***}$ & 0.095$^{***}$ & 0.086$^{***}$ & 0.118$^{***}$ \\ 
  & (0.026) & (0.027) & (0.027) & (0.027) \\ 
  & & & & \\ 
 Right or Very right & $-$0.139$^{***}$ & $-$0.060$^{**}$ & $-$0.109$^{***}$ & $-$0.109$^{***}$ \\ 
  & (0.026) & (0.028) & (0.028) & (0.028) \\ 
  & & & & \\ 
 Center &  &  &  &  \\ 
  &  &  &  &  \\ 
  & & & & \\ 
 Climate treatment only & 0.011 & $-$0.056$^{*}$ & $-$0.004 & $-$0.020 \\ 
  & (0.029) & (0.030) & (0.030) & (0.030) \\ 
  & & & & \\ 
 Policy treatment only & 0.081$^{***}$ & $-$0.018 & $-$0.007 & $-$0.013 \\ 
  & (0.029) & (0.030) & (0.030) & (0.030) \\ 
  & & & & \\ 
 Both treatments & 0.077$^{***}$ & $-$0.048 & 0.015 & $-$0.017 \\ 
  & (0.030) & (0.031) & (0.031) & (0.031) \\ 
  & & & & \\ 
\hline \\[-1.8ex] 

Observations & 2,010 & 2,010 & 2,010 & 2,010 \\ 
\hline 
\hline \\[-1.8ex] 
\end{tabular} }
		\end{center}
	{\footnotesize Note: The dependent variables are indicator variables equal to one if the respondent thinks the factor is ``A lot'' or ``A great deal'' important in order for her to adopt a sustainable lifestyle. For instance, \textit{Ambitious policies} variable equals one if the respondent thinks that ambitious policies are a ``A lot'' or ``A great deal'' important for her to adopt a sutainable lifestyle. See notes under Table \ref{table polviews} and Table \ref{table exists} for a description of the covariates.
	\newline *p$<$0.1; **p$<$0.05; ***p$<$0.01}
	\end{table}		
\end{landscape}



\clearpage
\subsection{Preferences 1: Ban on combustion engine cars}



\begin{table}[h!]
	\caption{Opinion on ban on combustion engine cars} \label{table standard opinion}
	\begin{center}
		\scalebox{0.7}{
\begin{tabular}{@{\extracolsep{5pt}}lcccccc} 
\\[-1.8ex]\hline 
\hline \\[-1.8ex] 
 & \multicolumn{6}{c}{C02 emission limit for cars policy in the U.S.} \\ 
\cline{2-7} 
\\[-1.8ex] & Does exist & Trust federal gov. & Effective & Positive impact on jobs & Positive side effects & Support \\ 
\hline \\[-1.8ex] 
 Control group mean & 0.188 & 0.312 & 0.519 & 0.312 & 0.479 & 0  \\ \hline \\[-1.8ex] race: White only & $-$0.059 & $-$0.142 & $-$0.062 & 0.090 & 0.157$^{*}$ & 0.000 \\ 
  & (0.083) & (0.114) & (0.150) & (0.084) & (0.090) & (0.000) \\ 
  & & & & & & \\ 
 Male & 0.071 & $-$0.235$^{**}$ & $-$0.177 & $-$0.003 & 0.093 & 0.000 \\ 
  & (0.071) & (0.099) & (0.116) & (0.073) & (0.078) & (0.000) \\ 
  & & & & & & \\ 
 Children & 0.138$^{*}$ & $-$0.065 & $-$0.197 & 0.028 & 0.024 & 0.000 \\ 
  & (0.073) & (0.097) & (0.123) & (0.074) & (0.079) & (0.000) \\ 
  & & & & & & \\ 
 No college & $-$0.030 & 0.091 & $-$0.070 & $-$0.082 & $-$0.119 & 0.000 \\ 
  & (0.082) & (0.107) & (0.126) & (0.083) & (0.089) & (0.000) \\ 
  & & & & & & \\ 
 status: Retired & 0.063 & $-$0.039 & $-$0.103 & 0.070 & 0.217 & 0.000 \\ 
  & (0.129) & (0.170) & (0.199) & (0.132) & (0.140) & (0.000) \\ 
  & & & & & & \\ 
 status: Student & $-$0.387 & $-$0.190 & $-$0.447 & $-$0.048 & $-$0.371 & 0.000 \\ 
  & (0.308) & (0.401) & (0.388) & (0.314) & (0.335) & (0.000) \\ 
  & & & & & & \\ 
 staths: Working & 0.078 & 0.012 & $-$0.136 & 0.102 & 0.132 & 0.000 \\ 
  & (0.128) & (0.164) & (0.191) & (0.130) & (0.139) & (0.000) \\ 
  & & & & & & \\ 
 Income Q2 & $-$0.040 & $-$0.164 & 0.144 & 0.029 & $-$0.063 & 0.000 \\ 
  & (0.109) & (0.147) & (0.179) & (0.111) & (0.118) & (0.000) \\ 
  & & & & & & \\ 
 Income Q3 & $-$0.012 & 0.104 & 0.158 & 0.082 & 0.028 & 0.000 \\ 
  & (0.103) & (0.133) & (0.166) & (0.106) & (0.112) & (0.000) \\ 
  & & & & & & \\ 
 Income Q4 & 0.149 & 0.013 & 0.047 & 0.063 & 0.026 & 0.000 \\ 
  & (0.110) & (0.147) & (0.192) & (0.112) & (0.119) & (0.000) \\ 
  & & & & & & \\ 
 age: 30-49 & $-$0.123 & 0.411 & 0.586$^{**}$ & $-$0.093 & $-$0.218 & 0.000 \\ 
  & (0.178) & (0.287) & (0.263) & (0.181) & (0.193) & (0.000) \\ 
  & & & & & & \\ 
 age: 50-87 & $-$0.348$^{*}$ & 0.613$^{**}$ & 0.765$^{***}$ & $-$0.357$^{*}$ & $-$0.353$^{*}$ & 0.000 \\ 
  & (0.180) & (0.284) & (0.274) & (0.184) & (0.196) & (0.000) \\ 
  & & & & & & \\ 
 vote: Biden & 0.010 & 0.094 & $-$0.086 & 0.177$^{*}$ & 0.139 & 0.000 \\ 
  & (0.100) & (0.137) & (0.168) & (0.102) & (0.109) & (0.000) \\ 
  & & & & & & \\ 
 vote: Trump & 0.062 & $-$0.168 & $-$0.405$^{**}$ & $-$0.085 & $-$0.112 & 0.000 \\ 
  & (0.109) & (0.142) & (0.174) & (0.111) & (0.118) & (0.000) \\ 
  & & & & & & \\ 
 Both treatments & 0.231$^{**}$ & 0.138 & $-$0.153 & 0.136 & 0.093 & 0.000 \\ 
  & (0.102) & (0.149) & (0.186) & (0.104) & (0.111) & (0.000) \\ 
  & & & & & & \\ 
 Climate treatment only & 0.119 & $-$0.036 & $-$0.147 & 0.042 & 0.003 & 0.000 \\ 
  & (0.096) & (0.125) & (0.150) & (0.098) & (0.104) & (0.000) \\ 
  & & & & & & \\ 
 Policy treatment only & 0.151$^{*}$ & 0.166 & $-$0.005 & 0.040 & 0.117 & 0.000 \\ 
  & (0.088) & (0.120) & (0.156) & (0.090) & (0.096) & (0.000) \\ 
  & & & & & & \\ 
 Constant & 0.250 & 0.041 & 0.419 & 0.332 & 0.390$^{*}$ & 0.000 \\ 
  & (0.214) & (0.303) & (0.309) & (0.218) & (0.232) & (0.000) \\ 
  & & & & & & \\ 
\hline \\[-1.8ex] 

Observations & 191 & 112 & 84 & 191 & 191 & 80 \\ 
\hline 
\hline \\[-1.8ex] 
\end{tabular} }
	\end{center}
	{\footnotesize Note: The dependent variables are indicator variables equal to one if the respondent ``Somewhat agrees'' or ``Strongly agrees'' with the proposition. For instance, the \textit{Reduce car emissions} variable equals one if the respondent ``Somewhat agrees'' or ``Strongly agrees'' with the fact that a ban on combustion engine cars would reduce $CO_2$ emissions from cars. The \textit{Reduce pollution} variable corresponds to the proposition that a ban on combustion engine cars would reduce air pollution. The \textit{Negative effect} variable corresponds to the proposition that a ban on combustion engine cars would have negative effect on the U.S. economy and employment. The \textit{Large effect} variable corresponds to the proposition that a ban on combustion engine cars would have a large effect on the U.S. economy and employment. The \textit{Costly} variable corresponds to the proposition that a ban on combustion engine cars would be a costly way to fight climate change. See notes under Table \ref{table polviews} and Table \ref{table exists} for a description of the covariates.
	\newline *p$<$0.1; **p$<$0.05; ***p$<$0.01}
\end{table}	

\begin{table}[h!]
	\caption{Perceived winners of a ban on combustion engine cars}
	\begin{center}
		\scalebox{0.7}{
\begin{tabular}{@{\extracolsep{5pt}}lcccccc} 
\\[-1.8ex]\hline 
\hline \\[-1.8ex] 
 & \multicolumn{6}{c}{Winners of emission limits for cars policy} \\ 
\cline{2-7} 
\\[-1.8ex] & Poorest & Middle class & Richest & Urban & Rural & Own household \\ 
\hline \\[-1.8ex] 
 Control group mean & 0.297 & 0.254 & 0.347 & 0.314 & 0.229 & 0.229  \\ \hline \\[-1.8ex] race: White only & 0.055 & 0.034 & 0.025 & 0.065 & 0.035 & 0.047 \\ 
  & (0.049) & (0.047) & (0.047) & (0.048) & (0.046) & (0.046) \\ 
  & & & & & & \\ 
 Male & 0.071 & 0.026 & 0.080$^{*}$ & 0.045 & $-$0.022 & 0.082$^{**}$ \\ 
  & (0.043) & (0.041) & (0.041) & (0.042) & (0.041) & (0.041) \\ 
  & & & & & & \\ 
 Children & 0.060 & 0.118$^{***}$ & 0.150$^{***}$ & 0.107$^{**}$ & 0.094$^{**}$ & 0.122$^{***}$ \\ 
  & (0.046) & (0.043) & (0.044) & (0.044) & (0.043) & (0.043) \\ 
  & & & & & & \\ 
 No college & $-$0.092$^{*}$ & $-$0.124$^{***}$ & 0.069 & $-$0.069 & $-$0.058 & $-$0.094$^{**}$ \\ 
  & (0.050) & (0.047) & (0.048) & (0.049) & (0.047) & (0.047) \\ 
  & & & & & & \\ 
 status: Retired & $-$0.037 & 0.046 & 0.020 & 0.057 & 0.069 & $-$0.043 \\ 
  & (0.080) & (0.075) & (0.076) & (0.077) & (0.075) & (0.075) \\ 
  & & & & & & \\ 
 status: Student & $-$0.153 & 0.005 & 0.172 & 0.012 & $-$0.041 & $-$0.238$^{*}$ \\ 
  & (0.143) & (0.135) & (0.136) & (0.138) & (0.134) & (0.134) \\ 
  & & & & & & \\ 
 staths: Working & $-$0.052 & $-$0.058 & $-$0.021 & $-$0.018 & 0.060 & $-$0.049 \\ 
  & (0.070) & (0.066) & (0.066) & (0.067) & (0.065) & (0.065) \\ 
  & & & & & & \\ 
 Income Q2 & $-$0.022 & $-$0.076 & 0.022 & 0.041 & $-$0.041 & $-$0.030 \\ 
  & (0.063) & (0.059) & (0.060) & (0.060) & (0.058) & (0.059) \\ 
  & & & & & & \\ 
 Income Q3 & $-$0.030 & $-$0.052 & 0.079 & 0.130$^{**}$ & 0.045 & $-$0.019 \\ 
  & (0.064) & (0.061) & (0.061) & (0.062) & (0.060) & (0.060) \\ 
  & & & & & & \\ 
 Income Q4 & 0.014 & $-$0.051 & 0.107 & 0.105 & 0.039 & $-$0.005 \\ 
  & (0.070) & (0.066) & (0.067) & (0.067) & (0.065) & (0.066) \\ 
  & & & & & & \\ 
 age: 30-49 & 0.001 & 0.034 & 0.007 & $-$0.097 & 0.024 & $-$0.025 \\ 
  & (0.070) & (0.066) & (0.067) & (0.068) & (0.066) & (0.066) \\ 
  & & & & & & \\ 
 age: 50-87 & $-$0.093 & $-$0.133$^{*}$ & $-$0.209$^{***}$ & $-$0.166$^{**}$ & $-$0.086 & $-$0.153$^{**}$ \\ 
  & (0.076) & (0.072) & (0.073) & (0.073) & (0.071) & (0.072) \\ 
  & & & & & & \\ 
 vote: Biden & 0.156$^{**}$ & 0.258$^{***}$ & 0.241$^{***}$ & 0.283$^{***}$ & 0.183$^{***}$ & 0.221$^{***}$ \\ 
  & (0.062) & (0.059) & (0.059) & (0.060) & (0.058) & (0.058) \\ 
  & & & & & & \\ 
 vote: Trump & $-$0.024 & 0.017 & 0.157$^{**}$ & 0.037 & $-$0.004 & 0.047 \\ 
  & (0.066) & (0.062) & (0.063) & (0.064) & (0.062) & (0.062) \\ 
  & & & & & & \\ 
 Both treatments & 0.087 & 0.101$^{*}$ & 0.021 & 0.069 & 0.108$^{*}$ & 0.137$^{**}$ \\ 
  & (0.060) & (0.056) & (0.057) & (0.057) & (0.056) & (0.056) \\ 
  & & & & & & \\ 
 Climate treatment only & 0.001 & 0.091 & $-$0.014 & 0.035 & 0.058 & 0.096$^{*}$ \\ 
  & (0.061) & (0.057) & (0.058) & (0.059) & (0.057) & (0.057) \\ 
  & & & & & & \\ 
 Policy treatment only & 0.204$^{***}$ & 0.083 & $-$0.047 & 0.069 & 0.085 & 0.115$^{**}$ \\ 
  & (0.058) & (0.055) & (0.056) & (0.056) & (0.054) & (0.055) \\ 
  & & & & & & \\ 
 Constant & 0.216$^{**}$ & 0.138 & 0.008 & 0.044 & 0.030 & 0.093 \\ 
  & (0.107) & (0.101) & (0.102) & (0.104) & (0.100) & (0.101) \\ 
  & & & & & & \\ 
\hline \\[-1.8ex] 

Observations & 499 & 499 & 499 & 499 & 499 & 499 \\ 
\hline 
\hline \\[-1.8ex] 
\end{tabular} }
	\end{center}
	{\footnotesize Note: The dependent variables are indicator variables equal to one if the respondent thinks the category would ``Mostly win' or ``Win a lot''' from a ban on combustion engine cars policy. For instance, the variable \textit{Poorest} equals one if the respondent thinks the poorest would ``Mostly win'' or ``Win a lot'' if such a policy was implemented. See notes under Table \ref{table polviews} and Table \ref{table exists} for a description of the covariates.
	\newline *p$<$0.1; **p$<$0.05; ***p$<$0.01}
\end{table}	

\begin{table}[h!]
	\caption{Perceived losers of a ban on combustion engine cars}
	\begin{center}
		\scalebox{0.7}{
\begin{tabular}{@{\extracolsep{5pt}}lcccccc} 
\\[-1.8ex]\hline 
\hline \\[-1.8ex] 
 & \multicolumn{6}{c}{Losers of emission limits for cars policy} \\ 
\cline{2-7} 
\\[-1.8ex] & Poorest & Middle class & Richest & Urban & Rural & Own household \\ 
\hline \\[-1.8ex] 
 Control group mean & 0.347 & 0.407 & 0.203 & 0.263 & 0.347 & 0.288  \\ \hline \\[-1.8ex] race: White only & $-$0.005 & 0.003 & 0.002 & 0.011 & $-$0.090$^{*}$ & $-$0.001 \\ 
  & (0.049) & (0.049) & (0.046) & (0.046) & (0.049) & (0.047) \\ 
  & & & & & & \\ 
 Male & $-$0.004 & $-$0.006 & 0.072$^{*}$ & 0.065 & 0.051 & 0.003 \\ 
  & (0.043) & (0.042) & (0.041) & (0.040) & (0.043) & (0.041) \\ 
  & & & & & & \\ 
 Children & 0.047 & $-$0.030 & 0.014 & $-$0.008 & $-$0.016 & 0.062 \\ 
  & (0.046) & (0.045) & (0.043) & (0.043) & (0.046) & (0.044) \\ 
  & & & & & & \\ 
 No college & 0.042 & 0.148$^{***}$ & $-$0.025 & 0.014 & 0.032 & 0.011 \\ 
  & (0.050) & (0.050) & (0.048) & (0.047) & (0.051) & (0.049) \\ 
  & & & & & & \\ 
 status: Retired & $-$0.120 & $-$0.149$^{*}$ & $-$0.028 & $-$0.002 & $-$0.114 & $-$0.137$^{*}$ \\ 
  & (0.082) & (0.082) & (0.078) & (0.077) & (0.082) & (0.079) \\ 
  & & & & & & \\ 
 status: Student & 0.097 & $-$0.153 & 0.069 & 0.092 & 0.098 & 0.328$^{***}$ \\ 
  & (0.131) & (0.130) & (0.125) & (0.123) & (0.131) & (0.126) \\ 
  & & & & & & \\ 
 staths: Working & $-$0.071 & $-$0.038 & $-$0.012 & 0.065 & $-$0.037 & $-$0.050 \\ 
  & (0.070) & (0.070) & (0.067) & (0.066) & (0.070) & (0.067) \\ 
  & & & & & & \\ 
 Income Q2 & 0.092 & 0.112$^{*}$ & $-$0.006 & 0.046 & 0.135$^{**}$ & 0.133$^{**}$ \\ 
  & (0.065) & (0.065) & (0.062) & (0.061) & (0.065) & (0.063) \\ 
  & & & & & & \\ 
 Income Q3 & 0.103 & 0.134$^{**}$ & 0.020 & 0.053 & 0.111 & 0.053 \\ 
  & (0.068) & (0.068) & (0.065) & (0.064) & (0.068) & (0.066) \\ 
  & & & & & & \\ 
 Income Q4 & 0.145$^{**}$ & 0.254$^{***}$ & $-$0.017 & 0.106$^{*}$ & 0.246$^{***}$ & 0.088 \\ 
  & (0.068) & (0.068) & (0.065) & (0.064) & (0.068) & (0.065) \\ 
  & & & & & & \\ 
 age: 30-49 & 0.016 & 0.035 & 0.018 & 0.011 & 0.056 & $-$0.027 \\ 
  & (0.066) & (0.066) & (0.063) & (0.062) & (0.067) & (0.064) \\ 
  & & & & & & \\ 
 age: 50-87 & 0.036 & 0.131$^{*}$ & 0.052 & 0.012 & 0.119 & 0.107 \\ 
  & (0.073) & (0.073) & (0.070) & (0.069) & (0.074) & (0.071) \\ 
  & & & & & & \\ 
 vote: Biden & 0.046 & $-$0.050 & $-$0.101$^{*}$ & $-$0.162$^{***}$ & 0.009 & $-$0.031 \\ 
  & (0.062) & (0.062) & (0.059) & (0.058) & (0.062) & (0.060) \\ 
  & & & & & & \\ 
 vote: Trump & 0.300$^{***}$ & 0.183$^{***}$ & $-$0.018 & 0.021 & 0.267$^{***}$ & 0.250$^{***}$ \\ 
  & (0.066) & (0.065) & (0.063) & (0.062) & (0.066) & (0.063) \\ 
  & & & & & & \\ 
 Climate treatment only & $-$0.042 & $-$0.156$^{***}$ & 0.025 & $-$0.071 & $-$0.052 & $-$0.088 \\ 
  & (0.060) & (0.060) & (0.058) & (0.057) & (0.061) & (0.058) \\ 
  & & & & & & \\ 
 Policy treatment only & $-$0.123$^{**}$ & $-$0.125$^{**}$ & 0.149$^{***}$ & $-$0.010 & $-$0.015 & $-$0.004 \\ 
  & (0.058) & (0.058) & (0.055) & (0.054) & (0.058) & (0.056) \\ 
  & & & & & & \\ 
 Both treatments & $-$0.061 & $-$0.113$^{*}$ & 0.007 & $-$0.062 & 0.011 & $-$0.099$^{*}$ \\ 
  & (0.060) & (0.060) & (0.057) & (0.056) & (0.060) & (0.058) \\ 
  & & & & & & \\ 
 wave: Pilote 2 & 0.0002 & $-$0.024 & 0.025 & $-$0.026 & 0.053 & 0.045 \\ 
  & (0.042) & (0.042) & (0.040) & (0.040) & (0.042) & (0.041) \\ 
  & & & & & & \\ 
\hline \\[-1.8ex] 

Observations & 499 & 499 & 499 & 499 & 499 & 499 \\ 
\hline 
\hline \\[-1.8ex] 
\end{tabular} }
	\end{center}
	{\footnotesize Note: The dependent variables are indicator variables equal to one if the respondent thinks the category would ``Mostly lose' or ``Lose a lot''' from a ban on combustion engine cars policy. For instance, the variable \textit{Poorest} equals one if the respondent thinks the poorest would ``Mostly lose'' or ``Lose a lot'' if such a policy was implemented. See notes under Table \ref{table polviews} and Table \ref{table exists} for a description of the covariates.
	\newline *p$<$0.1; **p$<$0.05; ***p$<$0.01}
\end{table}	


\begin{table}[h!]
	\caption{Perception of a ban on combustion engine cars}
	\begin{center}
		\scalebox{0.7}{
\begin{tabular}{@{\extracolsep{5pt}}lccc} 
\\[-1.8ex]\hline 
\hline \\[-1.8ex] 
\\[-1.8ex] & Fair & Support & Support with alternatives \\ 
\hline \\[-1.8ex] 
 Control group mean & 0.376 & 0.409 & 0.458  \\ \hline \\[-1.8ex] race: White only & $-$0.024 & $-$0.026 & 0.021 \\ 
  & (0.025) & (0.025) & (0.025) \\ 
  & & & \\ 
 Female & $-$0.005 & 0.011 & 0.058$^{***}$ \\ 
  & (0.022) & (0.022) & (0.022) \\ 
  & & & \\ 
 Children & 0.043$^{*}$ & 0.002 & 0.013 \\ 
  & (0.023) & (0.023) & (0.024) \\ 
  & & & \\ 
 No college & $-$0.107$^{***}$ & $-$0.099$^{***}$ & $-$0.156$^{***}$ \\ 
  & (0.025) & (0.025) & (0.025) \\ 
  & & & \\ 
 status: Retired & 0.030 & 0.061 & 0.097$^{**}$ \\ 
  & (0.045) & (0.045) & (0.045) \\ 
  & & & \\ 
 status: Student & $-$0.124$^{**}$ & $-$0.119$^{*}$ & 0.056 \\ 
  & (0.062) & (0.063) & (0.063) \\ 
  & & & \\ 
 status: Working & 0.049 & 0.077$^{**}$ & 0.056 \\ 
  & (0.034) & (0.035) & (0.035) \\ 
  & & & \\ 
 Income Q2 & 0.021 & 0.003 & 0.039 \\ 
  & (0.032) & (0.032) & (0.032) \\ 
  & & & \\ 
 Income Q3 & 0.064$^{*}$ & 0.041 & 0.044 \\ 
  & (0.034) & (0.034) & (0.034) \\ 
  & & & \\ 
 Income Q4 & 0.093$^{***}$ & 0.078$^{**}$ & 0.081$^{**}$ \\ 
  & (0.035) & (0.035) & (0.035) \\ 
  & & & \\ 
 age: 25-34 & $-$0.026 & 0.037 & $-$0.021 \\ 
  & (0.042) & (0.042) & (0.042) \\ 
  & & & \\ 
 age: 35-49 & 0.049 & 0.052 & 0.003 \\ 
  & (0.042) & (0.042) & (0.043) \\ 
  & & & \\ 
 age: 50-64 & $-$0.091$^{**}$ & $-$0.045 & $-$0.160$^{***}$ \\ 
  & (0.045) & (0.045) & (0.045) \\ 
  & & & \\ 
 age: 65+ & $-$0.088 & $-$0.046 & $-$0.182$^{***}$ \\ 
  & (0.054) & (0.054) & (0.054) \\ 
  & & & \\ 
 Left or Very left & 0.182$^{***}$ & 0.191$^{***}$ & 0.131$^{***}$ \\ 
  & (0.026) & (0.026) & (0.026) \\ 
  & & & \\ 
 Right or Very right & $-$0.114$^{***}$ & $-$0.137$^{***}$ & $-$0.161$^{***}$ \\ 
  & (0.026) & (0.026) & (0.026) \\ 
  & & & \\ 
 Center &  &  &  \\ 
  &  &  &  \\ 
  & & & \\ 
 Climate treatment only & 0.011 & 0.008 & $-$0.043 \\ 
  & (0.029) & (0.029) & (0.029) \\ 
  & & & \\ 
 Policy treatment only & 0.066$^{**}$ & 0.053$^{*}$ & $-$0.016 \\ 
  & (0.028) & (0.028) & (0.028) \\ 
  & & & \\ 
 Both treatments & 0.045 & 0.014 & $-$0.007 \\ 
  & (0.029) & (0.030) & (0.030) \\ 
  & & & \\ 
\hline \\[-1.8ex] 

Observations & 2,010 & 2,010 & 2,010 \\ 
\hline 
\hline \\[-1.8ex] 
\end{tabular} }
	\end{center}
	{\footnotesize Note: The dependent variables are indicator variables equal to one if the respondent perceives the category as losers of an emission limits for cars policy. For instance, the variable \textit{Poorest} equals one if the respondent thinks the poorest would lose if such a policy was implemented. See notes under Table \ref{table polviews} and Table \ref{table exists} for a description of the covariates.
	\newline *p$<$0.1; **p$<$0.05; ***p$<$0.01}
\end{table}	



\clearpage
\subsection{Preferences 2: Green investments}



\begin{table}[h!]
	\caption{Opinion on green investments}
	\begin{center}
		\scalebox{0.7}{
\begin{tabular}{@{\extracolsep{5pt}}lcccccc} 
\\[-1.8ex]\hline 
\hline \\[-1.8ex] 
 & \multicolumn{6}{c}{Effects of green infrastructure program} \\ 
\cline{2-7} 
\\[-1.8ex] & Greener electricity & More use of public transport & Reduce pollution & Negative effect & Large effect & Costly \\ 
\hline \\[-1.8ex] 
 Control group mean & 0.662 & 0.54 & 0.716 & 0.404 & 0.57 & 0.539  \\ \hline \\[-1.8ex] race: White only & 0.033 & 0.003 & 0.040$^{*}$ & 0.021 & 0.031 & $-$0.056$^{**}$ \\ 
  & (0.024) & (0.026) & (0.024) & (0.025) & (0.026) & (0.025) \\ 
  & & & & & & \\ 
 Female & 0.054$^{**}$ & 0.009 & 0.080$^{***}$ & $-$0.077$^{***}$ & $-$0.047$^{**}$ & $-$0.028 \\ 
  & (0.022) & (0.023) & (0.021) & (0.022) & (0.023) & (0.023) \\ 
  & & & & & & \\ 
 Children & $-$0.0002 & 0.008 & 0.010 & 0.051$^{**}$ & 0.034 & 0.001 \\ 
  & (0.023) & (0.024) & (0.022) & (0.023) & (0.024) & (0.024) \\ 
  & & & & & & \\ 
 No college & $-$0.106$^{***}$ & $-$0.138$^{***}$ & $-$0.096$^{***}$ & $-$0.057$^{**}$ & $-$0.080$^{***}$ & $-$0.027 \\ 
  & (0.025) & (0.026) & (0.024) & (0.025) & (0.026) & (0.026) \\ 
  & & & & & & \\ 
 status: Retired & 0.007 & $-$0.018 & 0.021 & 0.067 & 0.006 & 0.119$^{**}$ \\ 
  & (0.044) & (0.046) & (0.043) & (0.045) & (0.046) & (0.046) \\ 
  & & & & & & \\ 
 status: Student & 0.068 & $-$0.043 & 0.029 & $-$0.082 & $-$0.116$^{*}$ & $-$0.045 \\ 
  & (0.062) & (0.065) & (0.059) & (0.062) & (0.065) & (0.064) \\ 
  & & & & & & \\ 
 status: Working & 0.030 & 0.032 & 0.063$^{*}$ & $-$0.011 & $-$0.050 & $-$0.039 \\ 
  & (0.034) & (0.036) & (0.033) & (0.034) & (0.036) & (0.036) \\ 
  & & & & & & \\ 
 Income Q2 & 0.032 & 0.019 & 0.040 & $-$0.029 & $-$0.026 & 0.049 \\ 
  & (0.032) & (0.033) & (0.031) & (0.032) & (0.034) & (0.033) \\ 
  & & & & & & \\ 
 Income Q3 & 0.068$^{**}$ & 0.038 & 0.032 & 0.018 & 0.043 & 0.086$^{**}$ \\ 
  & (0.034) & (0.036) & (0.033) & (0.034) & (0.036) & (0.035) \\ 
  & & & & & & \\ 
 Income Q4 & 0.098$^{***}$ & 0.070$^{*}$ & 0.102$^{***}$ & 0.044 & 0.070$^{*}$ & 0.131$^{***}$ \\ 
  & (0.035) & (0.036) & (0.033) & (0.035) & (0.036) & (0.036) \\ 
  & & & & & & \\ 
 age: 25-34 & 0.039 & 0.027 & 0.026 & $-$0.113$^{***}$ & 0.054 & 0.021 \\ 
  & (0.041) & (0.043) & (0.040) & (0.042) & (0.043) & (0.043) \\ 
  & & & & & & \\ 
 age: 35-49 & 0.036 & $-$0.013 & 0.004 & $-$0.032 & 0.050 & 0.061 \\ 
  & (0.042) & (0.044) & (0.040) & (0.042) & (0.044) & (0.044) \\ 
  & & & & & & \\ 
 age: 50-64 & 0.021 & $-$0.041 & 0.027 & $-$0.046 & 0.016 & $-$0.013 \\ 
  & (0.044) & (0.047) & (0.043) & (0.045) & (0.047) & (0.046) \\ 
  & & & & & & \\ 
 age: 65+ & 0.010 & $-$0.043 & 0.099$^{*}$ & $-$0.133$^{**}$ & 0.031 & $-$0.099$^{*}$ \\ 
  & (0.053) & (0.056) & (0.051) & (0.054) & (0.056) & (0.055) \\ 
  & & & & & & \\ 
 Left or Very left & 0.068$^{***}$ & 0.076$^{***}$ & 0.013 & $-$0.105$^{***}$ & $-$0.013 & $-$0.128$^{***}$ \\ 
  & (0.025) & (0.027) & (0.024) & (0.026) & (0.027) & (0.026) \\ 
  & & & & & & \\ 
 Right or Very right & $-$0.162$^{***}$ & $-$0.140$^{***}$ & $-$0.185$^{***}$ & 0.203$^{***}$ & 0.100$^{***}$ & 0.140$^{***}$ \\ 
  & (0.026) & (0.027) & (0.025) & (0.026) & (0.027) & (0.027) \\ 
  & & & & & & \\ 
 Center &  &  &  &  &  &  \\ 
  &  &  &  &  &  &  \\ 
  & & & & & & \\ 
 Climate treatment only & 0.006 & $-$0.010 & 0.010 & $-$0.068$^{**}$ & $-$0.067$^{**}$ & $-$0.014 \\ 
  & (0.028) & (0.030) & (0.027) & (0.029) & (0.030) & (0.030) \\ 
  & & & & & & \\ 
 Policy treatment only & 0.020 & 0.021 & $-$0.009 & $-$0.011 & 0.014 & 0.067$^{**}$ \\ 
  & (0.028) & (0.029) & (0.027) & (0.028) & (0.029) & (0.029) \\ 
  & & & & & & \\ 
 Both treatments & 0.009 & 0.045 & $-$0.010 & 0.019 & 0.045 & 0.014 \\ 
  & (0.029) & (0.031) & (0.028) & (0.029) & (0.031) & (0.030) \\ 
  & & & & & & \\ 
\hline \\[-1.8ex] 

Observations & 2,010 & 2,010 & 2,010 & 2,010 & 2,010 & 2,010 \\ 
\hline 
\hline \\[-1.8ex] 
\end{tabular} }
	\end{center}
	{\footnotesize Note: The dependent variables are indicator variables equal to one if the respondent agrees with the proposition. For instance, \textit{Trust federal government} equals one if the respondent thinks she can trust the U.S. government to correctly implement a green infrastructure program. See notes under Table \ref{table polviews} and Table \ref{table exists} for a description of the covariates.
	\newline *p$<$0.1; **p$<$0.05; ***p$<$0.01}
\end{table}	

\begin{table}[h!]
	\caption{Perceived winners of a green investments policy}
	\begin{center}
		\scalebox{0.7}{
\begin{tabular}{@{\extracolsep{5pt}}lccccc} 
\\[-1.8ex]\hline 
\hline \\[-1.8ex] 
 & \multicolumn{5}{c}{Winners of green infrastructure program} \\ 
\cline{2-6} 
\\[-1.8ex] & Poorest & Middle class & Richest & Rural & Own household \\ 
\hline \\[-1.8ex] 
 Control group mean & 0.273 & 0.263 & 0.381 & 0.249 & 0.281  \\ \hline \\[-1.8ex] race: White only & $-$0.043$^{*}$ & 0.035 & 0.042$^{*}$ & $-$0.003 & 0.036 \\ 
  & (0.023) & (0.023) & (0.025) & (0.023) & (0.022) \\ 
  & & & & & \\ 
 Male & 0.074$^{***}$ & 0.035$^{*}$ & 0.008 & 0.073$^{***}$ & 0.035$^{*}$ \\ 
  & (0.021) & (0.020) & (0.022) & (0.020) & (0.020) \\ 
  & & & & & \\ 
 Children & 0.049$^{**}$ & 0.024 & 0.037 & 0.084$^{***}$ & 0.056$^{***}$ \\ 
  & (0.022) & (0.022) & (0.023) & (0.021) & (0.021) \\ 
  & & & & & \\ 
 No college & $-$0.006 & $-$0.031 & $-$0.057$^{**}$ & $-$0.056$^{**}$ & $-$0.046$^{**}$ \\ 
  & (0.023) & (0.023) & (0.025) & (0.023) & (0.023) \\ 
  & & & & & \\ 
 status: Retired & 0.057 & 0.018 & 0.038 & $-$0.003 & 0.014 \\ 
  & (0.042) & (0.041) & (0.045) & (0.041) & (0.041) \\ 
  & & & & & \\ 
 status: Student & 0.053 & 0.015 & $-$0.148$^{**}$ & 0.002 & $-$0.038 \\ 
  & (0.059) & (0.058) & (0.063) & (0.057) & (0.057) \\ 
  & & & & & \\ 
 status: Working & 0.128$^{***}$ & 0.082$^{***}$ & 0.008 & 0.037 & 0.063$^{**}$ \\ 
  & (0.032) & (0.032) & (0.035) & (0.031) & (0.031) \\ 
  & & & & & \\ 
 Income Q2 & 0.058$^{*}$ & $-$0.003 & 0.077$^{**}$ & 0.039 & 0.054$^{*}$ \\ 
  & (0.030) & (0.030) & (0.032) & (0.029) & (0.029) \\ 
  & & & & & \\ 
 Income Q3 & 0.090$^{***}$ & 0.026 & 0.023 & 0.063$^{**}$ & 0.113$^{***}$ \\ 
  & (0.032) & (0.032) & (0.034) & (0.031) & (0.031) \\ 
  & & & & & \\ 
 Income Q4 & 0.077$^{**}$ & 0.045 & 0.032 & 0.035 & 0.078$^{**}$ \\ 
  & (0.033) & (0.032) & (0.035) & (0.032) & (0.032) \\ 
  & & & & & \\ 
 age: 25-34 & 0.096$^{**}$ & 0.086$^{**}$ & 0.027 & 0.002 & 0.010 \\ 
  & (0.039) & (0.039) & (0.042) & (0.038) & (0.038) \\ 
  & & & & & \\ 
 age: 35-49 & 0.056 & 0.082$^{**}$ & 0.035 & $-$0.015 & 0.019 \\ 
  & (0.040) & (0.039) & (0.042) & (0.039) & (0.038) \\ 
  & & & & & \\ 
 age: 50-64 & $-$0.013 & $-$0.010 & $-$0.076$^{*}$ & $-$0.125$^{***}$ & $-$0.142$^{***}$ \\ 
  & (0.042) & (0.041) & (0.045) & (0.041) & (0.041) \\ 
  & & & & & \\ 
 age: 65+ & $-$0.034 & $-$0.059 & $-$0.160$^{***}$ & $-$0.192$^{***}$ & $-$0.195$^{***}$ \\ 
  & (0.050) & (0.050) & (0.054) & (0.049) & (0.049) \\ 
  & & & & & \\ 
 vote: Biden & 0.228$^{***}$ & 0.232$^{***}$ & 0.101$^{***}$ & 0.171$^{***}$ & 0.215$^{***}$ \\ 
  & (0.034) & (0.033) & (0.036) & (0.033) & (0.033) \\ 
  & & & & & \\ 
 vote: Trump & 0.019 & 0.013 & $-$0.081$^{**}$ & 0.009 & 0.020 \\ 
  & (0.036) & (0.036) & (0.039) & (0.035) & (0.035) \\ 
  & & & & & \\ 
 Climate treatment only & $-$0.013 & 0.006 & 0.040 & 0.022 & $-$0.034 \\ 
  & (0.027) & (0.027) & (0.029) & (0.026) & (0.026) \\ 
  & & & & & \\ 
 Policy treatment only & 0.108$^{***}$ & 0.070$^{***}$ & $-$0.021 & 0.042 & 0.028 \\ 
  & (0.027) & (0.026) & (0.028) & (0.026) & (0.026) \\ 
  & & & & & \\ 
 Both treatments & 0.130$^{***}$ & 0.066$^{**}$ & $-$0.053$^{*}$ & 0.052$^{*}$ & 0.047$^{*}$ \\ 
  & (0.028) & (0.027) & (0.030) & (0.027) & (0.027) \\ 
  & & & & & \\ 
\hline \\[-1.8ex] 

Observations & 2,010 & 2,010 & 2,010 & 2,010 & 2,010 \\ 
\hline 
\hline \\[-1.8ex] 
\end{tabular} }
	\end{center}
	{\footnotesize Note: The dependent variables are indicator variables equal to one if the respondent thinks the category would ``Mostly win' or ``Win a lot''' from a green infrastructure program. For instance, the variable \textit{Poorest} equals one if the respondent thinks the poorest would ``Mostly win'' or ``Win a lot'' if such a policy was implemented. See notes under Table \ref{table polviews} and Table \ref{table exists} for a description of the covariates.
	\newline *p$<$0.1; **p$<$0.05; ***p$<$0.01}
\end{table}	

\begin{table}[h!]
	\caption{Perceived losers of a green investments policy}
	\begin{center}
		\scalebox{0.7}{
\begin{tabular}{@{\extracolsep{5pt}}lcccccc} 
\\[-1.8ex]\hline 
\hline \\[-1.8ex] 
 & \multicolumn{6}{c}{Losers of green investments} \\ 
\cline{2-7} 
\\[-1.8ex] & Poorest & Middle class & Richest & Urban & Rural & Own household \\ 
\hline \\[-1.8ex] 
 Control group mean & 0.322 & 0.347 & 0.161 & 0.229 & 0.263 & 0.22  \\ \hline \\[-1.8ex] race: White only & $-$0.029 & $-$0.024 & 0.048 & $-$0.086$^{**}$ & $-$0.011 & $-$0.026 \\ 
  & (0.045) & (0.045) & (0.043) & (0.043) & (0.045) & (0.041) \\ 
  & & & & & & \\ 
 Male & $-$0.013 & $-$0.002 & $-$0.009 & 0.025 & $-$0.032 & $-$0.035 \\ 
  & (0.040) & (0.040) & (0.038) & (0.038) & (0.039) & (0.036) \\ 
  & & & & & & \\ 
 Children & $-$0.027 & $-$0.004 & 0.021 & 0.022 & $-$0.026 & 0.016 \\ 
  & (0.042) & (0.042) & (0.040) & (0.040) & (0.042) & (0.038) \\ 
  & & & & & & \\ 
 No college & $-$0.004 & 0.035 & $-$0.003 & $-$0.027 & 0.046 & $-$0.040 \\ 
  & (0.047) & (0.047) & (0.044) & (0.044) & (0.047) & (0.042) \\ 
  & & & & & & \\ 
 status: Retired & $-$0.089 & $-$0.216$^{***}$ & 0.024 & $-$0.196$^{***}$ & $-$0.102 & $-$0.141$^{**}$ \\ 
  & (0.076) & (0.076) & (0.072) & (0.072) & (0.076) & (0.069) \\ 
  & & & & & & \\ 
 status: Student & 0.147 & 0.403$^{***}$ & $-$0.078 & 0.297$^{**}$ & 0.108 & 0.152 \\ 
  & (0.121) & (0.121) & (0.115) & (0.115) & (0.121) & (0.110) \\ 
  & & & & & & \\ 
 staths: Working & $-$0.021 & $-$0.125$^{*}$ & 0.107$^{*}$ & $-$0.048 & $-$0.014 & $-$0.058 \\ 
  & (0.065) & (0.065) & (0.061) & (0.061) & (0.065) & (0.059) \\ 
  & & & & & & \\ 
 Income Q2 & 0.052 & 0.105$^{*}$ & 0.096$^{*}$ & 0.076 & 0.128$^{**}$ & 0.102$^{*}$ \\ 
  & (0.060) & (0.060) & (0.057) & (0.057) & (0.060) & (0.055) \\ 
  & & & & & & \\ 
 Income Q3 & 0.083 & 0.084 & 0.083 & 0.036 & 0.118$^{*}$ & 0.084 \\ 
  & (0.063) & (0.063) & (0.060) & (0.060) & (0.063) & (0.057) \\ 
  & & & & & & \\ 
 Income Q4 & 0.094 & 0.198$^{***}$ & 0.084 & 0.091 & 0.203$^{***}$ & 0.114$^{**}$ \\ 
  & (0.063) & (0.063) & (0.060) & (0.060) & (0.063) & (0.057) \\ 
  & & & & & & \\ 
 age: 30-49 & 0.009 & 0.144$^{**}$ & $-$0.078 & $-$0.007 & 0.053 & 0.059 \\ 
  & (0.062) & (0.062) & (0.058) & (0.058) & (0.061) & (0.056) \\ 
  & & & & & & \\ 
 age: 50-87 & 0.081 & 0.270$^{***}$ & $-$0.008 & 0.098 & 0.140$^{**}$ & 0.189$^{***}$ \\ 
  & (0.068) & (0.068) & (0.064) & (0.064) & (0.068) & (0.062) \\ 
  & & & & & & \\ 
 vote: Biden & $-$0.008 & $-$0.018 & $-$0.093$^{*}$ & 0.019 & $-$0.089 & $-$0.086$^{*}$ \\ 
  & (0.057) & (0.057) & (0.054) & (0.054) & (0.057) & (0.052) \\ 
  & & & & & & \\ 
 vote: Trump & 0.311$^{***}$ & 0.272$^{***}$ & 0.034 & 0.258$^{***}$ & 0.175$^{***}$ & 0.252$^{***}$ \\ 
  & (0.061) & (0.061) & (0.058) & (0.058) & (0.061) & (0.055) \\ 
  & & & & & & \\ 
 Climate treatment only & $-$0.124$^{**}$ & $-$0.160$^{***}$ & $-$0.017 & $-$0.091$^{*}$ & $-$0.082 & $-$0.113$^{**}$ \\ 
  & (0.056) & (0.056) & (0.053) & (0.053) & (0.056) & (0.051) \\ 
  & & & & & & \\ 
 Policy treatment only & $-$0.167$^{***}$ & $-$0.134$^{**}$ & 0.100$^{**}$ & $-$0.059 & $-$0.030 & 0.011 \\ 
  & (0.054) & (0.054) & (0.051) & (0.051) & (0.054) & (0.049) \\ 
  & & & & & & \\ 
 Both treatments & $-$0.096$^{*}$ & $-$0.164$^{***}$ & 0.050 & $-$0.072 & $-$0.092$^{*}$ & $-$0.110$^{**}$ \\ 
  & (0.056) & (0.056) & (0.053) & (0.053) & (0.055) & (0.050) \\ 
  & & & & & & \\ 
 wave: Pilote 2 & 0.025 & 0.034 & $-$0.049 & $-$0.084$^{**}$ & 0.016 & 0.024 \\ 
  & (0.039) & (0.039) & (0.037) & (0.037) & (0.039) & (0.036) \\ 
  & & & & & & \\ 
\hline \\[-1.8ex] 

Observations & 499 & 499 & 499 & 499 & 499 & 499 \\ 
\hline 
\hline \\[-1.8ex] 
\end{tabular} }
	\end{center}
	{\footnotesize Note: The dependent variables are indicator variables equal to one if the respondent thinks the category would ``Mostly lose' or ``Lose a lot''' from a green infrastructure program. For instance, the variable \textit{Poorest} equals one if the respondent thinks the poorest would ``Mostly lose'' or ``Lose a lot'' if such a policy was implemented. See notes under Table \ref{table polviews} and Table \ref{table exists} for a description of the covariates.
	\newline *p$<$0.1; **p$<$0.05; ***p$<$0.01}
\end{table}	


\begin{table}[h!]
	\caption{Perception of a green investments policy}
	\begin{center}
		\scalebox{0.7}{
\begin{tabular}{@{\extracolsep{5pt}}lcc} 
\\[-1.8ex]\hline 
\hline \\[-1.8ex] 
\\[-1.8ex] & Fair & Support \\ 
\hline \\[-1.8ex] 
 Control group mean & 0.505 & 0.52  \\ \hline \\[-1.8ex] race: White only & 0.012 & 0.046$^{*}$ \\ 
  & (0.024) & (0.024) \\ 
  & & \\ 
 Male & $-$0.027 & $-$0.011 \\ 
  & (0.021) & (0.021) \\ 
  & & \\ 
 Children & 0.004 & $-$0.0004 \\ 
  & (0.023) & (0.023) \\ 
  & & \\ 
 No college & $-$0.042$^{*}$ & $-$0.112$^{***}$ \\ 
  & (0.024) & (0.024) \\ 
  & & \\ 
 status: Retired & 0.022 & 0.020 \\ 
  & (0.031) & (0.031) \\ 
  & & \\ 
 status: Student & 0.078$^{**}$ & 0.071$^{**}$ \\ 
  & (0.033) & (0.033) \\ 
  & & \\ 
 status: Working & 0.086$^{**}$ & 0.074$^{**}$ \\ 
  & (0.034) & (0.034) \\ 
  & & \\ 
 Income Q2 & 0.020 & 0.039 \\ 
  & (0.040) & (0.041) \\ 
  & & \\ 
 Income Q3 & $-$0.029 & $-$0.032 \\ 
  & (0.041) & (0.041) \\ 
  & & \\ 
 Income Q4 & $-$0.095$^{**}$ & $-$0.058 \\ 
  & (0.043) & (0.044) \\ 
  & & \\ 
 age: 25-34 & $-$0.096$^{*}$ & $-$0.064 \\ 
  & (0.052) & (0.052) \\ 
  & & \\ 
 age: 35-49 & 0.375$^{***}$ & 0.285$^{***}$ \\ 
  & (0.035) & (0.035) \\ 
  & & \\ 
 age: 50-64 & $-$0.038 & $-$0.126$^{***}$ \\ 
  & (0.037) & (0.038) \\ 
  & & \\ 
 age: 64+ & $-$0.017 & $-$0.029 \\ 
  & (0.028) & (0.028) \\ 
  & & \\ 
 vote: Biden & 0.022 & 0.021 \\ 
  & (0.027) & (0.028) \\ 
  & & \\ 
 vote: Trump & 0.042 & 0.051$^{*}$ \\ 
  & (0.029) & (0.029) \\ 
  & & \\ 
\hline \\[-1.8ex] 

Observations & 2,010 & 2,010 \\ 
\hline 
\hline \\[-1.8ex] 
\end{tabular} }
	\end{center}
	{\footnotesize Note: The dependent variables are indicator variables equal to one if the respondent perceives the category as losers of a green infrastructure program. For instance, the variable \textit{Poorest} equals one if the respondent thinks the poorest would lose if such a policy was implemented. See notes under Table \ref{table polviews} and Table \ref{table exists} for a description of the covariates.
	\newline *p$<$0.1; **p$<$0.05; ***p$<$0.01}
\end{table}	

\begin{table}[h!]
	\caption{Funding preferences for a green investments policy}
	\begin{center}
		\scalebox{0.7}{
\begin{tabular}{@{\extracolsep{5pt}}lccccc} 
\\[-1.8ex]\hline 
\hline \\[-1.8ex] 
 & \multicolumn{5}{c}{Appropriate source of funding} \\ 
\cline{2-6} 
\\[-1.8ex] & Public debt & Sales tax & Wealth tax & Reduce social spending & Reduce military spending \\ 
\hline \\[-1.8ex] 
 Control group mean & 0.239 & 0.216 & 0.631 & 0.301 & 0.298  \\ \hline \\[-1.8ex] race: White only & 0.015 & $-$0.046$^{**}$ & 0.019 & 0.027 & 0.021 \\ 
  & (0.022) & (0.023) & (0.024) & (0.024) & (0.022) \\ 
  & & & & & \\ 
 Male & 0.064$^{***}$ & 0.055$^{***}$ & $-$0.054$^{**}$ & 0.011 & 0.024 \\ 
  & (0.020) & (0.020) & (0.022) & (0.021) & (0.019) \\ 
  & & & & & \\ 
 Children & 0.043$^{**}$ & 0.040$^{*}$ & $-$0.062$^{***}$ & 0.007 & $-$0.080$^{***}$ \\ 
  & (0.021) & (0.021) & (0.023) & (0.022) & (0.021) \\ 
  & & & & & \\ 
 No college & 0.029 & $-$0.064$^{***}$ & $-$0.001 & $-$0.032 & $-$0.029 \\ 
  & (0.022) & (0.023) & (0.025) & (0.024) & (0.022) \\ 
  & & & & & \\ 
 status: Retired & 0.005 & 0.042 & $-$0.038 & $-$0.038 & $-$0.002 \\ 
  & (0.040) & (0.041) & (0.044) & (0.043) & (0.039) \\ 
  & & & & & \\ 
 status: Student & $-$0.052 & 0.053 & 0.038 & $-$0.130$^{**}$ & 0.155$^{***}$ \\ 
  & (0.056) & (0.057) & (0.062) & (0.060) & (0.055) \\ 
  & & & & & \\ 
 status: Working & 0.044 & 0.083$^{***}$ & $-$0.076$^{**}$ & $-$0.062$^{*}$ & $-$0.034 \\ 
  & (0.031) & (0.031) & (0.034) & (0.033) & (0.030) \\ 
  & & & & & \\ 
 Income Q2 & 0.049$^{*}$ & $-$0.007 & 0.027 & 0.049 & 0.025 \\ 
  & (0.029) & (0.030) & (0.032) & (0.031) & (0.028) \\ 
  & & & & & \\ 
 Income Q3 & 0.068$^{**}$ & 0.013 & $-$0.061$^{*}$ & 0.058$^{*}$ & 0.037 \\ 
  & (0.031) & (0.031) & (0.034) & (0.033) & (0.030) \\ 
  & & & & & \\ 
 Income Q4 & 0.059$^{*}$ & $-$0.004 & $-$0.092$^{***}$ & 0.104$^{***}$ & 0.035 \\ 
  & (0.031) & (0.032) & (0.035) & (0.034) & (0.031) \\ 
  & & & & & \\ 
 age: 25-34 & 0.042 & $-$0.007 & 0.088$^{**}$ & $-$0.062 & 0.056 \\ 
  & (0.037) & (0.038) & (0.041) & (0.040) & (0.037) \\ 
  & & & & & \\ 
 age: 35-49 & 0.002 & $-$0.0001 & 0.095$^{**}$ & $-$0.044 & $-$0.001 \\ 
  & (0.038) & (0.039) & (0.042) & (0.040) & (0.037) \\ 
  & & & & & \\ 
 age: 50-64 & 0.027 & $-$0.004 & 0.132$^{***}$ & 0.018 & 0.083$^{**}$ \\ 
  & (0.040) & (0.041) & (0.044) & (0.043) & (0.039) \\ 
  & & & & & \\ 
 age: 65+ & 0.097$^{**}$ & 0.020 & 0.133$^{**}$ & 0.001 & 0.066 \\ 
  & (0.048) & (0.049) & (0.053) & (0.051) & (0.047) \\ 
  & & & & & \\ 
 vote: Biden & 0.099$^{***}$ & 0.048 & 0.220$^{***}$ & $-$0.148$^{***}$ & 0.120$^{***}$ \\ 
  & (0.032) & (0.033) & (0.036) & (0.035) & (0.032) \\ 
  & & & & & \\ 
 vote: Trump & $-$0.011 & $-$0.040 & $-$0.130$^{***}$ & 0.099$^{***}$ & $-$0.061$^{*}$ \\ 
  & (0.035) & (0.035) & (0.038) & (0.037) & (0.034) \\ 
  & & & & & \\ 
 Climate treatment only & $-$0.014 & 0.022 & $-$0.005 & $-$0.010 & $-$0.019 \\ 
  & (0.026) & (0.026) & (0.028) & (0.028) & (0.025) \\ 
  & & & & & \\ 
 Policy treatment only & 0.015 & 0.064$^{**}$ & $-$0.038 & 0.023 & $-$0.092$^{***}$ \\ 
  & (0.025) & (0.026) & (0.028) & (0.027) & (0.025) \\ 
  & & & & & \\ 
 Both treatments & 0.005 & 0.073$^{***}$ & 0.039 & $-$0.046 & $-$0.099$^{***}$ \\ 
  & (0.026) & (0.027) & (0.029) & (0.028) & (0.026) \\ 
  & & & & & \\ 
\hline \\[-1.8ex] 

Observations & 2,010 & 2,010 & 2,010 & 2,010 & 2,010 \\ 
\hline 
\hline \\[-1.8ex] 
\end{tabular} }
	\end{center}
	{\footnotesize Note: The dependent variables are indicator variables equal to one if the respondent perceives the category as losers of a green infrastructure program. For instance, the variable \textit{Poorest} equals one if the respondent thinks the poorest would lose if such a policy was implemented. See notes under Table \ref{table polviews} and Table \ref{table exists} for a description of the covariates.
	\newline *p$<$0.1; **p$<$0.05; ***p$<$0.01}
\end{table}	



\clearpage
\subsection{Preferences 3: Tax and dividend}



\begin{table}[h!]
	\caption{Opinion on carbon tax with cash transfers}
	\begin{center}
		\scalebox{0.7}{
\begin{tabular}{@{\extracolsep{5pt}}lccccccc} 
\\[-1.8ex]\hline 
\hline \\[-1.8ex] 
 & \multicolumn{7}{c}{Effects of carbon tax with cash transfers} \\ 
\cline{2-8} 
\\[-1.8ex] & Drive less & Insulate more & Reduce fossil fuels & Reduce pollution & Negative effect & Large effect & Costly \\ 
\hline \\[-1.8ex] 
 Control group mean & 0.534 & 0.605 & 0.596 & 0.653 & 0.443 & 0.546 & 0.554  \\ \hline \\[-1.8ex] race: White only & $-$0.013 & 0.008 & 0.059$^{**}$ & 0.017 & 0.009 & 0.039 & $-$0.026 \\ 
  & (0.025) & (0.025) & (0.025) & (0.024) & (0.025) & (0.026) & (0.025) \\ 
  & & & & & & & \\ 
 Male & $-$0.006 & $-$0.055$^{**}$ & 0.003 & $-$0.011 & 0.110$^{***}$ & 0.075$^{***}$ & 0.040$^{*}$ \\ 
  & (0.023) & (0.022) & (0.022) & (0.021) & (0.023) & (0.023) & (0.023) \\ 
  & & & & & & & \\ 
 Children & 0.061$^{**}$ & 0.046$^{*}$ & 0.023 & 0.030 & 0.038 & 0.061$^{**}$ & 0.079$^{***}$ \\ 
  & (0.024) & (0.023) & (0.023) & (0.023) & (0.024) & (0.025) & (0.024) \\ 
  & & & & & & & \\ 
 No college & $-$0.078$^{***}$ & $-$0.051$^{**}$ & $-$0.065$^{***}$ & $-$0.023 & $-$0.019 & $-$0.026 & $-$0.035 \\ 
  & (0.026) & (0.025) & (0.025) & (0.024) & (0.026) & (0.026) & (0.026) \\ 
  & & & & & & & \\ 
 status: Retired & 0.023 & 0.053 & $-$0.013 & $-$0.013 & 0.128$^{***}$ & 0.073 & 0.072 \\ 
  & (0.046) & (0.045) & (0.045) & (0.043) & (0.046) & (0.047) & (0.046) \\ 
  & & & & & & & \\ 
 status: Student & 0.037 & 0.121$^{*}$ & 0.100 & 0.074 & $-$0.050 & $-$0.008 & $-$0.099 \\ 
  & (0.064) & (0.062) & (0.062) & (0.060) & (0.064) & (0.065) & (0.064) \\ 
  & & & & & & & \\ 
 status: Working & 0.055 & 0.087$^{**}$ & 0.028 & 0.023 & 0.039 & 0.017 & 0.055 \\ 
  & (0.035) & (0.034) & (0.034) & (0.033) & (0.035) & (0.036) & (0.035) \\ 
  & & & & & & & \\ 
 Income Q2 & $-$0.031 & 0.034 & 0.011 & 0.001 & $-$0.013 & $-$0.020 & 0.116$^{***}$ \\ 
  & (0.033) & (0.032) & (0.032) & (0.031) & (0.033) & (0.034) & (0.033) \\ 
  & & & & & & & \\ 
 Income Q3 & $-$0.016 & 0.034 & 0.013 & 0.007 & $-$0.009 & 0.005 & 0.094$^{***}$ \\ 
  & (0.035) & (0.034) & (0.034) & (0.033) & (0.035) & (0.036) & (0.035) \\ 
  & & & & & & & \\ 
 Income Q4 & 0.040 & 0.091$^{***}$ & 0.070$^{**}$ & 0.016 & 0.034 & 0.018 & 0.080$^{**}$ \\ 
  & (0.036) & (0.035) & (0.035) & (0.034) & (0.036) & (0.037) & (0.036) \\ 
  & & & & & & & \\ 
 age: 25-34 & 0.115$^{***}$ & 0.076$^{*}$ & 0.089$^{**}$ & 0.086$^{**}$ & $-$0.088$^{**}$ & 0.038 & 0.017 \\ 
  & (0.043) & (0.042) & (0.042) & (0.040) & (0.043) & (0.044) & (0.043) \\ 
  & & & & & & & \\ 
 age: 35-49 & 0.063 & 0.076$^{*}$ & 0.057 & 0.090$^{**}$ & $-$0.006 & 0.031 & 0.024 \\ 
  & (0.043) & (0.042) & (0.042) & (0.041) & (0.043) & (0.045) & (0.044) \\ 
  & & & & & & & \\ 
 age: 50-64 & 0.038 & 0.039 & 0.014 & 0.078$^{*}$ & $-$0.057 & 0.006 & 0.034 \\ 
  & (0.046) & (0.045) & (0.045) & (0.043) & (0.046) & (0.047) & (0.046) \\ 
  & & & & & & & \\ 
 age: 65+ & 0.016 & 0.075 & 0.057 & 0.101$^{*}$ & $-$0.133$^{**}$ & $-$0.031 & $-$0.025 \\ 
  & (0.055) & (0.054) & (0.054) & (0.052) & (0.055) & (0.057) & (0.055) \\ 
  & & & & & & & \\ 
 vote: Biden & 0.260$^{***}$ & 0.204$^{***}$ & 0.246$^{***}$ & 0.224$^{***}$ & $-$0.105$^{***}$ & 0.046 & $-$0.082$^{**}$ \\ 
  & (0.037) & (0.036) & (0.036) & (0.035) & (0.037) & (0.038) & (0.037) \\ 
  & & & & & & & \\ 
 vote: Trump & 0.045 & $-$0.051 & $-$0.035 & $-$0.058 & 0.156$^{***}$ & 0.138$^{***}$ & 0.135$^{***}$ \\ 
  & (0.040) & (0.038) & (0.038) & (0.037) & (0.039) & (0.041) & (0.040) \\ 
  & & & & & & & \\ 
 Climate treatment only & 0.043 & 0.033 & 0.022 & 0.035 & $-$0.058$^{*}$ & $-$0.047 & $-$0.046 \\ 
  & (0.030) & (0.029) & (0.029) & (0.028) & (0.030) & (0.030) & (0.030) \\ 
  & & & & & & & \\ 
 Policy treatment only & 0.058$^{**}$ & 0.062$^{**}$ & 0.069$^{**}$ & 0.067$^{**}$ & $-$0.021 & 0.059$^{**}$ & 0.011 \\ 
  & (0.029) & (0.028) & (0.028) & (0.027) & (0.029) & (0.030) & (0.029) \\ 
  & & & & & & & \\ 
 Both treatments & 0.068$^{**}$ & 0.073$^{**}$ & 0.068$^{**}$ & 0.084$^{***}$ & 0.024 & 0.018 & 0.004 \\ 
  & (0.030) & (0.030) & (0.030) & (0.029) & (0.030) & (0.031) & (0.030) \\ 
  & & & & & & & \\ 
\hline \\[-1.8ex] 

Observations & 2,010 & 2,010 & 2,010 & 2,010 & 2,010 & 2,010 & 2,010 \\ 
\hline 
\hline \\[-1.8ex] 
\end{tabular} }
	\end{center}
	{\footnotesize Note: The dependent variables are indicator variables equal to one if the respondent agrees with the proposition. For instance, \textit{Trust federal government} equals one if the respondent thinks she can trust the U.S. government to correctly implement a carbon tax with cash transfers. See notes under Table \ref{table polviews} and Table \ref{table exists} for a description of the covariates.
	\newline *p$<$0.1; **p$<$0.05; ***p$<$0.01}
\end{table}	

\begin{table}[h!]
	\caption{Perceived winners of a carbon tax with cash transfers policy}
	\begin{center}
		\scalebox{0.7}{
\begin{tabular}{@{\extracolsep{5pt}}lccccc} 
\\[-1.8ex]\hline 
\hline \\[-1.8ex] 
 & \multicolumn{5}{c}{Winners of carbon tax with cash transfers} \\ 
\cline{2-6} 
\\[-1.8ex] & Poorest & Middle class & Richest & Rural & Own household \\ 
\hline \\[-1.8ex] 
 Control group mean & 0.21 & 0.238 & 0.339 & 0.21 & 0.218  \\ \hline \\[-1.8ex] race: White only & 0.021 & 0.011 & 0.024 & $-$0.006 & 0.019 \\ 
  & (0.023) & (0.022) & (0.024) & (0.022) & (0.022) \\ 
  & & & & & \\ 
 Male & 0.057$^{***}$ & 0.069$^{***}$ & 0.031 & 0.060$^{***}$ & 0.053$^{***}$ \\ 
  & (0.021) & (0.020) & (0.022) & (0.019) & (0.020) \\ 
  & & & & & \\ 
 Children & 0.040$^{*}$ & 0.058$^{***}$ & 0.041$^{*}$ & 0.088$^{***}$ & 0.072$^{***}$ \\ 
  & (0.022) & (0.021) & (0.023) & (0.020) & (0.021) \\ 
  & & & & & \\ 
 No college & $-$0.020 & 0.002 & $-$0.074$^{***}$ & 0.005 & $-$0.026 \\ 
  & (0.023) & (0.023) & (0.025) & (0.022) & (0.022) \\ 
  & & & & & \\ 
 status: Retired & 0.025 & 0.016 & 0.095$^{**}$ & 0.037 & $-$0.041 \\ 
  & (0.042) & (0.041) & (0.044) & (0.039) & (0.040) \\ 
  & & & & & \\ 
 status: Student & 0.076 & $-$0.019 & $-$0.020 & $-$0.036 & $-$0.046 \\ 
  & (0.058) & (0.057) & (0.062) & (0.055) & (0.056) \\ 
  & & & & & \\ 
 status: Working & 0.055$^{*}$ & 0.050 & 0.051 & 0.065$^{**}$ & 0.016 \\ 
  & (0.032) & (0.031) & (0.034) & (0.030) & (0.031) \\ 
  & & & & & \\ 
 Income Q2 & 0.030 & 0.029 & $-$0.003 & 0.028 & 0.051$^{*}$ \\ 
  & (0.030) & (0.029) & (0.032) & (0.028) & (0.029) \\ 
  & & & & & \\ 
 Income Q3 & 0.066$^{**}$ & 0.026 & $-$0.018 & 0.037 & 0.095$^{***}$ \\ 
  & (0.032) & (0.031) & (0.034) & (0.030) & (0.031) \\ 
  & & & & & \\ 
 Income Q4 & 0.078$^{**}$ & 0.033 & $-$0.018 & 0.007 & 0.086$^{***}$ \\ 
  & (0.033) & (0.032) & (0.034) & (0.031) & (0.032) \\ 
  & & & & & \\ 
 age: 25-34 & 0.079$^{**}$ & 0.030 & 0.050 & $-$0.080$^{**}$ & 0.100$^{***}$ \\ 
  & (0.039) & (0.038) & (0.041) & (0.037) & (0.038) \\ 
  & & & & & \\ 
 age: 35-49 & 0.065 & 0.030 & 0.049 & $-$0.039 & 0.065$^{*}$ \\ 
  & (0.039) & (0.038) & (0.042) & (0.037) & (0.038) \\ 
  & & & & & \\ 
 age: 50-64 & $-$0.034 & $-$0.101$^{**}$ & $-$0.098$^{**}$ & $-$0.200$^{***}$ & $-$0.044 \\ 
  & (0.042) & (0.041) & (0.044) & (0.039) & (0.040) \\ 
  & & & & & \\ 
 age: 65+ & $-$0.028 & $-$0.138$^{***}$ & $-$0.168$^{***}$ & $-$0.226$^{***}$ & $-$0.094$^{*}$ \\ 
  & (0.050) & (0.049) & (0.053) & (0.047) & (0.048) \\ 
  & & & & & \\ 
 vote: Biden & 0.199$^{***}$ & 0.194$^{***}$ & 0.055 & 0.200$^{***}$ & 0.213$^{***}$ \\ 
  & (0.034) & (0.033) & (0.036) & (0.032) & (0.033) \\ 
  & & & & & \\ 
 vote: Trump & 0.034 & $-$0.009 & $-$0.063$^{*}$ & 0.064$^{*}$ & 0.023 \\ 
  & (0.036) & (0.035) & (0.038) & (0.034) & (0.035) \\ 
  & & & & & \\ 
 Climate treatment only & 0.024 & $-$0.011 & 0.024 & 0.005 & 0.036 \\ 
  & (0.027) & (0.026) & (0.028) & (0.025) & (0.026) \\ 
  & & & & & \\ 
 Policy treatment only & 0.174$^{***}$ & 0.083$^{***}$ & $-$0.008 & 0.065$^{***}$ & 0.090$^{***}$ \\ 
  & (0.026) & (0.026) & (0.028) & (0.025) & (0.025) \\ 
  & & & & & \\ 
 Both treatments & 0.225$^{***}$ & 0.119$^{***}$ & $-$0.066$^{**}$ & 0.079$^{***}$ & 0.116$^{***}$ \\ 
  & (0.028) & (0.027) & (0.029) & (0.026) & (0.027) \\ 
  & & & & & \\ 
\hline \\[-1.8ex] 

Observations & 2,010 & 2,010 & 2,010 & 2,010 & 2,010 \\ 
\hline 
\hline \\[-1.8ex] 
\end{tabular} }
	\end{center}
	{\footnotesize Note: The dependent variables are indicator variables equal to one if the respondent thinks the category would ``Mostly win' or ``Win a lot''' from a carbon tax with cash transfers. For instance, the variable \textit{Poorest} equals one if the respondent thinks the poorest would ``Mostly win'' or ``Win a lot'' if such a policy was implemented. See notes under Table \ref{table polviews} and Table \ref{table exists} for a description of the covariates.
	\newline *p$<$0.1; **p$<$0.05; ***p$<$0.01}
\end{table}	

\begin{table}[h!]
	\caption{Perceived losers of a carbon tax with cash transfers policy}
	\begin{center}
		\scalebox{0.7}{
\begin{tabular}{@{\extracolsep{5pt}}lccccc} 
\\[-1.8ex]\hline 
\hline \\[-1.8ex] 
 & \multicolumn{5}{c}{Losers of carbon tax with cash transfers} \\ 
\cline{2-6} 
\\[-1.8ex] & Poorest & Middle class & Richest & Rural & Own household \\ 
\hline \\[-1.8ex] 
 Control group mean & 0.791 & 0.762 & 0.663 & 0.79 & 0.783  \\ \hline \\[-1.8ex] race: White only & $-$0.018 & $-$0.003 & $-$0.017 & 0.009 & $-$0.020 \\ 
  & (0.023) & (0.023) & (0.024) & (0.022) & (0.022) \\ 
  & & & & & \\ 
 Female & 0.055$^{***}$ & 0.062$^{***}$ & 0.027 & 0.052$^{***}$ & 0.050$^{**}$ \\ 
  & (0.021) & (0.020) & (0.022) & (0.019) & (0.020) \\ 
  & & & & & \\ 
 Children & $-$0.047$^{**}$ & $-$0.060$^{***}$ & $-$0.033 & $-$0.095$^{***}$ & $-$0.078$^{***}$ \\ 
  & (0.022) & (0.022) & (0.023) & (0.021) & (0.021) \\ 
  & & & & & \\ 
 No college & 0.039$^{*}$ & 0.024 & 0.077$^{***}$ & 0.013 & 0.046$^{**}$ \\ 
  & (0.023) & (0.023) & (0.025) & (0.022) & (0.022) \\ 
  & & & & & \\ 
 status: Retired & $-$0.023 & $-$0.023 & $-$0.096$^{**}$ & $-$0.047 & 0.030 \\ 
  & (0.042) & (0.041) & (0.044) & (0.039) & (0.040) \\ 
  & & & & & \\ 
 status: Student & $-$0.054 & 0.017 & 0.026 & 0.041 & 0.058 \\ 
  & (0.059) & (0.057) & (0.062) & (0.055) & (0.056) \\ 
  & & & & & \\ 
 status: Working & $-$0.063$^{*}$ & $-$0.062$^{*}$ & $-$0.056$^{*}$ & $-$0.073$^{**}$ & $-$0.026 \\ 
  & (0.033) & (0.032) & (0.034) & (0.030) & (0.031) \\ 
  & & & & & \\ 
 Income Q2 & $-$0.029 & $-$0.031 & 0.003 & $-$0.025 & $-$0.057$^{**}$ \\ 
  & (0.030) & (0.030) & (0.032) & (0.028) & (0.029) \\ 
  & & & & & \\ 
 Income Q3 & $-$0.069$^{**}$ & $-$0.032 & 0.011 & $-$0.041 & $-$0.103$^{***}$ \\ 
  & (0.032) & (0.031) & (0.034) & (0.030) & (0.031) \\ 
  & & & & & \\ 
 Income Q4 & $-$0.087$^{***}$ & $-$0.041 & 0.012 & $-$0.012 & $-$0.099$^{***}$ \\ 
  & (0.033) & (0.032) & (0.035) & (0.031) & (0.032) \\ 
  & & & & & \\ 
 age: 25-34 & $-$0.072$^{*}$ & $-$0.013 & $-$0.044 & 0.072$^{*}$ & $-$0.089$^{**}$ \\ 
  & (0.039) & (0.038) & (0.041) & (0.037) & (0.038) \\ 
  & & & & & \\ 
 age: 35-49 & $-$0.059 & $-$0.015 & $-$0.041 & 0.032 & $-$0.055 \\ 
  & (0.040) & (0.039) & (0.042) & (0.037) & (0.038) \\ 
  & & & & & \\ 
 age: 50-64 & 0.047 & 0.126$^{***}$ & 0.113$^{**}$ & 0.205$^{***}$ & 0.061 \\ 
  & (0.042) & (0.041) & (0.044) & (0.040) & (0.041) \\ 
  & & & & & \\ 
 age: 65+ & 0.021 & 0.150$^{***}$ & 0.173$^{***}$ & 0.222$^{***}$ & 0.103$^{**}$ \\ 
  & (0.051) & (0.050) & (0.053) & (0.048) & (0.049) \\ 
  & & & & & \\ 
 Left or Very left & $-$0.088$^{***}$ & $-$0.107$^{***}$ & $-$0.034 & $-$0.091$^{***}$ & $-$0.130$^{***}$ \\ 
  & (0.024) & (0.024) & (0.025) & (0.023) & (0.023) \\ 
  & & & & & \\ 
 Right or Very right & 0.055$^{**}$ & 0.040 & 0.037 & 0.004 & 0.049$^{**}$ \\ 
  & (0.025) & (0.024) & (0.026) & (0.023) & (0.024) \\ 
  & & & & & \\ 
 Center &  &  &  &  &  \\ 
  &  &  &  &  &  \\ 
  & & & & & \\ 
 Climate treatment only & $-$0.009 & 0.032 & $-$0.016 & 0.012 & $-$0.017 \\ 
  & (0.027) & (0.026) & (0.028) & (0.025) & (0.026) \\ 
  & & & & & \\ 
 Policy treatment only & $-$0.169$^{***}$ & $-$0.072$^{***}$ & 0.011 & $-$0.058$^{**}$ & $-$0.080$^{***}$ \\ 
  & (0.027) & (0.026) & (0.028) & (0.025) & (0.026) \\ 
  & & & & & \\ 
 Both treatments & $-$0.207$^{***}$ & $-$0.095$^{***}$ & 0.075$^{**}$ & $-$0.062$^{**}$ & $-$0.094$^{***}$ \\ 
  & (0.028) & (0.027) & (0.029) & (0.026) & (0.027) \\ 
  & & & & & \\ 
\hline \\[-1.8ex] 

Observations & 2,010 & 2,010 & 2,010 & 2,010 & 2,010 \\ 
\hline 
\hline \\[-1.8ex] 
\end{tabular} }
	\end{center}
	{\footnotesize Note: The dependent variables are indicator variables equal to one if the respondent thinks the category would ``Mostly lose' or ``Lose a lot''' from a carbon tax with cash transfers. For instance, the variable \textit{Poorest} equals one if the respondent thinks the poorest would ``Mostly lose'' or ``Lose a lot'' if such a policy was implemented. See notes under Table \ref{table polviews} and Table \ref{table exists} for a description of the covariates.
	\newline *p$<$0.1; **p$<$0.05; ***p$<$0.01}
\end{table}	

\begin{table}[h!]
	\caption{Perception of a carbon tax with cash transfers policy}
	\begin{center}
		\scalebox{0.7}{
\begin{tabular}{@{\extracolsep{5pt}}lcc} 
\\[-1.8ex]\hline 
\hline \\[-1.8ex] 
\\[-1.8ex] & Fair & Support \\ 
\hline \\[-1.8ex] 
 Control group mean & 0.34 & 0.317  \\ \hline \\[-1.8ex] race: White only & 0.044$^{*}$ & 0.018 \\ 
  & (0.024) & (0.024) \\ 
  & & \\ 
 Female & $-$0.023 & $-$0.037$^{*}$ \\ 
  & (0.022) & (0.022) \\ 
  & & \\ 
 Children & 0.060$^{***}$ & 0.065$^{***}$ \\ 
  & (0.023) & (0.023) \\ 
  & & \\ 
 No college & $-$0.095$^{***}$ & $-$0.090$^{***}$ \\ 
  & (0.025) & (0.024) \\ 
  & & \\ 
 status: Retired & 0.030 & 0.024 \\ 
  & (0.044) & (0.044) \\ 
  & & \\ 
 status: Student & 0.030 & 0.031 \\ 
  & (0.062) & (0.061) \\ 
  & & \\ 
 status: Working & 0.063$^{*}$ & 0.063$^{*}$ \\ 
  & (0.034) & (0.034) \\ 
  & & \\ 
 Income Q2 & 0.031 & 0.003 \\ 
  & (0.032) & (0.032) \\ 
  & & \\ 
 Income Q3 & 0.030 & 0.026 \\ 
  & (0.034) & (0.034) \\ 
  & & \\ 
 Income Q4 & 0.056 & 0.023 \\ 
  & (0.035) & (0.035) \\ 
  & & \\ 
 age: 25-34 & 0.040 & 0.092$^{**}$ \\ 
  & (0.041) & (0.041) \\ 
  & & \\ 
 age: 35-49 & 0.036 & 0.111$^{***}$ \\ 
  & (0.042) & (0.042) \\ 
  & & \\ 
 age: 50-64 & $-$0.079$^{*}$ & $-$0.036 \\ 
  & (0.044) & (0.044) \\ 
  & & \\ 
 age: 65+ & $-$0.058 & 0.011 \\ 
  & (0.053) & (0.053) \\ 
  & & \\ 
 Left or Very left & 0.168$^{***}$ & 0.126$^{***}$ \\ 
  & (0.025) & (0.025) \\ 
  & & \\ 
 Right or Very right & $-$0.137$^{***}$ & $-$0.156$^{***}$ \\ 
  & (0.026) & (0.026) \\ 
  & & \\ 
 Center &  &  \\ 
  &  &  \\ 
  & & \\ 
 Climate treatment only & 0.001 & 0.024 \\ 
  & (0.028) & (0.028) \\ 
  & & \\ 
 Policy treatment only & 0.089$^{***}$ & 0.120$^{***}$ \\ 
  & (0.028) & (0.028) \\ 
  & & \\ 
 Both treatments & 0.067$^{**}$ & 0.105$^{***}$ \\ 
  & (0.029) & (0.029) \\ 
  & & \\ 
\hline \\[-1.8ex] 

Observations & 2,010 & 2,010 \\ 
\hline 
\hline \\[-1.8ex] 
\end{tabular} }
	\end{center}
	{\footnotesize Note: The dependent variables are indicator variables equal to one if the respondent perceives the category as losers of a carbon taxwith cash transfers. For instance, the variable \textit{Poorest} equals one if the respondent thinks the poorest would lose if such a policy was implemented. See notes under Table \ref{table polviews} and Table \ref{table exists} for a description of the covariates.
	\newline *p$<$0.1; **p$<$0.05; ***p$<$0.01}
\end{table}	



\clearpage
\subsection{Preferences on climate policies}



\begin{table}[h!]
	\caption{Support for climate policies}
	\begin{center}
		\scalebox{0.7}{
\begin{tabular}{@{\extracolsep{5pt}}lccccc} 
\\[-1.8ex]\hline 
\hline \\[-1.8ex] 
 & \multicolumn{5}{c}{Support} \\ 
\cline{2-6} 
\\[-1.8ex] & Tax on flying & Tax on fossil fuels & Ban polluting vehicles in city centers & Technology subsidies & Global climate fund \\ 
\hline \\[-1.8ex] 
 Control group mean & 0.328 & 0.348 & 0.495 & 0.577 & 0.481  \\ \hline \\[-1.8ex] race: White only & 0.058$^{**}$ & 0.045$^{*}$ & 0.035 & 0.084$^{***}$ & 0.052$^{**}$ \\ 
  & (0.024) & (0.024) & (0.025) & (0.025) & (0.025) \\ 
  & & & & & \\ 
 Female & $-$0.038$^{*}$ & $-$0.083$^{***}$ & 0.001 & $-$0.020 & 0.008 \\ 
  & (0.022) & (0.021) & (0.023) & (0.022) & (0.022) \\ 
  & & & & & \\ 
 Children & 0.062$^{***}$ & 0.038$^{*}$ & 0.014 & $-$0.005 & $-$0.004 \\ 
  & (0.023) & (0.023) & (0.024) & (0.024) & (0.024) \\ 
  & & & & & \\ 
 No college & $-$0.076$^{***}$ & $-$0.116$^{***}$ & $-$0.088$^{***}$ & $-$0.093$^{***}$ & $-$0.089$^{***}$ \\ 
  & (0.025) & (0.024) & (0.026) & (0.025) & (0.025) \\ 
  & & & & & \\ 
 status: Retired & 0.044 & 0.067 & 0.031 & $-$0.101$^{**}$ & 0.001 \\ 
  & (0.044) & (0.043) & (0.046) & (0.045) & (0.045) \\ 
  & & & & & \\ 
 status: Student & 0.108$^{*}$ & 0.181$^{***}$ & $-$0.014 & $-$0.006 & 0.227$^{***}$ \\ 
  & (0.062) & (0.060) & (0.064) & (0.063) & (0.063) \\ 
  & & & & & \\ 
 status: Working & 0.075$^{**}$ & 0.079$^{**}$ & 0.035 & $-$0.032 & 0.080$^{**}$ \\ 
  & (0.034) & (0.033) & (0.036) & (0.035) & (0.035) \\ 
  & & & & & \\ 
 Income Q2 & $-$0.041 & 0.010 & $-$0.005 & 0.032 & $-$0.007 \\ 
  & (0.032) & (0.031) & (0.033) & (0.033) & (0.033) \\ 
  & & & & & \\ 
 Income Q3 & $-$0.068$^{**}$ & $-$0.006 & 0.055 & 0.077$^{**}$ & $-$0.008 \\ 
  & (0.034) & (0.033) & (0.035) & (0.034) & (0.035) \\ 
  & & & & & \\ 
 Income Q4 & $-$0.086$^{**}$ & 0.039 & 0.072$^{**}$ & 0.138$^{***}$ & 0.066$^{*}$ \\ 
  & (0.035) & (0.034) & (0.036) & (0.035) & (0.035) \\ 
  & & & & & \\ 
 age: 25-34 & 0.073$^{*}$ & 0.081$^{**}$ & 0.092$^{**}$ & 0.100$^{**}$ & 0.127$^{***}$ \\ 
  & (0.041) & (0.040) & (0.043) & (0.042) & (0.042) \\ 
  & & & & & \\ 
 age: 35-49 & 0.064 & 0.033 & 0.119$^{***}$ & 0.025 & 0.052 \\ 
  & (0.042) & (0.041) & (0.044) & (0.043) & (0.043) \\ 
  & & & & & \\ 
 age: 50-64 & $-$0.057 & $-$0.049 & 0.094$^{**}$ & 0.058 & $-$0.061 \\ 
  & (0.044) & (0.043) & (0.046) & (0.045) & (0.045) \\ 
  & & & & & \\ 
 age: 65+ & $-$0.009 & $-$0.033 & 0.132$^{**}$ & 0.087 & 0.013 \\ 
  & (0.053) & (0.052) & (0.055) & (0.054) & (0.054) \\ 
  & & & & & \\ 
 Left or Very left & 0.112$^{***}$ & 0.193$^{***}$ & 0.095$^{***}$ & 0.137$^{***}$ & 0.113$^{***}$ \\ 
  & (0.025) & (0.025) & (0.026) & (0.026) & (0.026) \\ 
  & & & & & \\ 
 Right or Very right & $-$0.122$^{***}$ & $-$0.145$^{***}$ & $-$0.217$^{***}$ & $-$0.224$^{***}$ & $-$0.210$^{***}$ \\ 
  & (0.026) & (0.025) & (0.027) & (0.027) & (0.027) \\ 
  & & & & & \\ 
 Center &  &  &  &  &  \\ 
  &  &  &  &  &  \\ 
  & & & & & \\ 
 Climate treatment only & $-$0.022 & $-$0.027 & $-$0.004 & $-$0.009 & 0.029 \\ 
  & (0.028) & (0.028) & (0.030) & (0.029) & (0.029) \\ 
  & & & & & \\ 
 Policy treatment only & 0.088$^{***}$ & 0.056$^{**}$ & 0.062$^{**}$ & $-$0.046 & 0.0003 \\ 
  & (0.028) & (0.027) & (0.029) & (0.029) & (0.029) \\ 
  & & & & & \\ 
 Both treatments & 0.038 & 0.042 & 0.095$^{***}$ & $-$0.001 & 0.009 \\ 
  & (0.029) & (0.028) & (0.030) & (0.030) & (0.030) \\ 
  & & & & & \\ 
\hline \\[-1.8ex] 

Observations & 2,010 & 2,010 & 2,010 & 2,010 & 2,010 \\ 
\hline 
\hline \\[-1.8ex] 
\end{tabular} }
	\end{center}
	{\footnotesize Note: The dependent variables are indicator variables equal to one if the respondent ``absolutely supports" or ``somewhat supports" the policy. For instance, \textit{Tax on flying} equals one if the respondent supports a tax on flying. See notes under Table \ref{table polviews} and Table \ref{table exists} for a description of the covariates.
	\newline *p$<$0.1; **p$<$0.05; ***p$<$0.01}
\end{table}	

\begin{landscape}
	\begin{table}[h!]
	\caption{Support carbon tax, depending on the use of revenues}
	\begin{center}
		\scalebox{0.55}{
\begin{tabular}{@{\extracolsep{5pt}}lccccccccc} 
\\[-1.8ex]\hline 
\hline \\[-1.8ex] 
 & \multicolumn{9}{c}{Support carbon tax if revenues allocated as/to…} \\ 
\cline{2-10} 
\\[-1.8ex] & Transfer to constrained HH & Transfers to poorest & Equal transfers & Tax rebates for affected firms & Reduce CIT & Reduce PIT & Infrastructure projects & Technology subsidies & Reduce deficit \\ 
\hline \\[-1.8ex] 
 Control group mean & 0.452 & 0.452 & 0.363 & 0.387 & 0.293 & 0.466 & 0.575 & 0.539 & 0.474  \\ \hline \\[-1.8ex] race: White only & 0.019 & 0.066$^{***}$ & $-$0.019 & 0.025 & $-$0.002 & 0.029 & 0.072$^{***}$ & 0.041$^{*}$ & $-$0.015 \\ 
  & (0.025) & (0.024) & (0.025) & (0.025) & (0.024) & (0.026) & (0.024) & (0.025) & (0.026) \\ 
  & & & & & & & & & \\ 
 Male & 0.022 & $-$0.023 & 0.017 & 0.009 & 0.053$^{**}$ & 0.007 & $-$0.012 & 0.004 & 0.038$^{*}$ \\ 
  & (0.022) & (0.022) & (0.022) & (0.022) & (0.021) & (0.023) & (0.021) & (0.022) & (0.023) \\ 
  & & & & & & & & & \\ 
 Children & $-$0.008 & 0.046$^{**}$ & 0.044$^{*}$ & 0.073$^{***}$ & 0.083$^{***}$ & 0.028 & 0.011 & 0.015 & 0.028 \\ 
  & (0.024) & (0.023) & (0.023) & (0.024) & (0.022) & (0.025) & (0.023) & (0.023) & (0.025) \\ 
  & & & & & & & & & \\ 
 No college & 0.010 & $-$0.034 & $-$0.015 & $-$0.016 & $-$0.062$^{**}$ & $-$0.027 & $-$0.102$^{***}$ & $-$0.101$^{***}$ & $-$0.101$^{***}$ \\ 
  & (0.025) & (0.025) & (0.025) & (0.025) & (0.024) & (0.026) & (0.024) & (0.025) & (0.026) \\ 
  & & & & & & & & & \\ 
 status: Retired & 0.061 & $-$0.038 & $-$0.049 & $-$0.104$^{**}$ & $-$0.039 & $-$0.052 & 0.030 & 0.046 & 0.066 \\ 
  & (0.045) & (0.044) & (0.045) & (0.045) & (0.043) & (0.047) & (0.044) & (0.044) & (0.047) \\ 
  & & & & & & & & & \\ 
 status: Student & 0.117$^{*}$ & 0.233$^{***}$ & 0.037 & 0.112$^{*}$ & $-$0.038 & 0.127$^{*}$ & 0.165$^{***}$ & 0.110$^{*}$ & 0.088 \\ 
  & (0.063) & (0.062) & (0.063) & (0.063) & (0.060) & (0.066) & (0.061) & (0.062) & (0.066) \\ 
  & & & & & & & & & \\ 
 status: Working & 0.029 & 0.038 & $-$0.002 & 0.032 & 0.014 & 0.011 & 0.056$^{*}$ & 0.003 & $-$0.011 \\ 
  & (0.035) & (0.034) & (0.034) & (0.035) & (0.033) & (0.036) & (0.033) & (0.034) & (0.036) \\ 
  & & & & & & & & & \\ 
 Income Q2 & $-$0.0004 & $-$0.045 & 0.051 & 0.016 & 0.090$^{***}$ & 0.078$^{**}$ & 0.014 & $-$0.018 & $-$0.012 \\ 
  & (0.033) & (0.032) & (0.032) & (0.033) & (0.031) & (0.034) & (0.031) & (0.032) & (0.034) \\ 
  & & & & & & & & & \\ 
 Income Q3 & 0.001 & $-$0.068$^{**}$ & 0.036 & 0.097$^{***}$ & 0.114$^{***}$ & 0.124$^{***}$ & 0.003 & 0.042 & 0.045 \\ 
  & (0.035) & (0.034) & (0.034) & (0.035) & (0.033) & (0.036) & (0.033) & (0.034) & (0.036) \\ 
  & & & & & & & & & \\ 
 Income Q4 & 0.023 & $-$0.105$^{***}$ & 0.073$^{**}$ & 0.054 & 0.098$^{***}$ & 0.106$^{***}$ & 0.035 & 0.065$^{*}$ & 0.095$^{**}$ \\ 
  & (0.035) & (0.035) & (0.035) & (0.036) & (0.034) & (0.037) & (0.034) & (0.035) & (0.037) \\ 
  & & & & & & & & & \\ 
 age: 25-34 & 0.200$^{***}$ & 0.168$^{***}$ & 0.144$^{***}$ & 0.088$^{**}$ & $-$0.021 & 0.109$^{**}$ & 0.155$^{***}$ & 0.146$^{***}$ & 0.030 \\ 
  & (0.042) & (0.041) & (0.042) & (0.042) & (0.040) & (0.044) & (0.041) & (0.041) & (0.044) \\ 
  & & & & & & & & & \\ 
 age: 35-49 & 0.150$^{***}$ & 0.091$^{**}$ & 0.062 & 0.126$^{***}$ & $-$0.062 & 0.115$^{***}$ & 0.141$^{***}$ & 0.086$^{**}$ & 0.057 \\ 
  & (0.043) & (0.042) & (0.042) & (0.043) & (0.041) & (0.045) & (0.041) & (0.042) & (0.045) \\ 
  & & & & & & & & & \\ 
 age: 50-64 & 0.033 & $-$0.012 & $-$0.029 & 0.017 & $-$0.179$^{***}$ & 0.053 & 0.083$^{*}$ & 0.038 & $-$0.027 \\ 
  & (0.045) & (0.044) & (0.045) & (0.045) & (0.043) & (0.047) & (0.044) & (0.044) & (0.047) \\ 
  & & & & & & & & & \\ 
 age: 65+ & $-$0.032 & $-$0.021 & $-$0.070 & 0.016 & $-$0.234$^{***}$ & 0.024 & 0.103$^{**}$ & 0.018 & $-$0.024 \\ 
  & (0.054) & (0.053) & (0.054) & (0.054) & (0.052) & (0.057) & (0.052) & (0.053) & (0.057) \\ 
  & & & & & & & & & \\ 
 vote: Biden & 0.241$^{***}$ & 0.307$^{***}$ & 0.253$^{***}$ & 0.203$^{***}$ & 0.127$^{***}$ & 0.131$^{***}$ & 0.279$^{***}$ & 0.232$^{***}$ & 0.115$^{***}$ \\ 
  & (0.037) & (0.036) & (0.036) & (0.037) & (0.035) & (0.038) & (0.035) & (0.036) & (0.038) \\ 
  & & & & & & & & & \\ 
 vote: Trump & $-$0.057 & $-$0.038 & 0.040 & 0.006 & 0.116$^{***}$ & 0.022 & $-$0.124$^{***}$ & $-$0.148$^{***}$ & 0.030 \\ 
  & (0.039) & (0.038) & (0.039) & (0.039) & (0.037) & (0.041) & (0.038) & (0.038) & (0.041) \\ 
  & & & & & & & & & \\ 
 Climate treatment only & 0.005 & 0.023 & $-$0.007 & 0.029 & 0.005 & 0.048 & $-$0.017 & 0.050$^{*}$ & 0.007 \\ 
  & (0.029) & (0.028) & (0.029) & (0.029) & (0.028) & (0.030) & (0.028) & (0.029) & (0.030) \\ 
  & & & & & & & & & \\ 
 Policy treatment only & 0.001 & 0.064$^{**}$ & 0.108$^{***}$ & 0.024 & 0.007 & 0.080$^{***}$ & 0.004 & 0.022 & 0.053$^{*}$ \\ 
  & (0.029) & (0.028) & (0.028) & (0.029) & (0.027) & (0.030) & (0.028) & (0.028) & (0.030) \\ 
  & & & & & & & & & \\ 
 Both treatments & 0.066$^{**}$ & 0.092$^{***}$ & 0.136$^{***}$ & 0.078$^{***}$ & 0.032 & 0.073$^{**}$ & 0.032 & 0.021 & 0.037 \\ 
  & (0.030) & (0.029) & (0.030) & (0.030) & (0.028) & (0.031) & (0.029) & (0.029) & (0.031) \\ 
  & & & & & & & & & \\ 
\hline \\[-1.8ex] 

Observations & 2,010 & 2,010 & 2,010 & 2,010 & 2,010 & 2,010 & 2,010 & 2,010 & 2,010 \\ 
\hline 
\hline \\[-1.8ex] 
\end{tabular} }
	\end{center}
	{\footnotesize Note: The dependent variables are indicator variables equal to one if the respondent ``Strongly supports" or ``Rather supports" the use of revenues from potential carbon taxes to finance the policy. For instance, the \textit{Transfer to constrained HH} variable equals one if the respondent supports the use of revenues from carbon taxes to finance ``Transfers to households with no alternative to using fossil fuels." \textit{Transfers to poorest} corresponds to ``Transfers to the poorest households", \textit{Equal transfers} to ``Equal cash transfers to all households", \textit{Tax rebates for affected firms} to ``Tax rebates for most affected firms", \textit{Infrastructures projects} to ``Funding environmental infrastructure projects", \textit{Technology subsidies} to ``Subsidizing low-carbon technologies, including renewable nergy", \textit{Reduce deficit} to ``A reduction in the public deficit", \textit{Reduce CIT} to ``A reduction of corporate income tax", and \textit{Reduce PIT} to ``A reduction of personal income tax." See notes under Table \ref{table polviews} and Table \ref{table exists} for a description of the covariates.
	\newline *p$<$0.1; **p$<$0.05; ***p$<$0.01}
\end{table}	
\end{landscape}



\clearpage
\subsection{WTP and Altruism}



\begin{table}[h!]
	\caption{Willingness to Pay}
	\begin{center}
		\scalebox{0.7}{
\begin{tabular}{@{\extracolsep{5pt}}lc} 
\\[-1.8ex]\hline 
\hline \\[-1.8ex] 
 & \multicolumn{1}{c}{WTP to limit global warming to safe levels} \\ 
\cline{2-2} 
\\[-1.8ex] & WTP \\ 
\hline \\[-1.8ex] 
 Control group mean & 0.541  \\ \hline \\[-1.8ex] race: White only & 0.057$^{**}$ \\ 
  & (0.024) \\ 
  & \\ 
 Male & 0.024 \\ 
  & (0.021) \\ 
  & \\ 
 Children & 0.036 \\ 
  & (0.022) \\ 
  & \\ 
 No college & $-$0.057$^{**}$ \\ 
  & (0.024) \\ 
  & \\ 
 status: Retired & 0.122$^{***}$ \\ 
  & (0.043) \\ 
  & \\ 
 status: Student & 0.079 \\ 
  & (0.059) \\ 
  & \\ 
 status: Working & 0.151$^{***}$ \\ 
  & (0.033) \\ 
  & \\ 
 Income Q2 & 0.030 \\ 
  & (0.031) \\ 
  & \\ 
 Income Q3 & 0.157$^{***}$ \\ 
  & (0.033) \\ 
  & \\ 
 Income Q4 & 0.180$^{***}$ \\ 
  & (0.033) \\ 
  & \\ 
 age: 25-34 & $-$0.025 \\ 
  & (0.040) \\ 
  & \\ 
 age: 35-49 & $-$0.070$^{*}$ \\ 
  & (0.040) \\ 
  & \\ 
 age: 50-64 & $-$0.187$^{***}$ \\ 
  & (0.043) \\ 
  & \\ 
 age: 65+ & $-$0.156$^{***}$ \\ 
  & (0.051) \\ 
  & \\ 
 vote: Biden & 0.251$^{***}$ \\ 
  & (0.034) \\ 
  & \\ 
 vote: Trump & $-$0.099$^{***}$ \\ 
  & (0.037) \\ 
  & \\ 
 Climate treatment only & 0.046$^{*}$ \\ 
  & (0.028) \\ 
  & \\ 
 Policy treatment only & $-$0.005 \\ 
  & (0.027) \\ 
  & \\ 
 Both treatments & 0.034 \\ 
  & (0.028) \\ 
  & \\ 
 WTP 30 & 0.051 \\ 
  & (0.036) \\ 
  & \\ 
 WTP 50 & $-$0.011 \\ 
  & (0.036) \\ 
  & \\ 
 WTP 100 & $-$0.009 \\ 
  & (0.037) \\ 
  & \\ 
 WTP 300 & $-$0.139$^{***}$ \\ 
  & (0.037) \\ 
  & \\ 
 WTP 500 & $-$0.164$^{***}$ \\ 
  & (0.037) \\ 
  & \\ 
 WTP 1000 & $-$0.242$^{***}$ \\ 
  & (0.036) \\ 
  & \\ 
\hline \\[-1.8ex] 

Observations & 2,010 \\ 
\hline 
\hline \\[-1.8ex] 
\end{tabular} }
	\end{center}
	{\footnotesize Note: The dependent variable is a continuous variable indicating the amount the respondent would be willing to pay annually to limit global warming to safe levels. See notes under Table \ref{table polviews} and Table \ref{table exists} for a description of the covariates.
	\newline *p$<$0.1; **p$<$0.05; ***p$<$0.01}
\end{table}	

\begin{table}[h!]
	\caption{Altruism}
	\begin{center}
		\scalebox{0.7}{
\begin{tabular}{@{\extracolsep{5pt}}lcc} 
\\[-1.8ex]\hline 
\hline \\[-1.8ex] 
 & \multicolumn{2}{c}{Altruism} \\ 
\cline{2-3} 
\\[-1.8ex] & Donation to charity \$ & Signed petition \\ 
\hline \\[-1.8ex] 
 Mean & 38.209 & 0  \\
Observations & 2,010 & 1,956 \\ 
\hline 
\hline \\[-1.8ex] 
\end{tabular} }
	\end{center}
	{\footnotesize Note: The dependent variable is a continuous variable indicating the amount the respondent would be willing to pay annually to limit global warming to safe levels. See notes under Table \ref{table polviews} and Table \ref{table exists} for a description of the covariates.
	\newline *p$<$0.1; **p$<$0.05; ***p$<$0.01}
\end{table}	



\clearpage
\subsection{International burden-sharing}



\begin{table}[h!]
	\caption{Best level to implement policies to tackle climate change}
	\begin{center}
		\scalebox{0.7}{
\begin{tabular}{@{\extracolsep{5pt}}lcccc} 
\\[-1.8ex]\hline 
\hline \\[-1.8ex] 
 & \multicolumn{4}{c}{Policy level} \\ 
\cline{2-5} 
\\[-1.8ex] & Local & State & Federal & Global \\ 
\hline \\[-1.8ex] 
 Mean & 0.303 & 0.446 & 0.41 & 0.508  \\
Observations & 191 & 191 & 191 & 191 \\ 
\hline 
\hline \\[-1.8ex] 
\end{tabular} }
	\end{center}
	{\footnotesize Note: The variables are indicator variables equal to one if the respondent thinks public policies to tackle climate change need to be put in place at this level.}
\end{table}	

\begin{table}[h!]
	\caption{How should the U.S. act}
	\begin{center}
		\scalebox{0.7}{
\begin{tabular}{@{\extracolsep{5pt}}lccc} 
\\[-1.8ex]\hline 
\hline \\[-1.8ex] 
 & \multicolumn{3}{c}{U.S. should… (if other countries do…)} \\ 
\cline{2-4} 
\\[-1.8ex] & U.S. more ambitious, if others less & U.S. more ambitious, if others as well & U.S. less ambitious, if others are \\ 
\hline \\[-1.8ex] 
 Mean & 0.522 & 0.403 & 0.075  \\ \hline \\[-1.8ex] race: White only & $-$0.004 & $-$0.034 & 0.038 \\ 
  & (0.104) & (0.108) & (0.059) \\ 
  & & & \\ 
 Male & 0.126 & $-$0.057 & $-$0.069 \\ 
  & (0.096) & (0.100) & (0.054) \\ 
  & & & \\ 
 Children & 0.071 & $-$0.009 & $-$0.063 \\ 
  & (0.094) & (0.098) & (0.053) \\ 
  & & & \\ 
 No college & 0.037 & $-$0.167 & 0.129$^{**}$ \\ 
  & (0.104) & (0.108) & (0.059) \\ 
  & & & \\ 
 status: Retired & 0.003 & $-$0.055 & 0.052 \\ 
  & (0.155) & (0.162) & (0.088) \\ 
  & & & \\ 
 status: Student & $-$0.460 & 0.563 & $-$0.103 \\ 
  & (0.546) & (0.569) & (0.309) \\ 
  & & & \\ 
 status: Working & 0.101 & $-$0.154 & 0.053 \\ 
  & (0.162) & (0.169) & (0.092) \\ 
  & & & \\ 
 Income Q2 & 0.035 & $-$0.003 & $-$0.032 \\ 
  & (0.139) & (0.145) & (0.079) \\ 
  & & & \\ 
 Income Q3 & 0.015 & 0.038 & $-$0.052 \\ 
  & (0.134) & (0.140) & (0.076) \\ 
  & & & \\ 
 Income Q4 & 0.127 & $-$0.072 & $-$0.055 \\ 
  & (0.139) & (0.145) & (0.079) \\ 
  & & & \\ 
 age: 30-49 & 0.037 & $-$0.089 & 0.053 \\ 
  & (0.248) & (0.258) & (0.140) \\ 
  & & & \\ 
 age: 50-87 & $-$0.220 & 0.198 & 0.022 \\ 
  & (0.255) & (0.265) & (0.144) \\ 
  & & & \\ 
 vote: Biden & 0.124 & $-$0.145 & 0.021 \\ 
  & (0.127) & (0.133) & (0.072) \\ 
  & & & \\ 
 vote: Trump & $-$0.197 & 0.022 & 0.175$^{**}$ \\ 
  & (0.144) & (0.151) & (0.082) \\ 
  & & & \\ 
 Constant & 0.388 & 0.606$^{*}$ & 0.006 \\ 
  & (0.298) & (0.310) & (0.169) \\ 
  & & & \\ 
\hline \\[-1.8ex] 

Observations & 133 & 133 & 133 \\ 
\hline 
\hline \\[-1.8ex] 
\end{tabular} }
	\end{center}
	{\footnotesize Note: The variables are indicator variables equal to one if the respondent thinks public policies to tackle climate change need to be put in place at this level.}
\end{table}	

\begin{landscape}
	\begin{table}[h!]
	\caption{Countries that should bear the costs}
	\begin{center}
		\scalebox{0.6}{
\begin{tabular}{@{\extracolsep{5pt}}lccccc} 
\\[-1.8ex]\hline 
\hline \\[-1.8ex] 
 & \multicolumn{5}{c}{Countries should} \\ 
\cline{2-6} 
\\[-1.8ex] & Pay in proportion to income & Pay in proportion to current emissions & Pay in proportion to past emissions (from 1990) & Richest pay alone & Richest pay, and even more to help vulnerable countries \\ 
\hline \\[-1.8ex] 
 Mean & 0.537 & 0.669 & 0.439 & 0.281 & 0.403  \\ \hline \\[-1.8ex] race: White only & 0.064$^{**}$ & $-$0.020 & 0.003 & $-$0.015 & 0.024 \\ 
  & (0.026) & (0.024) & (0.026) & (0.022) & (0.024) \\ 
  & & & & & \\ 
 Male & $-$0.008 & $-$0.006 & 0.022 & 0.083$^{***}$ & 0.031 \\ 
  & (0.023) & (0.022) & (0.023) & (0.020) & (0.022) \\ 
  & & & & & \\ 
 Children & 0.037 & 0.030 & 0.052$^{**}$ & 0.074$^{***}$ & 0.052$^{**}$ \\ 
  & (0.024) & (0.023) & (0.024) & (0.021) & (0.023) \\ 
  & & & & & \\ 
 No college & $-$0.061$^{**}$ & $-$0.109$^{***}$ & $-$0.126$^{***}$ & $-$0.024 & $-$0.085$^{***}$ \\ 
  & (0.026) & (0.024) & (0.026) & (0.023) & (0.024) \\ 
  & & & & & \\ 
 status: Retired & $-$0.048 & $-$0.001 & $-$0.095$^{**}$ & $-$0.064 & $-$0.074$^{*}$ \\ 
  & (0.046) & (0.044) & (0.047) & (0.041) & (0.044) \\ 
  & & & & & \\ 
 status: Student & 0.043 & 0.066 & 0.062 & 0.008 & 0.026 \\ 
  & (0.064) & (0.061) & (0.065) & (0.057) & (0.061) \\ 
  & & & & & \\ 
 status: Working & $-$0.003 & 0.026 & $-$0.064$^{*}$ & $-$0.051 & $-$0.041 \\ 
  & (0.035) & (0.034) & (0.036) & (0.031) & (0.034) \\ 
  & & & & & \\ 
 Income Q2 & $-$0.003 & 0.010 & 0.109$^{***}$ & $-$0.012 & $-$0.001 \\ 
  & (0.033) & (0.032) & (0.034) & (0.029) & (0.032) \\ 
  & & & & & \\ 
 Income Q3 & $-$0.013 & 0.049 & 0.079$^{**}$ & $-$0.039 & 0.002 \\ 
  & (0.035) & (0.034) & (0.036) & (0.031) & (0.034) \\ 
  & & & & & \\ 
 Income Q4 & 0.008 & 0.056 & 0.076$^{**}$ & $-$0.018 & 0.013 \\ 
  & (0.036) & (0.034) & (0.037) & (0.032) & (0.034) \\ 
  & & & & & \\ 
 age: 25-34 & 0.062 & 0.095$^{**}$ & 0.125$^{***}$ & 0.163$^{***}$ & 0.122$^{***}$ \\ 
  & (0.043) & (0.041) & (0.044) & (0.038) & (0.041) \\ 
  & & & & & \\ 
 age: 35-49 & 0.040 & 0.095$^{**}$ & 0.132$^{***}$ & 0.096$^{**}$ & $-$0.019 \\ 
  & (0.044) & (0.041) & (0.044) & (0.038) & (0.041) \\ 
  & & & & & \\ 
 age: 50-64 & $-$0.024 & 0.142$^{***}$ & 0.097$^{**}$ & $-$0.052 & $-$0.106$^{**}$ \\ 
  & (0.046) & (0.044) & (0.047) & (0.041) & (0.044) \\ 
  & & & & & \\ 
 age: 65+ & 0.051 & 0.228$^{***}$ & 0.085 & $-$0.143$^{***}$ & $-$0.150$^{***}$ \\ 
  & (0.056) & (0.053) & (0.056) & (0.049) & (0.053) \\ 
  & & & & & \\ 
 vote: Biden & 0.282$^{***}$ & 0.178$^{***}$ & 0.135$^{***}$ & 0.125$^{***}$ & 0.197$^{***}$ \\ 
  & (0.037) & (0.035) & (0.038) & (0.033) & (0.035) \\ 
  & & & & & \\ 
 vote: Trump & 0.028 & 0.016 & 0.023 & $-$0.062$^{*}$ & $-$0.108$^{***}$ \\ 
  & (0.040) & (0.038) & (0.040) & (0.035) & (0.038) \\ 
  & & & & & \\ 
 Climate treatment only & 0.032 & 0.001 & 0.032 & 0.003 & 0.036 \\ 
  & (0.030) & (0.028) & (0.030) & (0.026) & (0.028) \\ 
  & & & & & \\ 
 Policy treatment only & 0.055$^{*}$ & $-$0.011 & 0.059$^{**}$ & 0.012 & 0.002 \\ 
  & (0.029) & (0.028) & (0.030) & (0.026) & (0.028) \\ 
  & & & & & \\ 
 Both treatments & 0.057$^{*}$ & 0.004 & 0.056$^{*}$ & $-$0.013 & 0.052$^{*}$ \\ 
  & (0.031) & (0.029) & (0.031) & (0.027) & (0.029) \\ 
  & & & & & \\ 
\hline \\[-1.8ex] 

Observations & 2,010 & 2,010 & 2,010 & 2,010 & 2,010 \\ 
\hline 
\hline \\[-1.8ex] 
\end{tabular} }
	\end{center}
	{\footnotesize Note: The dependent variables are indicator variables equal to one if the respondent indicates to ``Strongly agree" or ``Somewhat agree" to the proposition regarding how countries should bear the costs of fighting climate change. For instance, \textit{Pay in proportion to income} equals one if the respondent agrees that all countries should pay in proportion to their income. See note under Table \ref{table polviews} for a description of the covariates.
	\newline *p$<$0.1; **p$<$0.05; ***p$<$0.01}
\end{table}	
\end{landscape}

\begin{landscape}
	\begin{table}[h!]
	\caption{International measures}
	\begin{center}
		\scalebox{0.6}{
\begin{tabular}{@{\extracolsep{5pt}}lccc} 
\\[-1.8ex]\hline 
\hline \\[-1.8ex] 
 & \multicolumn{3}{c}{Approve} \\ 
\cline{2-4} 
\\[-1.8ex] & Global democratic assembly to fight CC & Global tax on GHG emissions funding a global basic income (\textdollar 30/month/adult) & Global tax on top 1\% to finance poorest countries \\ 
\hline \\[-1.8ex] 
 Mean & 0.491 & 0.358 & 0.51  \\ \hline \\[-1.8ex] race: White only & 0.020 & 0.010 & 0.039 \\ 
  & (0.024) & (0.023) & (0.024) \\ 
  & & & \\ 
 Male & $-$0.017 & 0.073$^{***}$ & $-$0.016 \\ 
  & (0.022) & (0.021) & (0.022) \\ 
  & & & \\ 
 Children & 0.018 & 0.004 & 0.005 \\ 
  & (0.023) & (0.022) & (0.023) \\ 
  & & & \\ 
 No college & $-$0.028 & $-$0.067$^{***}$ & $-$0.033 \\ 
  & (0.025) & (0.024) & (0.024) \\ 
  & & & \\ 
 status: Retired & $-$0.023 & $-$0.026 & 0.023 \\ 
  & (0.044) & (0.042) & (0.044) \\ 
  & & & \\ 
 status: Student & $-$0.056 & 0.042 & 0.084 \\ 
  & (0.061) & (0.059) & (0.061) \\ 
  & & & \\ 
 status: Working & 0.021 & 0.007 & 0.063$^{*}$ \\ 
  & (0.034) & (0.032) & (0.034) \\ 
  & & & \\ 
 Income Q2 & 0.022 & 0.020 & $-$0.008 \\ 
  & (0.032) & (0.030) & (0.031) \\ 
  & & & \\ 
 Income Q3 & 0.087$^{**}$ & 0.073$^{**}$ & 0.008 \\ 
  & (0.034) & (0.032) & (0.033) \\ 
  & & & \\ 
 Income Q4 & 0.114$^{***}$ & 0.056$^{*}$ & $-$0.032 \\ 
  & (0.035) & (0.033) & (0.034) \\ 
  & & & \\ 
 age: 25-34 & 0.148$^{***}$ & 0.030 & 0.089$^{**}$ \\ 
  & (0.041) & (0.040) & (0.041) \\ 
  & & & \\ 
 age: 35-49 & 0.143$^{***}$ & 0.097$^{**}$ & 0.108$^{***}$ \\ 
  & (0.042) & (0.040) & (0.041) \\ 
  & & & \\ 
 age: 50-64 & 0.029 & $-$0.047 & $-$0.0002 \\ 
  & (0.044) & (0.042) & (0.044) \\ 
  & & & \\ 
 age: 65+ & 0.004 & $-$0.077 & $-$0.008 \\ 
  & (0.053) & (0.051) & (0.053) \\ 
  & & & \\ 
 vote: Biden & 0.291$^{***}$ & 0.275$^{***}$ & 0.329$^{***}$ \\ 
  & (0.036) & (0.034) & (0.035) \\ 
  & & & \\ 
 vote: Trump & $-$0.085$^{**}$ & $-$0.062$^{*}$ & $-$0.084$^{**}$ \\ 
  & (0.038) & (0.036) & (0.038) \\ 
  & & & \\ 
 Climate treatment only & 0.054$^{*}$ & $-$0.002 & $-$0.013 \\ 
  & (0.028) & (0.027) & (0.028) \\ 
  & & & \\ 
 Policy treatment only & 0.013 & 0.060$^{**}$ & $-$0.040 \\ 
  & (0.028) & (0.027) & (0.028) \\ 
  & & & \\ 
 Both treatments & 0.066$^{**}$ & 0.042 & 0.025 \\ 
  & (0.029) & (0.028) & (0.029) \\ 
  & & & \\ 
\hline \\[-1.8ex] 

Observations & 2,010 & 2,010 & 2,010 \\ 
\hline 
\hline \\[-1.8ex] 
\end{tabular} }
	\end{center}
	{\footnotesize Note: The dependent variables are indicator variables equal to one if the respondent approves the proposition. For instance, \textit{Global democratic assembly to fight CC} equals one if the respondent approves of ``establishing a global democratic assembly which role would be to take action against climate change." See note under Table \ref{table polviews} for a description of the covariates.
	\newline *p$<$0.1; **p$<$0.05; ***p$<$0.01}
\end{table}	
\end{landscape}



\clearpage
\subsection{Housing/Preference for bans vs. incentives}



\begin{table}[h!]
	\caption{Willingness to insulate}
	\begin{center}
		\scalebox{0.7}{
\begin{tabular}{@{\extracolsep{5pt}}lc} 
\\[-1.8ex]\hline 
\hline \\[-1.8ex] 
\\[-1.8ex] & Likely to insulate \\ 
\hline \\[-1.8ex] 
 Control group mean & 0.374  \\ \hline \\[-1.8ex] race: White only & 0.010 \\ 
  & (0.023) \\ 
  & \\ 
 Male & 0.041$^{**}$ \\ 
  & (0.020) \\ 
  & \\ 
 Children & 0.079$^{***}$ \\ 
  & (0.021) \\ 
  & \\ 
 No college & $-$0.071$^{***}$ \\ 
  & (0.023) \\ 
  & \\ 
 status: Retired & 0.018 \\ 
  & (0.029) \\ 
  & \\ 
 status: Student & 0.093$^{***}$ \\ 
  & (0.032) \\ 
  & \\ 
 status: Working & 0.088$^{***}$ \\ 
  & (0.033) \\ 
  & \\ 
 Income Q2 & 0.114$^{***}$ \\ 
  & (0.038) \\ 
  & \\ 
 Income Q3 & 0.084$^{**}$ \\ 
  & (0.039) \\ 
  & \\ 
 Income Q4 & 0.042 \\ 
  & (0.041) \\ 
  & \\ 
 age: 25-34 & $-$0.038 \\ 
  & (0.050) \\ 
  & \\ 
 age: 35-49 & 0.017 \\ 
  & (0.033) \\ 
  & \\ 
 age: 50-64 & $-$0.139$^{***}$ \\ 
  & (0.035) \\ 
  & \\ 
 age: 64+ & 0.029 \\ 
  & (0.026) \\ 
  & \\ 
 vote: Biden & $-$0.024 \\ 
  & (0.026) \\ 
  & \\ 
 vote: Trump & 0.038 \\ 
  & (0.027) \\ 
  & \\ 
 Climate treatment only & 0.298$^{***}$ \\ 
  & (0.036) \\ 
  & \\ 
 Policy treatment only & 0.384$^{***}$ \\ 
  & (0.022) \\ 
  & \\ 
 Both treatments & 0.220$^{***}$ \\ 
  & (0.031) \\ 
  & \\ 
 Insulation: not my choice & $-$0.019 \\ 
  & (0.034) \\ 
  & \\ 
 Insulation: cost & 0.164$^{***}$ \\ 
  & (0.024) \\ 
  & \\ 
\hline \\[-1.8ex] 

Observations & 2,010 \\ 
\hline 
\hline \\[-1.8ex] 
\end{tabular} }
	\end{center}
	{\footnotesize Note: The dependent variables correspond to indicator variables. For instance, the \textit{made mandatory} variable equals one if the respondent thinks that if the U.S. government would subsidize the thermal renovation of residential housing, it should made it mandatory. See notes under Table \ref{table polviews} and Table \ref{table exists} for a description of the covariates.
	\newline *p$<$0.1; **p$<$0.05; ***p$<$0.01}
\end{table}	

\begin{landscape}
	\begin{table}[h!]
	\caption{Mandatory insulation}
	\begin{center}
		\scalebox{0.6}{
\begin{tabular}{@{\extracolsep{5pt}}lc} 
\\[-1.8ex]\hline 
\hline \\[-1.8ex] 
\\[-1.8ex] & Support thermal renovation if subsidized \\ 
\hline \\[-1.8ex] 
 Control group mean & 0.532  \\ \hline \\[-1.8ex] race: White only & 0.022 \\ 
  & (0.025) \\ 
  & \\ 
 Female & 0.006 \\ 
  & (0.023) \\ 
  & \\ 
 Children & 0.046$^{*}$ \\ 
  & (0.024) \\ 
  & \\ 
 No college & $-$0.094$^{***}$ \\ 
  & (0.026) \\ 
  & \\ 
 status: Retired & 0.055 \\ 
  & (0.046) \\ 
  & \\ 
 status: Student & 0.031 \\ 
  & (0.064) \\ 
  & \\ 
 status: Working & 0.049 \\ 
  & (0.036) \\ 
  & \\ 
 Income Q2 & $-$0.001 \\ 
  & (0.033) \\ 
  & \\ 
 Income Q3 & 0.033 \\ 
  & (0.035) \\ 
  & \\ 
 Income Q4 & 0.077$^{**}$ \\ 
  & (0.036) \\ 
  & \\ 
 age: 25-34 & 0.165$^{***}$ \\ 
  & (0.043) \\ 
  & \\ 
 age: 35-49 & 0.161$^{***}$ \\ 
  & (0.044) \\ 
  & \\ 
 age: 50-64 & 0.057 \\ 
  & (0.046) \\ 
  & \\ 
 age: 65+ & $-$0.006 \\ 
  & (0.055) \\ 
  & \\ 
 Left or Very left & 0.079$^{***}$ \\ 
  & (0.026) \\ 
  & \\ 
 Right or Very right & $-$0.175$^{***}$ \\ 
  & (0.027) \\ 
  & \\ 
 Center &  \\ 
  &  \\ 
  & \\ 
 Climate treatment only & $-$0.031 \\ 
  & (0.030) \\ 
  & \\ 
 Policy treatment only & $-$0.051$^{*}$ \\ 
  & (0.029) \\ 
  & \\ 
 Both treatments & $-$0.025 \\ 
  & (0.030) \\ 
  & \\ 
 Formulation: Costs underlined & $-$0.054$^{**}$ \\ 
  & (0.022) \\ 
  & \\ 
\hline \\[-1.8ex] 

Observations & 2,010 \\ 
\hline 
\hline \\[-1.8ex] 
\end{tabular} }
	\end{center}
	{\footnotesize Note: The dependent variables are indicator variables equal to one. The \textit{Rationing} variables equal one if the respondent thinks no one should be allowed to fly more than the quota in brackets between now and 2040. The \textit{Tradable} variables equal one of the respondent thinks people should be able to trade their rights to fly. The quota used to frame the question is randomly selected from three different options. The \textit{(1000km)} variables refer to respondents who are asked about a quota of 1000km/person/year, the \textit{(3000km)} variables to respondents asked about a quota of 3000km/person/year, and the \textit{(0.5 round-trip/year)} to respondents asked about a quota of 1 round-trip/person/2years.  See notes under Table \ref{table polviews} and Table \ref{table exists} for a description of the covariates.
	\newline *p$<$0.1; **p$<$0.05; ***p$<$0.01}
\end{table}	
\end{landscape}

\begin{table}[h!]
	\caption{Cattle consumption restrictions enforcement}
	\begin{center}
		\scalebox{0.7}{
\begin{tabular}{@{\extracolsep{5pt}}lcccc} 
\\[-1.8ex]\hline 
\hline \\[-1.8ex] 
 & \multicolumn{4}{c}{If gov. limits cattle products, I would support…} \\ 
\cline{2-5} 
\\[-1.8ex] & Tax on cattle products (beefx2) & Subsidies Vegetables & No subsidies cattle & Ban intensive cattle \\ 
\hline \\[-1.8ex] 
 Control group mean & 0.331 & 0.444 & 0.409 & 0.371  \\ \hline \\[-1.8ex] race: White only & 0.011 & 0.022 & $-$0.031 & $-$0.015 \\ 
  & (0.023) & (0.025) & (0.025) & (0.024) \\ 
  & & & & \\ 
 Male & $-$0.012 & $-$0.004 & 0.034 & $-$0.017 \\ 
  & (0.020) & (0.022) & (0.022) & (0.021) \\ 
  & & & & \\ 
 Children & 0.006 & 0.040$^{*}$ & 0.005 & 0.002 \\ 
  & (0.021) & (0.023) & (0.023) & (0.023) \\ 
  & & & & \\ 
 No college & $-$0.108$^{***}$ & $-$0.087$^{***}$ & $-$0.101$^{***}$ & $-$0.090$^{***}$ \\ 
  & (0.023) & (0.025) & (0.025) & (0.024) \\ 
  & & & & \\ 
 status: Retired & 0.041 & 0.030 & 0.056 & 0.029 \\ 
  & (0.041) & (0.045) & (0.045) & (0.043) \\ 
  & & & & \\ 
 status: Student & $-$0.069 & 0.101 & 0.032 & $-$0.040 \\ 
  & (0.057) & (0.062) & (0.062) & (0.061) \\ 
  & & & & \\ 
 status: Working & 0.037 & 0.076$^{**}$ & 0.038 & 0.058$^{*}$ \\ 
  & (0.031) & (0.034) & (0.034) & (0.033) \\ 
  & & & & \\ 
 Income Q2 & 0.026 & 0.043 & 0.047 & 0.010 \\ 
  & (0.030) & (0.032) & (0.032) & (0.031) \\ 
  & & & & \\ 
 Income Q3 & 0.084$^{***}$ & 0.079$^{**}$ & 0.148$^{***}$ & 0.049 \\ 
  & (0.031) & (0.034) & (0.034) & (0.033) \\ 
  & & & & \\ 
 Income Q4 & 0.098$^{***}$ & 0.082$^{**}$ & 0.173$^{***}$ & 0.084$^{**}$ \\ 
  & (0.032) & (0.035) & (0.035) & (0.034) \\ 
  & & & & \\ 
 age: 25-34 & $-$0.018 & 0.103$^{**}$ & $-$0.015 & 0.001 \\ 
  & (0.038) & (0.042) & (0.042) & (0.041) \\ 
  & & & & \\ 
 age: 35-49 & $-$0.045 & 0.114$^{***}$ & 0.022 & $-$0.023 \\ 
  & (0.039) & (0.042) & (0.042) & (0.041) \\ 
  & & & & \\ 
 age: 50-64 & $-$0.172$^{***}$ & $-$0.029 & $-$0.065 & $-$0.142$^{***}$ \\ 
  & (0.041) & (0.045) & (0.045) & (0.044) \\ 
  & & & & \\ 
 age: 65+ & $-$0.218$^{***}$ & $-$0.122$^{**}$ & $-$0.062 & $-$0.181$^{***}$ \\ 
  & (0.049) & (0.054) & (0.054) & (0.052) \\ 
  & & & & \\ 
 vote: Biden & 0.249$^{***}$ & 0.209$^{***}$ & 0.192$^{***}$ & 0.235$^{***}$ \\ 
  & (0.033) & (0.036) & (0.036) & (0.035) \\ 
  & & & & \\ 
 vote: Trump & 0.006 & $-$0.026 & $-$0.058 & $-$0.035 \\ 
  & (0.035) & (0.038) & (0.038) & (0.037) \\ 
  & & & & \\ 
 Climate treatment only & $-$0.048$^{*}$ & 0.006 & 0.008 & 0.017 \\ 
  & (0.026) & (0.029) & (0.029) & (0.028) \\ 
  & & & & \\ 
 Policy treatment only & $-$0.030 & $-$0.046 & 0.015 & 0.013 \\ 
  & (0.026) & (0.028) & (0.028) & (0.028) \\ 
  & & & & \\ 
 Both treatments & 0.007 & $-$0.007 & 0.021 & 0.045 \\ 
  & (0.027) & (0.030) & (0.030) & (0.029) \\ 
  & & & & \\ 
\hline \\[-1.8ex] 

Observations & 2,010 & 2,010 & 2,010 & 2,010 \\ 
\hline 
\hline \\[-1.8ex] 
\end{tabular} }
	\end{center}
	{\footnotesize Note: The dependent variables are indicator variables equal to one if the respondent would approve the measure in a scenario where the U.S. government decides to limit the consumption of cattle products. The \textit{Tax on cattle products (beefx2)} refers to ``A high tax on cattle products, so that the price of beef doubles", the \textit{Sub. Vegetables} variable to ``Subsidies on organic and local vegetables, fruits and nuts", the \textit{No sub. cattle} variable to ``The removal of subsidies for cattle farming", and the \textit{Ban intensive cattle} to ``The ban of intensive cattle farming." See notes under Table \ref{table polviews} and Table \ref{table exists} for a description of the covariates.
	\newline *p$<$0.1; **p$<$0.05; ***p$<$0.01}
\end{table}	



\clearpage
\subsection{Trust, perceptions of institution, inequality, and the future}



\begin{table}[h!]
	\caption{Trust in government and others}
	\begin{center}
		\scalebox{0.7}{
\begin{tabular}{@{\extracolsep{5pt}}lcc} 
\\[-1.8ex]\hline 
\hline \\[-1.8ex] 
 & \multicolumn{2}{c}{Trust…} \\ 
\cline{2-3} 
\\[-1.8ex] & most people & government to do what is right \\ 
\hline \\[-1.8ex] 
 Mean & 0.411 & 0.231  \\ \hline \\[-1.8ex] race: White only & 0.053$^{**}$ & 0.028 \\ 
  & (0.025) & (0.021) \\ 
  & & \\ 
 Female & $-$0.074$^{***}$ & $-$0.093$^{***}$ \\ 
  & (0.023) & (0.019) \\ 
  & & \\ 
 Children & 0.071$^{***}$ & 0.115$^{***}$ \\ 
  & (0.024) & (0.020) \\ 
  & & \\ 
 No college & $-$0.105$^{***}$ & $-$0.100$^{***}$ \\ 
  & (0.025) & (0.021) \\ 
  & & \\ 
 status: Retired & 0.026 & $-$0.005 \\ 
  & (0.046) & (0.038) \\ 
  & & \\ 
 status: Student & $-$0.007 & 0.008 \\ 
  & (0.064) & (0.053) \\ 
  & & \\ 
 status: Working & 0.030 & 0.042 \\ 
  & (0.035) & (0.029) \\ 
  & & \\ 
 Income Q2 & 0.065$^{**}$ & 0.006 \\ 
  & (0.033) & (0.027) \\ 
  & & \\ 
 Income Q3 & 0.056 & 0.010 \\ 
  & (0.035) & (0.029) \\ 
  & & \\ 
 Income Q4 & 0.138$^{***}$ & 0.030 \\ 
  & (0.036) & (0.030) \\ 
  & & \\ 
 age: 25-34 & 0.046 & 0.053 \\ 
  & (0.043) & (0.036) \\ 
  & & \\ 
 age: 35-49 & 0.077$^{*}$ & 0.062$^{*}$ \\ 
  & (0.043) & (0.036) \\ 
  & & \\ 
 age: 50-64 & $-$0.042 & $-$0.155$^{***}$ \\ 
  & (0.046) & (0.038) \\ 
  & & \\ 
 age: 65+ & 0.054 & $-$0.170$^{***}$ \\ 
  & (0.055) & (0.046) \\ 
  & & \\ 
 Left or Very left & 0.072$^{***}$ & 0.055$^{**}$ \\ 
  & (0.026) & (0.022) \\ 
  & & \\ 
 Right or Very right & 0.015 & 0.025 \\ 
  & (0.027) & (0.022) \\ 
  & & \\ 
 Center &  &  \\ 
  &  &  \\ 
  & & \\ 
 Climate treatment only & 0.005 & 0.015 \\ 
  & (0.029) & (0.024) \\ 
  & & \\ 
 Policy treatment only & 0.007 & 0.006 \\ 
  & (0.029) & (0.024) \\ 
  & & \\ 
 Both treatments & 0.042 & 0.013 \\ 
  & (0.030) & (0.025) \\ 
  & & \\ 
\hline \\[-1.8ex] 

Observations & 2,010 & 2,010 \\ 
\hline 
\hline \\[-1.8ex] 
\end{tabular} }
	\end{center}
	{\footnotesize Note: The dependent variables are indicator variables. The \textit{most people} variable equals one if the respondent assigns a score greather than 5, on a scale from 0 to 10, to the question asking about trusting other people (0: ``One needs to be careful", 5: ``Most people can be trusted"). The \textit{government to do what is right} variable equals one if the respondent indicates trusting the U.S. government to do what is right ``Nearly all the time" or "Most of the time." The \textit{government to spend revenue wisely} variable equals one if the respondent indicates to ``fully agree" or ``somewhat agree" that authorities spend the revenue obtained from taxes and fees in a sensible way.
	See note under Table \ref{table polviews} for a description of the covariates.
		\newline *p$<$0.1; **p$<$0.05; ***p$<$0.01}	
\end{table}	

\begin{table}[h!]
	\caption{Intervention, inequality and future}
	\begin{center}
		\scalebox{0.7}{
\begin{tabular}{@{\extracolsep{5pt}}lccc} 
\\[-1.8ex]\hline 
\hline \\[-1.8ex] 
\\[-1.8ex] & Active government & Inequality serious problem & World poorer or same \\ 
\hline \\[-1.8ex] 
 Mean & 0.468 & 0.53 & 0.754  \\ \hline \\[-1.8ex] race: White only & $-$0.029 & 0.021 & $-$0.039$^{*}$ \\ 
  & (0.025) & (0.025) & (0.022) \\ 
  & & & \\ 
 Female & 0.056$^{**}$ & 0.046$^{**}$ & 0.115$^{***}$ \\ 
  & (0.022) & (0.022) & (0.020) \\ 
  & & & \\ 
 Children & $-$0.039 & $-$0.025 & $-$0.037$^{*}$ \\ 
  & (0.024) & (0.023) & (0.021) \\ 
  & & & \\ 
 No college & $-$0.011 & $-$0.046$^{*}$ & 0.015 \\ 
  & (0.025) & (0.025) & (0.022) \\ 
  & & & \\ 
 status: Retired & $-$0.015 & 0.043 & $-$0.085$^{**}$ \\ 
  & (0.046) & (0.045) & (0.040) \\ 
  & & & \\ 
 status: Student & 0.035 & $-$0.071 & 0.044 \\ 
  & (0.064) & (0.062) & (0.055) \\ 
  & & & \\ 
 status: Working & $-$0.045 & 0.032 & $-$0.053$^{*}$ \\ 
  & (0.035) & (0.035) & (0.031) \\ 
  & & & \\ 
 Income Q2 & $-$0.055$^{*}$ & $-$0.013 & $-$0.016 \\ 
  & (0.033) & (0.032) & (0.029) \\ 
  & & & \\ 
 Income Q3 & $-$0.035 & $-$0.025 & $-$0.042 \\ 
  & (0.035) & (0.034) & (0.030) \\ 
  & & & \\ 
 Income Q4 & 0.006 & $-$0.021 & $-$0.142$^{***}$ \\ 
  & (0.036) & (0.035) & (0.031) \\ 
  & & & \\ 
 age: 25-34 & 0.132$^{***}$ & 0.045 & $-$0.074$^{**}$ \\ 
  & (0.043) & (0.042) & (0.037) \\ 
  & & & \\ 
 age: 35-49 & 0.107$^{**}$ & 0.019 & 0.002 \\ 
  & (0.043) & (0.042) & (0.038) \\ 
  & & & \\ 
 age: 50-64 & 0.057 & $-$0.084$^{*}$ & 0.149$^{***}$ \\ 
  & (0.046) & (0.045) & (0.040) \\ 
  & & & \\ 
 age: 65+ & 0.082 & $-$0.059 & 0.214$^{***}$ \\ 
  & (0.055) & (0.054) & (0.048) \\ 
  & & & \\ 
 Left or Very left & 0.174$^{***}$ & 0.158$^{***}$ & $-$0.070$^{***}$ \\ 
  & (0.026) & (0.026) & (0.023) \\ 
  & & & \\ 
 Right or Very right & $-$0.184$^{***}$ & $-$0.254$^{***}$ & $-$0.071$^{***}$ \\ 
  & (0.027) & (0.026) & (0.023) \\ 
  & & & \\ 
 Center &  &  &  \\ 
  &  &  &  \\ 
  & & & \\ 
 Climate treatment only & $-$0.026 & $-$0.030 & 0.004 \\ 
  & (0.029) & (0.029) & (0.025) \\ 
  & & & \\ 
 Policy treatment only & $-$0.007 & 0.001 & $-$0.037 \\ 
  & (0.029) & (0.028) & (0.025) \\ 
  & & & \\ 
 Both treatments & 0.024 & $-$0.047 & $-$0.031 \\ 
  & (0.030) & (0.030) & (0.026) \\ 
  & & & \\ 
\hline \\[-1.8ex] 

Observations & 2,010 & 2,010 & 2,010 \\ 
\hline 
\hline \\[-1.8ex] 
\end{tabular} }
	\end{center}
	{\footnotesize Note: The dependent variables are indicator variables. The \textit{Active government} variable equals one if the respondent assigns a score greather than 3, on a scale from 1 to 5 asking about the purpose of government (1: ``Government should focus on most basic functions", 5: "Government should play an active role"). The \textit{Inequality serious problem} equals one if the respondent indicates that in the U.S. inequality is ``A serious problem" or ``A very serious problem." The \textit{World poorer or same} variable equals one if the respondent indicates that in 100 years the world will be ``About as rich as now on average" or ``Poorer."
	See note under Table \ref{table polviews} for a description of the covariates.
	\newline *p$<$0.1; **p$<$0.05; ***p$<$0.01}
\end{table}	



\clearpage
\subsection{Feedback}



\begin{table}[h!]
	\caption{Survey biased}
	\begin{center}
		\scalebox{0.7}{
\begin{tabular}{@{\extracolsep{5pt}}lccc} 
\\[-1.8ex]\hline 
\hline \\[-1.8ex] 
 & \multicolumn{3}{c}{Biased} \\ 
\cline{2-4} 
\\[-1.8ex] & No & Yes, anti environment & Yes, pro environment \\ 
\hline \\[-1.8ex] 
 Control group mean & 0.583 & 0.104 & 0.312  \\ \hline \\[-1.8ex] race: White only & $-$0.005 & $-$0.029 & 0.034 \\ 
  & (0.093) & (0.045) & (0.091) \\ 
  & & & \\ 
 Male & $-$0.095 & $-$0.012 & 0.107 \\ 
  & (0.080) & (0.039) & (0.079) \\ 
  & & & \\ 
 Children & $-$0.069 & 0.051 & 0.019 \\ 
  & (0.082) & (0.040) & (0.080) \\ 
  & & & \\ 
 No college & $-$0.097 & 0.051 & 0.046 \\ 
  & (0.091) & (0.045) & (0.090) \\ 
  & & & \\ 
 status: Retired & 0.157 & $-$0.019 & $-$0.138 \\ 
  & (0.145) & (0.071) & (0.143) \\ 
  & & & \\ 
 status: Student & 0.087 & 0.149 & $-$0.235 \\ 
  & (0.345) & (0.169) & (0.340) \\ 
  & & & \\ 
 staths: Working & 0.202 & 0.007 & $-$0.209 \\ 
  & (0.143) & (0.070) & (0.141) \\ 
  & & & \\ 
 Income Q2 & 0.096 & $-$0.059 & $-$0.037 \\ 
  & (0.122) & (0.060) & (0.120) \\ 
  & & & \\ 
 Income Q3 & $-$0.060 & $-$0.098$^{*}$ & 0.158 \\ 
  & (0.116) & (0.057) & (0.114) \\ 
  & & & \\ 
 Income Q4 & $-$0.118 & $-$0.110$^{*}$ & 0.227$^{*}$ \\ 
  & (0.123) & (0.060) & (0.121) \\ 
  & & & \\ 
 age: 30-49 & 0.258 & $-$0.138 & $-$0.120 \\ 
  & (0.199) & (0.097) & (0.196) \\ 
  & & & \\ 
 age: 50-87 & 0.406$^{**}$ & $-$0.174$^{*}$ & $-$0.233 \\ 
  & (0.202) & (0.099) & (0.199) \\ 
  & & & \\ 
 vote: Biden & $-$0.164 & 0.070 & 0.093 \\ 
  & (0.112) & (0.055) & (0.110) \\ 
  & & & \\ 
 vote: Trump & $-$0.340$^{***}$ & 0.039 & 0.301$^{**}$ \\ 
  & (0.122) & (0.060) & (0.120) \\ 
  & & & \\ 
 Both treatments & $-$0.009 & $-$0.104$^{*}$ & 0.113 \\ 
  & (0.115) & (0.056) & (0.113) \\ 
  & & & \\ 
 Climate treatment only & $-$0.174 & $-$0.114$^{**}$ & 0.288$^{***}$ \\ 
  & (0.107) & (0.052) & (0.106) \\ 
  & & & \\ 
 Policy treatment only & $-$0.236$^{**}$ & $-$0.026 & 0.263$^{***}$ \\ 
  & (0.099) & (0.048) & (0.098) \\ 
  & & & \\ 
 Constant & 0.486$^{**}$ & 0.279$^{**}$ & 0.235 \\ 
  & (0.240) & (0.117) & (0.236) \\ 
  & & & \\ 
\hline \\[-1.8ex] 

Observations & 191 & 191 & 191 \\ 
\hline 
\hline \\[-1.8ex] 
\end{tabular} }
	\end{center}
	{\footnotesize Note: The dependent variables are indicator variables. The \textit{No} variable equals one if the respondent does not feel that the survey was biased, the \textit{Yes, anti environment} variable equals one if the respondent feels the survey was biased towards environmental causes, the \textit{Yes, pro environment} equals one if the respondent feels the survey was biased against environment. See notes under Table \ref{table polviews} and Table \ref{table exists} for a description of the covariates.
	\newline *p$<$0.1; **p$<$0.05; ***p$<$0.01}
\end{table}	

\end{document}